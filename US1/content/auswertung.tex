% Messwerte: Alle gemessenen Größen tabellarisch darstellen
% Auswertung: Berechnung geforderter Ergebnisse mit Schritten/Fehlerformeln/Erläuterung/Grafik (Programme)
\section{Auswertung}
\label{sec:auswertung}

\subsection{Fehlerrechnung}
\label{sec:Fehlerrechnung}
Die Fehlerrechnung, für die Bestimmung der Messunsicherheiten, wird mit Uncertainties \cite{uncertainties} gemacht.
Die Formel der Gauß Fehlerfortpflanzung ist gegeben durch
\begin{equation}
    \Delta f=\sqrt{\sum_{i=1}^N\left(\frac{\partial f}{\partial x_i}\right)^2 \cdot\left(\Delta x_i\right)^2}.
    \label{eqn:gauss}
\end{equation}
Für den Mittelwert gilt 
\begin{equation}
    \bar{x} = \frac{1}{N}\sum\limits_{i = 1}^N x_i .
    \label{eqn:mittelwert}
\end{equation}
Der Fehler des Mittelwertes ist gegeben durch 
\begin{equation}
    \Delta \bar{x}=\frac{1}{\sqrt{N}} \sqrt{\frac{1}{N-1} \sum_{i=1}^N\left(x_i-\bar{x}\right)^2}.
    \label{eqn:mittelwertfehler}
\end{equation}

\subsection{Vermessung des Acrylblocks mit einer Schieblehre}
\label{Vermessung des Acrylblocks mit einer Schieblehre}

Der gegeben Acrylblock und dessen Bohrungen wurde wie in \autoref{fig:abmes} dargestellt, vermessen.
Der Block ist $\SI{80.4}{\milli\meter}$ breit, $\SI{150.3}{\milli\meter}$ lang und $\SI{40.3}{\milli\meter}$ tief.
Für die Abstände $a_k$ und $b_k$ und den Durchmesser $d$ der jeweiligen Bohrungen wurde ein Ablesefehler von $\SI{0.05}{\milli\meter}$
angenommen.
\begin{figure}[H]
    \centering
    \includegraphics[width=0.9\linewidth]{content/grafik/abmessung.jpg}
	\captionsetup{width=0.765\linewidth}
	\caption{Acrylblock mit Bohrungen und Maßen. \cite{scan}}
    \label{fig:abmes}
\end{figure}
Die Messdaten der Abstände $a_k$ und $b_k$ sind in \autoref{tab:lab} zu sehen. 
\begin{table}[H]
    \centering
    \caption{Abmessung der Löcher des Acrylblocks.}
    \label{tab:lab}
\begin{tabular}{c c c c}
    \toprule
    Loch & $a_k / \si{\milli\meter}$ & $b_k / \si{\milli\meter}$ & $d / \si{\milli\meter}$\\
    \midrule
    1 &   -  &    -   & 1.45 \\
    2 &   -  &    -   & 1.5 \\
    3 & 13.5 & 61.85 &    6  \\
    4 &21.85 &  54.4 &  4.9  \\
    5 &30.3 &  47.0 &    4  \\
    6 & 38.7 &  39.5 &  2.9  \\
    7 &46.8 &  31.0 &    3  \\
    8 &54.7 &  23.0 &  2.9  \\
    9 & 62.7 & 15.35 & 2.85  \\
    10 & 70.6 &   7.2 & 2.85  \\
    11 &15.2 &  55.8 &  9.5 \\
    \bottomrule
    \end{tabular}
\end{table}

\subsection{Untersuchung der Störstellen des Acrylblocks mit A-Scan}
\label{Untersuchung der Störstellen des Aceylblocks mit A-Scan}
Mit Hilfe des A-Scans wurden die Laufzeiten für die Löcher k = 3 bis k = 9 bestimmt und in der 
\autoref{tab:Laufzeit} aufgetragen. Es wurde dabei von oben gemessen und somit wurden die Laufzeiten für die
$a_k's$ aufgenommen. 
\begin{table}[H]
    \centering
    \caption{Laufzeit von Loch k=3 bis k=9.}
    \label{tab:Laufzeit}
\begin{tabular}{c c}
    \toprule
    Loch & $\text{Laufzeit } t / \si{\micro\second} $\\
    \midrule
     3 & 10.83 \\
     4 &  17.0 \\
     5 &  23.6 \\
     6 &  29.8 \\
     7 &  35.4 \\
     8 &  41.1 \\
     9 &  46.7 \\
    \bottomrule
\end{tabular}
\end{table}
Die Messwerte der Laufzeiten wurden in einem Plot dargestellt, wobei eine lineare Regression der Form 
\begin{equation*}
    a \cdot x + b
\end{equation*}
durchgeführt wurde. Der Plot ist in der \autoref{fig:plot1} dargestellt.

\begin{figure}[H]
	\includegraphics{build/plot1.pdf}
	\captionsetup{width=0.765\linewidth}
	\caption{Die Laufzeit des A-Scans gegen die Abstände $a_k$ aufgetragen und die lineare Ausgleichsgerade.}
	\label{fig:plot1}
\end{figure}

Für die Parameter der Ausgleichsfunktion gilt
\begin{align*}
    a &= \left(2708 \pm 23\right) \si{\meter \per \second}\\
    b &= \left(14 \pm 0.7\right) \si{\milli\meter}.
\end{align*}
Dabei entspricht $a$ der Steigung der Geraden, welche die Schallgeschwindigkeit $\zeta_{\text{exp}}$ in dem Material ist.
Der Literaturwert der Schallgeschwindigkeit $\zeta_{\text{theo}} = \SI{2730}{\meter\per\second}$ wird im weiteren Verlauf für die 
Messungen des A-Scans als auch des B-Scans verwendet.
Der y-Achsen-Abschnitt $b$ gibt die Dicke der Anpassungsschicht an. 

Die aufgenommenen Messdaten des A-Scans sind in der \autoref{tab:ascan} dargestellt.
\begin{table}[H]
    \centering
    \caption{Messwerte des A-Scan.}
    \label{tab:ascan}
\begin{tabular}{c c c}
    \toprule
    Loch & $\text{Laufzeit} t / \si{\micro\second} $ & $ d/ \si{mm}$\\
    \midrule
     3 & 14.7 & 63.0 \\
     4 & 23.5 & 55.4 \\
     5 & 32.1 & 47.8 \\
     6 & 40.8 & 40.4 \\
     7 & 48.4 & 32.2 \\
     8 & 56.2 & 24.4 \\
     9 & 64.2 & 16.8 \\
    10 & 73.1 &  8.7 \\
    11 & 17.2 & 56.8 \\
    \bottomrule
\end{tabular}
\end{table}

Um die Abstände, welche mittels Schieblehre bestimmt worden sind, mit den Abständen, welche über einen A-Scan aufgenommen worden sind,
vergleichen zu können, werden die Differenzen der jeweiligen $a_k$ und $b_k$ gemittelt.
Die Messwerte der Schieblehre sind in der \autoref{tab:lab} aufgelistet und die des A-Scans in der \autoref{tab:ascan}.
Für die Mittel der Differenzen ergeben sich die Werte
\begin{align*}
\bar{\increment a_k} &= \SI{1.761}{\milli\meter} \\
\bar{\increment b_k} &= \SI{3.719}{\milli\meter}.
\end{align*}

\subsection{Untersuchung der Störstellen des Acrylblocks mit B-Scan}
\label{sec:Untersuchung der Störstellen des Aceylblocks mit B-Scan}

Mittels des B-Scan wurden Aufnahmen von der oberen und der unteren Kante des Acryl-Blocks durchgeführt.
Diese sind in \autoref{fig:oben} und \autoref{fig:unten} zu sehen. Anhand des Cursors können die jeweiligen Störstellen 
lokalisiert werden.

\begin{figure}[H]
    \centering
	\includegraphics[width=0.8\linewidth]{data/US1_daten/b_scan_oben.jpg}
    \captionsetup{width=0.765\linewidth}
	\caption{Abbildung des Acrylblocks mit B-Scan.}
	\label{fig:oben}
\end{figure}

\begin{figure}[H]
    \centering
	\includegraphics[width=0.8\linewidth]{data/US1_daten/b_scan_u.jpg}
    \captionsetup{width=0.765\linewidth}
	\caption{Abbildung des Acrylblocks mit B-Scan.}
	\label{fig:unten}
\end{figure}

Die Messwerte des B-Scan sind in der \autoref{tab:bscan} aufgetragen. Für $k =10$ konnte kein Wert aufgenommen werden, daher wird dieser
im Folgenden ignoriert.
\begin{table}[H]
    \centering
    \caption{Messwerte des B-Scan.}
    \label{tab:bscan}
\begin{tabular}{c c c}
\toprule
Loch & $\text{Laufzeit} t / \si{\micro\second} $& $ d/ \si{mm}$\\
\midrule
 3 & 15.1 & 63.3 \\
 4 & 23.6 & 55.8 \\
 5 & 32.4 & 48.2 \\
 6 & 40.8 & 40.8 \\
 7 & 48.9 & 33.0 \\
 8 & 56.5 & 24.8 \\
 9 & 64.7 & 16.8 \\
10 & - &  9.0 \\
11 & 17.2 & 57.4 \\
\bottomrule
\end{tabular}
\end{table}

Für die Mittel der Differenzen ergeben sich die Werte
\begin{align*}
\bar{\increment a_k} &= \SI{0.539}{\milli\meter} \\
\bar{\increment b_k} &= \SI{3.719}{\milli\meter}.
\end{align*}

\subsection{Untersuchung des Auflösungsvermögens} % Auslösungsverfahrens} % (fold)
\label{sec:Untersuchung des Auslösungsverfahrens}

Die Löcher $k = 1$ und $k = 2$ wurden mittels A-Scan jeweils mit einer $\SI{1}{\mega\hertz}$ Sonde und mit einer
$\SI{2}{\mega\hertz}$ näher betrachet. Die dabei aufgenommen Messergebnisse sind in \autoref{fig:plot2} und \autoref{fig:plot3}
zu sehen.

\begin{figure}[H]
	\includegraphics{build/plot2.pdf}
	\captionsetup{width=0.765\linewidth}
	\caption{A-Scan für die Löcher $k = 1$ und $k = 2$ mit einer $\SI{1}{\mega\hertz}$ Sonde.}
	\label{fig:plot2}
\end{figure}
In der \autoref{fig:plot2} ist bei $d = \SI{20}{\milli\meter}$ nur ein breiter Peak zu sehen, anhand dessen können die beiden Löcher nicht unterschieden werden.

\begin{figure}[H]
	\includegraphics{build/plot3.pdf}
	\captionsetup{width=0.765\linewidth}
	\caption{A-Scan für die Löcher $k = 1$ und $k = 2$ mit einer $\SI{2}{\mega\hertz}$ Sonde.}
	\label{fig:plot3}
\end{figure}
An der Stelle $d = \SI{20}{\milli\meter}$ sind zwei eindeutige Peaks zu erkennen, welche den beiden Löchern $k = 1$ und $k = 2$
zugeordent werden können.
% subsection Untersuchung des Auslösungsverfahrens (end)

\subsection{Untersuchung des Brustmodells mittels B-Scan}
\label{sec:Untersuchung des Brustmodells mittels B-Scan}

Das gegebene Brustmodell wurde mittels B-Scan mehrmals untersucht, dabei waren die Messungen nicht alle gut, daher wurden 
die unbrauchbaren Bilder verworfen.

\begin{figure}[H]
    \centering
	\includegraphics[width=0.8\linewidth]{content/grafik/Tumor_2.pdf}
    \captionsetup{width=0.765\linewidth}
	\caption{Erster B-Scan der Brustmodells.}
	\label{fig:b1}
\end{figure}

\begin{figure}[H]
    \centering
	\includegraphics[width=0.8\linewidth]{content/grafik/Super_Tumor.pdf}
    \captionsetup{width=0.765\linewidth}
	\caption{Zweiter B-Scan der Brustmodells.}
	\label{fig:b2}
\end{figure}

Anhand der \autoref{fig:b2} kann gesagt werden, dass es sich bei den zwei oberen Arealen um die gesuchten
Störstellen handelt. Der rechte obere Fleck ist aufgrund der farblich gekennzeichneten höheren Dichte der Tumor. Der linke Bereich deutet auf eine 
Zyste hin, da diese kein Knoten, sondern ein Hohlraum ist und auf dem Bild diffuser erscheint.
