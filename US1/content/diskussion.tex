% Diskussion: Resultate mit Fehler/Genauigkeit zusammenstellen, Literaturwerte/Messmethoden/Ursachen vergleichen
% Literatur: Verwendete Literatur/Grafiken/Werte/Programme
% Anhang: Kopie der analog eingetragenen Messdaten
\section{Diskussion}
\label{sec:diskussion}

Die experimentell bestimmte Schallgeschwindigkeit $\zeta_{\text{exp}} = \left(2708 \pm 23\right) \si{\meter \per \second}$ weicht von dem 
theoretischen Wert $\zeta_{\text{theo}} = 2730 \si{\meter \per \second}$ um $0.8 \%$ ab. Damit kann der Theoriewert bestätigt werden.

Bei der Bestimmung der Abmessungen der Störstellen des Acrylblocks wurde zunächst eine Schieblehre verwendet. Die bestimmten Abstände
wurden dann mit einem A-Scan und mit einem B-Scan untersucht. Schließlich wurden die Messwerte mittels Schieblehre mit denen des A-Scan und des 
B-Scan verglichen, indem die Differenzen der Abstände gebildet worden sind und diese darauf gemittelt wurden. Die gemittelten Differenzen 
sind im \autoref{sec:Untersuchung der Störstellen des Aceylblocks mit B-Scan} zu finden.
Dabei fällt auf, dass die Differenz der $b_k$ größer ist, sowohl bei dem A-Scan als auch bei dem B-Scan.
Die Abweichungen kommen zustande, da bei dem B-Scan, die Abstände graphisch bestimmt worden sind. Dies führte wiederum zu zufälligen Fehlern, die beispielsweise
beim Ablesen passiert sind. Zusammenfassend weist der A-Scan eine höhere Präzision auf als der B-Scan.

Für die Untersuchung des Auflösungsvermögen wurde eine $\SI{1}{\mega\hertz}$ Sonde und eine $\SI{2}{\mega\hertz}$ Sonde verwendet. 
Es konnte festgestellt werden, dass die $\SI{2}{\mega\hertz}$ Sonde eine bessere Auflösung hergibt.
Damit kann die Theorie, dass eine höhere Frequenz der Sonde zu einem bessern Ergebnis führt, bestätigt werden.

Das Brustmodell konnte grob untersucht werden. Durch vorheriges Abtasten war die ungefähre Position der Gewebeknoten bestimmt.
Bei dem B-Scan wurde entlang einer gedachten Linie gemessen, dass führte dazu, dass die Bilder leicht verzehrt wirkten. Ohne medizinisches Wissen 
kann keine exakte Aussage über die genau Lage und die Art des Gewebes getroffen werden. Stattdessen wurde eine ungefähre Einschätzung anhand der Bilder 
aufgestellt.
