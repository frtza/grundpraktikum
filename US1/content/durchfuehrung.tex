% Vorbereitung: Vorbereitungsaufgaben bearbeiten
% Versuchsaufbau: Verwendete Apparatur, Beschreibung Funktionsweise/Nutzen mit Skizze/Foto
\section{Durchführung}
\label{sec:durchführung}

Zur Vorbereitung können die Schallgeschwindigkeiten von Acryl $c_\text{Acryl} = \qty{2700}{\meter\per\second}$ \cite{doppler}
sowie von destilliertem Wasser und Luft mit $c_\text{Wasser} = \qty{1497}{\meter\per\second}$ und
$c_\text{Luft} = \qty{346}{\meter\per\second}$~\cite{lid_chem_phy} bei einer Temperatur von $\qty{25}{\celsius}$ angegeben werden.
Wegen $\lambda = c / \nu$ und $T = 1 / \nu$ lauten Wellenlänge und Periode in Acryl bei Frequenzen von $\qty{1}{\mega\hertz}$,
$\qty{2}{\mega\hertz}$ und $\qty{4}{\mega\hertz}$ jeweils:

\begin{table}[H]
	\centering
	\begin{tabular}{S[table-format=1.0] S[table-format=1.3] S[table-format=1.2]}
		\toprule
		{$\nu \mathbin{/} \unit{\mega\hertz}$} &
		{$\lambda \mathbin{/} \unit{\milli\meter}$} &
		{$T \mathbin{/} \unit{\micro\second}$} \\
		\midrule
		1 & 2.700 & 1.00 \\
		2 & 1.350 & 0.50 \\
		4 & 0.675 & 0.25 \\
		\bottomrule
	\end{tabular}
\end{table}

Zur Aufnahme und Analyse der Daten stehen ein Ultraschallechoskop im Impuls-Betrieb, Ultraschallsonden der Frequenzen $\qty{1}{\mega\hertz}$
und $\qty{2}{\mega\hertz}$ für das Impuls-Echo-Verfahren, sowie ein Rechner mit dem Programm \verb+AScan+ zur Verfügung. Damit lassen sich
A- und B-Scans erstellen, wobei das Signal des A-Scans wahlweise als Funktion der Laufzeit oder bei bekannter Schallgeschwindigkeit in
Abhängigkeit der Eindringtiefe dargestellt werden kann. Zum B-Scan werden die aufgenommenen Grafiken exportiert.

Exemplarisch für eine Anwendung in der Werkstoffprüfung wird zunächst ein Acrylblock vermessen. Dieser enthält Bohrungen unterschiedlicher
Größe und Position, unter Verwendung der $\qty{2}{\mega\hertz}$ Sonde werden diese Maße je mittels A- und B-Scan bestimmt. Durch Einsatz beider
Sonden lässt sich anhand zwei benachbarter Störstellen zudem das Auflösungsvermögen der verfügbaren Frequenzen $\qty{1}{\mega\hertz}$ und
$\qty{2}{\mega\hertz}$ vergleichen. Als Kontaktmittel wird bidestilliertes Wasser verwendet. Als Referenz werden alle Messungen auch mit
einer Schieblehre durchgeführt.

Anschließend wird als Besipiel für eine medizinische Anwendung ein Brustmodell auf Lage, Größe und Art verschiedener Tumore untersucht. Diese
werden dazu ertastet und dann per B-Scan aufgenommen. Hierbei dient ein zäheres Ultraschallgel als Kopplungsmittel.
\newpage
