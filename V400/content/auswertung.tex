% Messwerte: Alle gemessenen Größen tabellarisch darstellen
% Auswertung: Berechnung geforderter Ergebnisse mit Schritten/Fehlerformeln/Erläuterung/Grafik (Programme)
\section{Auswertung}
\label{sec:auswertung}


\subsection{Reflexion}

\begin{figure}[H]
	\includegraphics{build/fig_reflexion.pdf}
	\caption{}
	\label{fig:reflexion}
\end{figure}

\begin{align*}
	A = \input{build/fit_ref_0.tex} && B = \input{build/fit_ref_1.tex}
\end{align*}

\begin{table}[H]
	\centering
	\caption{}
	\input{build/tab_reflexion.tex}
	\label{tab:reflexion}
\end{table}

\begin{equation*}
	\beta / \alpha = \input{build/stat_ref.tex}
\end{equation*}


\subsection{Brechung}

\begin{figure}[H]
	\includegraphics{build/fig_brechung.pdf}
	\caption{}
	\label{fig:brechung}
\end{figure}

\begin{align*}
	A = \input{build/fit_bre_0.tex} && B = \input{build/fit_bre_1.tex}
\end{align*}

\begin{equation*}
	n = \input{build/n_bre.tex}
\end{equation*}

$c = \input{build/c.tex}$

\begin{equation*}
	c = \input{build/c_n.tex}
\end{equation*}

\begin{table}[H]
	\centering
	\caption{}
	\input{build/tab_brechung.tex}
	\label{tab:brechung}
\end{table}

\begin{equation*}
	n = \input{build/stat_bre.tex}
\end{equation*}

\begin{equation*}
	c = \input{build/c_stat.tex}
\end{equation*}


\subsection{Strahlversatz}

\begin{equation*}
	n = \input{build/n_plan.tex}
\end{equation*}

\begin{table}[H]
	\centering
	\caption{}
	\input{build/tab_plan.tex}
	\label{tab:plan}
\end{table}


\subsection{Dispersion}

\begin{equation*}
	n = \input{build/n_pris.tex}
\end{equation*}

\begin{table}[H]
	\centering
	\caption{}
	\makebox[\textwidth][c]{\input{build/tab_pris.tex}}
	\label{tab:pris}
\end{table}

\begin{equation*}
	\delta_G = \input{build/d_pris_g.tex}
\end{equation*}

\begin{equation*}
	\delta_R = \input{build/d_pris_r.tex}
\end{equation*}


\subsection{Beugung}

\begin{figure}[H]
	\includegraphics{build/fig_beugung.pdf}
	\caption{}
	\label{fig:beugung}
\end{figure}
