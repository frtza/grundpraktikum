% Messwerte: Alle gemessenen Größen tabellarisch darstellen
% Auswertung: Berechnung geforderter Ergebnisse mit Schritten/Fehlerformeln/Erläuterung/Grafik (Programme)
\newpage
\section{Auswertung}
\label{sec:auswertung}

Im folgenden Abschnitt sollen die zuvor beschriebenen Zusammenhänge experimentell realisiert und überprüft werden.

\subsection{Statistik}

Dazu ist an dieser Stelle zunächst das Vorgehen zur Behandlung von fehlerbehafteten Daten ausgeführt. Alle Messungen werden mit einer
Unsicherheit von einer Einheit der jeweils zum Ablesen verwendeten Skala angegeben. Mithilfe von \verb+uncertainties+ \cite{uncertainties}
lassen sich diese Abweichungen bei der Berechnung abgeleiteter Größen mittels linearer Fehlerfortpflanzung automatisiert mitführen. Weiter gibt
\begin{equation*}
	\overline{x}_{\hspace{-0.2ex}\Delta} =
	\frac{\sum_k x_k / \hspace{-0.2ex}\Delta\hspace{0.2ex} x_k}{\raisebox{-0.2ex}{$\sum_k 1/ \hspace{-0.2ex}\Delta\hspace{0.2ex} x_k$}}
\end{equation*}
das mit dem reziproken Fehler $\hspace{-0.2ex}\Delta\hspace{0.2ex} x_k$ gewichtete arithmetische Mittel einer Messreihe $x_k$ an. Als zugehöriges
Streumaß wird die Standardabweichung
\begin{equation*}
	\hspace{-0.2ex}\Delta\hspace{0.2ex}\overline{x}_{\hspace{-0.2ex}\Delta} =
	\frac{\raisebox{-0.4ex}{$1$}}{\sqrt{\sum_k 1/ \hspace{-0.2ex}\Delta\hspace{0.2ex} x_k}}
\end{equation*}
verwendet, sodass der Mittelwert
$x = \overline{x}_{\hspace{-0.2ex}\Delta\hspace{-0.2ex}} {}\pm{} \hspace{-0.2ex}\Delta\hspace{0.2ex}\overline{x}_{\hspace{-0.2ex}\Delta}$
definiert werden kann. Zur grafischen Auswertung durch \verb+matplotlib+ \cite{matplotlib} wird entlang den Datenpunkten eine lineare Regression
unter Anwendung von \verb+numpy+ \cite{numpy} durchgeführt. Abweichungen der Optimierungsvariablen werden dabei der Kovarianzmatrix entnommen.


\subsection{Reflexion}

Unter Variation des Einfallswinkels $\alpha$ wird in Tabelle \ref{tab:reflexion} der Reflexionswinkel $\beta$ aufgetragen, für den nach
Reflexionsgesetz $\beta = \alpha$ gelten muss. Das ebenso angegebene Verhältnis $\beta / \alpha$ bemisst die relative Abweichung vom
erwarteten Zusammenhang.

Für die in Abbildung \ref{fig:reflexion} dargestellte Ausgleichsrechnung wird ein Modell der Form
\begin{equation*}
	\beta = A \alpha + B
\end{equation*}
herangezogen, wobei die Verschiebung $B$ als zusätzlicher Freiheitsgrad dient.

\begin{figure}[H]
	\includegraphics{build/fig_reflexion.pdf}
	\caption{Reflexionswinkel $\beta$ gegen Einfallswinkel $\alpha$ aufgetragen.}
	\label{fig:reflexion}
\end{figure}

Dies liefert die Parameter
\begin{align*}
	A = \input{build/fit_ref_0.tex} && B = \input{build/fit_ref_1.tex}
\end{align*}
als Ergebnis. Mit der Geradensteigung $A = \input{build/fit_ref_0.tex}$ ist analog zum Winkelverhältnis ein Maß für die Güte der Theorie gegeben.
Anhand der Messwerte in Tabelle \ref{tab:reflexion} lässt sich dazu noch der Mittelwert $\beta / \alpha = \input{build/stat_ref.tex}$ schreiben.

\begin{table}[H]
	\centering
	\caption{Datenpunkte der Winkel $\alpha$ und $\beta$ sowie deren Verhältnis.}
	\input{build/tab_reflexion.tex}
	\label{tab:reflexion}
\end{table}


\subsection{Brechung}

Laut Brechungsgesetz sollten Einfallswinkel $\alpha$ und Brechungswinkel $\beta$ eines Lichtstrahls an der Grenzfläche zwischen Luft mit
Brechungsindex $n_0 \approx 1$ und einer optisch dichteren planparallelen Platte aus Plexiglas mit $n$ als Brechungsindex der Formel
\begin{equation*}
	\frac{\sin(\alpha)}{\sin(\beta)} = n
\end{equation*}
gehorchen. Es wird entlang der Messung in Tabelle \ref{tab:brechung} eine Regression nach
\begin{equation*}
	\sin(\beta) = A \sin(\alpha) + B
\end{equation*}
durchgeführt und das Ergebnis in Abbildung \ref{fig:brechung} angezeigt.

\begin{figure}[H]
	\includegraphics{build/fig_brechung.pdf}
	\caption{Brechungswinkel $\sin(\beta)$ gegen Einfallswinkel $\sin(\alpha)$ aufgetragen.}
	\label{fig:brechung}
\end{figure}

Anhand den so erhaltenen Fit-Parametern
\begin{align*}
	A = \input{build/fit_bre_0.tex} && B = \input{build/fit_bre_1.tex}
\end{align*}
lässt sich aus dem Kehrwert von $A$ der Brechungsindex $n = \input{build/n_bre.tex}$ bestimmen. 

\begin{table}[H]
	\centering
	\caption{Messung der Winkel $\alpha$ und $\beta$ mit Brechungsindex $n$ als abgeleiteter Wert.}
	\input{build/tab_brechung.tex}
	\label{tab:brechung}
\end{table}

Über die Lichtgeschwindigkeit im Vakuum $c = \input{build/c.tex}$ \cite{scipy} und die Beziehung $c = nv$ folgt mit $n = \input{build/n_bre.tex}$
die Geschwindigkeit $v = \input{build/c_n.tex}$ für die Ausbreitung in Plexiglas. Analog liefert der Mittelwert $n = \input{build/stat_bre.tex}$
mit $v = \input{build/c_stat.tex}$ ein leicht abweichendes Ergebnis.



\subsection{Strahlversatz}

Zur Untersuchung der Verschiebung des Lichtstrahls wird weiterhin der bereits in Tabelle~\ref{tab:brechung} einsehbare Datensatz für die
planparallele Platte mit Dicke $d = \input{build/d_plan.tex}$ \cite{reflex} genutzt. Deren Brechungsindex ist hier als Mittelwert
$n = \input{build/n_plan.tex}$ der vorherigen Ergebnisse angenommen. Aus geometrischer Überlegung lässt sich der Versatz
\begin{equation*}
	s = d \: \frac{\sin(\alpha - \beta)}{\cos(\beta)}
\end{equation*}
in Abhängigkeit von Einfallswinkel $\alpha$ und Brechungswinkel $\beta$ formulieren.

\begin{table}[H]
	\centering
	\caption{Strahlversatz $s$ mit verwendeten Winkelmessungen.}
	\input{build/tab_plan.tex}
	\label{tab:plan}
\end{table}

Tabelle \ref{tab:plan} enthält die Brechungswinkel $\beta$ aus direkter Messung sowie $\hat{\beta}$ aus der Vorhersage des Brechungsgesetzes zum
eingestellten Einfallswinkel $\alpha$. Entsprechend wird $s$ aus $\beta$ und $\hat{s}$ aus $\hat{\beta}$ berechnet.


\subsection{Dispersion}

Nun wird ein Prisma aus Kronglas mit einem Brechungsindex $n = \input{build/n_pris.tex}$ und Innenwinkeln $\gamma = \qty{60}{\degree}$ \cite{reflex}
verwendet. Durchläuft ein Lichtstrahl das Prisma, erfährt dieser abhängig von seiner jeweiligen Wellenlänge $\lambda$ eine Ablenkung, welche sich nach
\begin{equation*}
	\delta = (\alpha_1 + \alpha_2) - (\beta_1 + \beta_2)
\end{equation*}
berechnet. Gemessen werden die Winkel $\alpha$ in Luft, über $\sin(\alpha) = n \sin(\beta)$ ergeben sich die Verläufe innerhalb des Mediums.
Mithilfe der Beziehung $\gamma = \beta_1 + \beta_2$ kann die Gültigkeit der Rechnung überprüft werden. Tabelle \ref{tab:pris_g} gibt die
Ergebnisse der Messung für grünes Licht mit $\lambda = \qty{532}{\nano\meter}$ an, in Tabelle \ref{tab:pris_r} sind Daten und abgeleitete Größen
zum roten Laser bei $\lambda = \qty{635}{\nano\meter}$ eingetragen.

\begin{table}[H]
	\centering
	\caption{Messwerte zur Bestimmung der Ablenkung $\delta$ für grünes Licht.}
%	\makebox[\textwidth][c]{\input{build/tab_pris.tex}}
	\input{build/tab_pris_g.tex}
	\label{tab:pris_g}
\end{table}

Im Mittel gilt eine Ablenkung von $\delta_G = \input{build/d_pris_g.tex}$ für den grünen Laser.

\begin{table}[H]
	\centering
	\caption{Messwerte zur Bestimmung der Ablenkung $\delta$ für rotes Licht.}
%	\makebox[\textwidth][c]{\input{build/tab_pris.tex}}
	\input{build/tab_pris_r.tex}
	\label{tab:pris_r}
\end{table}

Für rotes Licht gibt $\delta_R = \input{build/d_pris_r.tex}$ den Mittelwert an.

\newpage
\subsection{Beugung}

Zuletzt soll die Wellennatur des Lichts untersucht werden. Zur Berechnung der Wellenlänge wird in den Tabellen \ref{tab:beugung_g} und
\ref{tab:beugung_r} die umgestellte Gleichung
\begin{equation*}
	\lambda = d \: \frac{\sin(\varphi)}{k}
\end{equation*}
verwendet. Da Maxima nullter Beugungsordnung trivial sind und keine Rückschlüsse auf die Wellenlänge zulassen, werden sie an dieser Stelle
vernachlässigt. Zur linearen Ausgleichsrechnung wird der Ausdruck
\begin{equation*}
	d \sin(\varphi) = \lambda k + \Lambda
\end{equation*}
mit konstantem Verschiebungsfaktor $\Lambda$ gewählt.

\begin{figure}[H]
	\includegraphics{build/fig_beugung.pdf}
	\captionsetup{width=0.997\linewidth}
	\caption{Wegunterschied $d \sin(\varphi)$ gegen Beugungsordnung $k$ aufgetragen. Passend farblich kodiert und zur besseren
			 Lesbarkeit entlang der ganzzahligen Ordnung verschoben.}
	\label{fig:beugung}
\end{figure}

Aus den Daten für grünes Licht in Tabelle \ref{tab:beugung_g} ergeben sich
\begin{align*}
	\lambda_G = \input{build/fit_beu_g_0.tex} && \Lambda_G = \input{build/fit_beu_g_1.tex}
\end{align*}
als optimale Parameter. Die Koeffizienten
\begin{align*}
	\lambda_R = \input{build/fit_beu_r_0.tex} && \Lambda_R = \input{build/fit_beu_r_1.tex}
\end{align*}
folgen aus den Messwerten in Tabelle \ref{tab:beugung_r}.
\newpage

\begin{table}[H]
	\centering
	\caption{Messwerte zur Bestimmung der Wellenlänge $\lambda$ für grünes Licht.}
	\input{build/tab_beugung_g.tex}
	\label{tab:beugung_g}
\end{table}

Für den grünen Laser beschreibt $\lambda_G = \input{build/stat_beu_g.tex}$ die mittlere Wellenlänge.

\begin{table}[H]
	\centering
	\caption{Messwerte zur Bestimmung der Wellenlänge $\lambda$ für rotes Licht.}
	\input{build/tab_beugung_r.tex}
	\label{tab:beugung_r}
\end{table}

Mit $\lambda_R = \input{build/stat_beu_r.tex}$ ist der Mittelwert des roten Lasers angegeben.

\newpage
