% Diskussion: Resultate mit Fehler/Genauigkeit zusammenstellen, Literaturwerte/Messmethoden/Ursachen vergleichen
% Literatur: Verwendete Literatur/Grafiken/Werte/Programme
% Anhang: Kopie der analog eingetragenen Messdaten
\section{Diskussion}
\label{sec:diskussion}

Insgesamt sind alle Messreihen von ähnlichen potentiellen Fehlerquellen betroffen. Dabei treten statistische Abweichungen unvermeidbar als Resultat
von Ableseungenauigkeiten auf, haben mit wachsender Stichprobengröße jedoch einen verschwindend geringen Einfluss. Systematische Fehler sind vor
allem auf eine ungenaue Ausrichtung des Aufbaus mit der Messvorlage oder Verunreinigungen der Oberflächen von Spiegel, Platte, Prisma und Gitter
zurückzuführen.

Das Reflexionsgesetz darf mit $A = \input{build/fit_ref_0.tex}$ und $\beta / \alpha = \input{build/stat_ref.tex}$ als bestätigt angenommen
werden, da beide Maße kaum von $1$ als Erwartungswert abweichen. Bei Betrachtung der zugehörigen Tabelle \ref{tab:reflexion} fällt allerdings
eine leichte Tendenz $\beta \gtrsim \alpha$ auf, die wahrscheinlich mit einem oder mehreren der beschriebenen systematischen Fehlereinflüsse
erklärt werden kann.

Mit einem ermittelten Brechungsindex von $n = \input{build/n_plan.tex}$ weist das Ergebnis des angewendeten Brechungsgesetzes eine gute
Verträglichkeit mit dem Literaturwert für Plexiglas $n = \num{1.490}$ auf. Als Verifikation der Gültigkeit des theoretischen Zusammenhangs
nach Snellius können auch die nahezu identischen Strahlversätze aus direkter Messung $s$ und indirekter Berechnung $\hat{s}$ in
Tabelle \ref{tab:plan} betrachtet werden. 

Am Prisma stimmen die Ablenkungen $\delta_G = \input{build/d_pris_g.tex}$ für grünes und $\delta_R = \input{build/d_pris_r.tex}$ für rotes Licht
fast überein. Unter Berücksichtigung der Tabellen \ref{tab:pris_g} und \ref{tab:pris_r} lässt sich andeutungsweise $\delta_R \lesssim \delta_G$
erkennen. Dies deutet zwar auf Dispersion hin, ist aber nicht ausreichend, um deren Existenz eindeutig zu belegen. Damit das gelingen kann,
bräuchte es exaktere Skalen und Apparaturen.

Die aus den Beugungsphänomenen bestimmten Wellenlängen $\lambda_G = \input{build/fit_beu_g_0.tex}$ und $\lambda_G = \input{build/stat_beu_g.tex}$
für grünes sowie $\lambda_R = \input{build/stat_beu_r.tex}$ und $\lambda_R = \input{build/fit_beu_r_0.tex}$ für rotes Licht liegen etwas unterhalb
der vorgegebenen Werte $\lambda_G = \qty{532}{\nano\meter}$ und $\lambda_R = \qty{635}{\nano\meter}$. In Anbetracht der möglicherweise
signifikanten Fehlerursachen liegt aber auch hier eine gute Übereinstimmung vor.
\newpage
