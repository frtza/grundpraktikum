% Theorie: Physikalische Grundlagen von Versuch/Messverfahren, Gleichungen ohne Herleitung knapp erklären
\section{Theorie}
\label{sec:theorie}

Im Folgenden werden elementare Begriffe der Strahlen- und der Wellenoptik eingeführt und erläutert.

Licht ist eine Form der elektromagnetischen Strahlung. Das optische Spektrum erstreckt sich von ultraviolettem Licht, welches 
in einem Wellenlängenbereich von $\SI{100}{\nano\meter}$ bis $\SI{380}{\nano\meter}$ vorkommt und reicht bis in das Infrarotspektrum, welches 
den Wellenlänenbereich von $\SI{780}{\nano\meter}$ bis $\SI{1}{\milli\meter}$ hat. Das für den Menschen sichtbare Licht ist dabei in dem 
Wellenlängenbereich von $\SI{380}{\nano\meter}$ bis $\SI{780}{\nano\meter}$.

\subsection{Strahlenotik}
\label{sec:Strahlenotik}
 
Für die Beschreibung von Refelxion und Brechnung an Grenzflächen können die Regeln der Strahlenotik angewandt werden.
Dabei wird die Wellenausbreitung über die Normalen der Wellenflächen beschrieben. Diese wird als Lichtstrahl bezeichnet.
Lichtstrahlen breiten sich in einem homogenen Medium gradlinig aus. Wenn sich zwei oder mehr Lichtstrahlen kreuzen haben diese
keine Einflüsse aufeinander.
Für unterschiedliche Materialien ist auch die Ausbreitungsgeschwindigkeit der Welle anders. Daher wird beim Übergang von einem
Medium in ein anderes die Welle gebrochen. Für die Aubreitungsgeschwindigkeiten $v_1$ und $v_2$ ergibt sich die Beziehung
\begin{equation}
    \frac{\sin \alpha}{\sin \beta} =\frac{v_1}{v_2} = \frac{n_1}{n_2}.
    \label{eqn:bez}
\end{equation}
Dabei beschreibt der Winkel $\alpha$ den Einfallswinkel und $\beta$ den Ausfallswinkel, beide Winkel werden  zur Normalen der Grenzfläche gemessen. 
$n$ ist der Brechungsindex, welcher eine optische Materialeigenschaft ist.
Wenn die Ausbreitungsgeschwindigkeit in Medium 1 größer ist als die in Mediu  2, wird das Medium 1 als optisch dünner 
bezeichnet. Andersherum ist das Medium 1 optisch dicker.

\subsubsection{Reflexion}
\label{sec:Reflexion}

