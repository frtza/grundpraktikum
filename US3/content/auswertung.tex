% Messwerte: Alle gemessenen Größen tabellarisch darstellen
% Auswertung: Berechnung geforderter Ergebnisse mit Schritten/Fehlerformeln/Erläuterung/Grafik (Programme)
\section{Auswertung}
\label{sec:auswertung}

Im Folgenden wird die Strömungsgeschwindigkeit der aufgenommen Messwerte bestimmt.
Außerdem wird das Strömungsprofil für zwei Pumpleistungen betrachtet.

% \subsection{Vorbereitungsaufgabe}
% \label{sec:Vorbereitungsaufgabe}
% 
% Der Dopplerwinkel lässt sich über die Gleichung 
% \begin{equation*}
%     \alpha = \qty{90}{°} - \arcsin\left(\sin\theta \frac{c_L}{c_p}\right)
% \end{equation*}
% bestimmen. Die berechneten Werte sind in der \autoref{tab:doppler} aufgetragen.
% 
% \newpage
% \begin{table}[H]
%     \centering
%     \caption{Vorbereitungsaufgabe: Dopplerwinkel zu den jeweiligen Prismenwinkeln.}
%     \label{tab:doppler}
% \begin{tabular}{c c}
%     \toprule
%     $\theta$ & $\alpha$ \\
%     \midrule
%     $\qty{15}{°}$ & $\qty{80.06}{°}$ \\
%     $\qty{30}{°}$ & $\qty{70.53}{°}$ \\
%     $\qty{45}{°}$ & $\qty{54.74}{°}$ \\
%     \bottomrule
% \end{tabular}
% \end{table}

\subsection{Strömungsgeschwindigkeit}
\label{sec:Strömungsgeschwindigkeit}

In der \autoref{tab:messwerte1} sind die die aufgenommenen Messwerte der Frequenzdifferenzen $ \increment \nu = \nu_{\text{mean}} - \nu_{\text{max}}$ 
der jeweiligen Prismenwinkel $\theta$ dargestellt. Die Frequenzdifferenz wird für fünf verschiedene Leistungen für die drei Winkel $\theta$ gemessen. Die 
verwendete Pumpe hat dabei eine maximale Leistung von $\SI{7.5}{\liter\per\minute}$.

\begin{table}[H]
    \centering
    \caption{Die Frequenzverschiebung der drei Prismenwinkel.}
    \label{tab:messwerte1}
\begin{tabular}{c c c c}
    \toprule
     $v / \si{\liter\per\minute}$ & $ \increment \nu_{\qty{15}{°}}$ & $\increment \nu_{\qty{30}{°}}$ & $\increment \nu_{\qty{60}{°}}$ \\
    \midrule
       2 &  56 &  94 & 153 \\
     2.5 &  65 & 132 & 208 \\
     3.0 &  90 & 167 & 292 \\
     3.5 & 114 & 253 & 334 \\
       4 & 165 & 311 & 468 \\
    \bottomrule
\end{tabular}
\end{table}

Anhand der \autoref{eqn:frequenz} wird die Strömungsgeschwindigkeit berechnet. Für die verwendete Sonde ist 
$\nu_0$ angegeben als $\SI{2}{\mega\hertz}$.

\begin{table}[H]
    \centering
    \caption{Die Strömungsgeschwindigkeit der drei Prismenwinkel.}
    \label{tab:geschw}
\begin{tabular}{c c c}
    \toprule
     $v_{\qty{15}{°}} / \si{\meter \per \second}$ &  $v_{\qty{30}{°}} / \si{\meter \per \second}$ &  $v_{\qty{60}{°}} / \si{\meter \per \second}$ \\
    \midrule
       0.243 &    0.211 &    0.199 \\
       0.282 &    0.297 &    0.270 \\
       0.391 &    0.376 &    0.379 \\
       0.495 &    0.569 &    0.434 \\
       0.716 &    0.699 &    0.608 \\
    \bottomrule
    \end{tabular}
\end{table}

In den Abbildungen \ref{fig:plot1}, \ref{fig:plot2} und \ref{fig:plot3} wird das Verhältnis von $\increment \nu$ in Abhängigkeit der Strömungsgeschwindigkeit
$\nu$ aufgetragen. Zudem wurde eine Ausgleichgerade der Form 
\begin{equation*}
  f(x) =a x + b 
\end{equation*}
durch die Messwerte gefittet.

\begin{figure}[H]
	\includegraphics{build/plot1.pdf}
	\captionsetup{width=0.765\linewidth}
	\caption{Die Frequenzverschiebung in Abhängigkeit der Strömungsgeschwindigkeit für $\theta = \qty{15}{°}$.}
	\label{fig:plot1}
\end{figure}
Die Parameter der Fitfunktion für den Winkel $\theta = \qty{15}{°}$ entsprechen
\begin{align*}
  a &= (2222.222 \pm 0.0000028) \mathrm{m}^{-1}\\
  b &= -(2.5\pm 0.8) \cdot 10^{-11}\,\mathrm{s}^{-1}.
\end{align*}
\begin{figure}[H]
	\includegraphics{build/plot2.pdf}
	\captionsetup{width=0.765\linewidth}
	\caption{Die Frequenzverschiebung in Abhängigkeit der Strömungsgeschwindigkeit für $\theta = \qty{30}{°}$.}
	\label{fig:plot2}
\end{figure}
Die Parameter der Fitfunktion für den Winkel $\theta = \qty{30}{°}$ entsprechen
\begin{align*}
  a &= (2222.222 \pm 0.0000027) \mathrm{m}^{-1}\\
  b &= -(2.4\pm 0.6) \cdot 10^{-11}\,\mathrm{s}^{-1}.
\end{align*}
\begin{figure}[H]
	\includegraphics{build/plot3.pdf}
	\captionsetup{width=0.765\linewidth}
	\caption{Die Frequenzverschiebung in Abhängigkeit der Strömungsgeschwindigkeit für $\theta = \qty{60}{°}$.}
	\label{fig:plot3}
\end{figure}
Die Parameter der Fitfunktion für den Winkel $\theta = \qty{60}{°}$ entsprechen
\begin{align*}
  a &= (2222.222 \pm 0.0000017) \mathrm{m}^{-1}\\
  b &= -(1.58\pm 0.18) \cdot 10^{-11}\,\mathrm{s}^{-1}.
\end{align*}
\subsection{Strömungsprofil}
\label{sec:Strömungsprofil}

Für den Prismawinkel $\theta = \qty{15}{°}$ wird für die Pumpleistungen $ 3.4 \si{\liter\per\minute}$ und $ 5.2 \si{\liter\per\minute}$, was jeweils
$45\%$  und $70\%$ der maximalen Pumpleistung entspricht, die Strömungsgeschwindigkeit sowie die Streuintensität in Abhängigkeit der Tiefe gemessen.
Dabei entspricht eine Messtiefe von $\qty{4}{\micro\second}$ in der Dopplerflüssigkeit $\qty{6}{\milli\meter}$.
Die aufgenommenen Werte sind in der \autoref{tab:sp1} und der \autoref{tab:sp2} aufgetragen.

\begin{table}[H]
    \centering
    \caption{Die Strömungsgeschwindigkeit und die Streuintensität für größerwerdende Tiefen bei einer Leistung von $45 \%$.}
    \label{tab:sp1}
\begin{tabular}{c c c}
    \toprule
    $d / \si{\micro\second}$ & $ v / \si{\centi \meter \per \second}$ & $I_S / 1000 \si{ V^2 \per \second} $\\
    \midrule
      12 &     0 &  20 \\
    12.5 &     0 &  22 \\
      13 &     0 &  25 \\
    13.5 &  13.3 &  28 \\
      14 &  37.1 &  35 \\
    14.5 &  39.8 &  38 \\
      15 &  42.4 &  40 \\
    15.5 &  42.8 &  44 \\
      16 &  42.4 &  46 \\
    16.5 &  39.8 &  47 \\
      17 &  34.5 &  46 \\
    17.5 &  31.8 &  43 \\
      18 &  34.5 &  65 \\
    18.5 &  37.1 & 114 \\
      19 &  37.1 & 117 \\
    19.5 & 39.08 &  81 \\
    \bottomrule
    \end{tabular}
\end{table}

\begin{table}[H]
    \centering
    \caption{Die Strömungsgeschwindigkeit und die Streuintensität für größerwerdende Tiefen bei einer Leistung von $70 \%$.}
    \label{tab:sp2}
\begin{tabular}{c c c}
    \toprule
    $d / \si{\micro\second}$ & $ v / \si{\centi \meter \per \second}$ & $I_S / 1000 \si{ V^2 \per \second} $\\
    \midrule
      12 &   71.6 &  119 \\
    12.5 &   63.7 &  170 \\
      13 &   69.0 &  256 \\
    13.5 &  82.2 &  332 \\
      14 &  98.2 &  440 \\
    14.5 &  106.1 &  561 \\
      15 &  108.8 &  600 \\
    15.5 &  103.5 &  640 \\
      16 &  95.5 &  630 \\
    16.5 &  82.2 &  663 \\
      17 &  74.3 &  865 \\
    17.5 &  76.9 &  1399 \\
      18 &  87.5 &  2060 \\
    18.5 &  90.2 & 2478 \\
      19 &  87.5 & 2174 \\
    \bottomrule
    \end{tabular}
\end{table}

In der \autoref{fig:str1} und der \autoref{fig:str2} sind die Momentangeschwindigkeiten und die Streuintensitäten für die jeweiligen
Pumpleistungen graphisch dargestellt.


\begin{figure}
  \begin{subfigure}{0.48\textwidth}
    \centering
    \includegraphics[height=4.5cm]{build/plot4.pdf}
    \caption{Momentangeschwindigkeit für v = $45\%$.}
    \label{fig:plot4}
  \end{subfigure}
  \hfill
  \begin{subfigure}{0.48\textwidth}
    \centering
    \includegraphics[height=4.5cm]{build/plot5.pdf}
    \caption{Streuintensität für v = $45\%$.}
    \label{fig:plot5}
  \end{subfigure}
  \caption{Strömungsprofil für $45\%$ der maximalen Pumpleistung.}
  \label{fig:str1}
\end{figure}

\begin{figure}
  \begin{subfigure}{0.48\textwidth}
    \centering
    \includegraphics[height=4.5cm]{build/plot6.pdf}
    \caption{Momentangeschwindigkeit für v = $70\%$.}
    \label{fig:plot6}
  \end{subfigure}
  \hfill
  \begin{subfigure}{0.48\textwidth}
    \centering
    \includegraphics[height=4.5cm]{build/plot7.pdf}
    \caption{Streuintensität für v = $70\%$.}
    \label{fig:plot7}
  \end{subfigure}
  \caption{Strömungsprofil für $70\%$ der maximalen Pumpleistung.}
  \label{fig:str2}
\end{figure}
