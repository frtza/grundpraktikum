% Diskussion: Resultate mit Fehler/Genauigkeit zusammenstellen, Literaturwerte/Messmethoden/Ursachen vergleichen
% Literatur: Verwendete Literatur/Grafiken/Werte/Programme
% Anhang: Kopie der analog eingetragenen Messdaten
\section{Diskussion}
\label{sec:diskussion}

Bei der Bestimmung der Strömungsgeschwindigkeit fällt ein linearer Zusammenhang zwischen
der auf den Cosinus des Prismawinkels normierten Frequenzverschiebung und der Strömungsgeschwindigkeit auf. Dementsprechend kann durch die experimentell 
aufgenommen Werte festgestellt werden, dass die Strömungsgeschwindigkeit mit der Leistung zunimmt.

Theoretisch wird erwartet, dass die Strömungsgeschwindigkeit an den Rändern des Rohres geringer ist als in der Mitte.
Dies lässt sich über die an den Rändern aufkommende Reibung erklären. In der Mitte des Rohres sollte voraussichtlich
die Streuintensität minimal sein, da die Strömung dort nahezu laminar verlaufen sollte.
Experimentell konnten diese Annahmen nicht bestätigt werden.

Zu den Fehlerquellen, welche die Ergebnisse hervorgerufen haben, gehört zunächst, dass die Sonde per Hand an das Prisma gehalten
worden ist. Daher ist die US-Sonde mehrmals verrutscht und hatte eine sehr unruhige Position.
Bei dem Aufschreiben der Messwerte kam es dazu, dass nur ein ungefähres Mittel der Signale abgelesen wurde, da die 
Signalstärken stark geschwankt haben. Für die Bestimmung des Strömungsprofils wurden mehr Messwerte in einem längeren Zeitraum 
aufgenommen, dementsprechend wirkte die unruhige Position der US-Sonde stärker. Wichtig war, darauf zu achten, dass sich ausreichend
Ultraschall-Gel zwischen dem Prisma und der Sonde befindet. Das konnte nicht ideal umgesetzt werden, sodass Luftblasen zwischen die Sonde
und das Prisma gekommen sind.
