% Vorbereitung: Vorbereitungsaufgaben bearbeiten
% Versuchsaufbau: Verwendete Apparatur, Beschreibung Funktionsweise/Nutzen mit Skizze/Foto
\section{Durchführung}
\label{sec:durchführung}

Zur Vorbereitung lassen sich für Prismenwinkel $\theta$ von $\qty{15}{\degree}$, $\qty{30}{\degree}$ und $\qty{60}{\degree}$ die
entsprechenden Doppler-Winkel $\alpha (\theta)$ berechnen, indem die Schallgeschwindigkeiten $c_\text{L} = \qty{1800}{\meter\per\second}$
und $c_\text{P} = \qty{2700}{\meter\per\second}$ in Ausdruck~\eqref{eqn:prisma} eingesetzt werden:
\begin{align*}
\alpha (\qty{15}{\degree}) = \qty{80.06}{\degree} &&
\alpha (\qty{30}{\degree}) = \qty{70.53}{\degree} &&
\alpha (\qty{60}{\degree}) = \qty{54.74}{\degree}
\end{align*}
