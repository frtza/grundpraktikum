% Vorbereitung: Vorbereitungsaufgaben bearbeiten
% Versuchsaufbau: Verwendete Apparatur, Beschreibung Funktionsweise/Nutzen mit Skizze/Foto
\section{Durchführung}
\label{sec:durchführung}

Zur Vorbereitung lassen sich für Prismenwinkel $\theta$ von $\qty{15}{\degree}$, $\qty{30}{\degree}$ und $\qty{60}{\degree}$ die
entsprechenden Doppler-Winkel $\alpha (\theta)$ berechnen, indem die Schallgeschwindigkeiten $c_\text{L} = \qty{1800}{\meter\per\second}$
und $c_\text{P} = \qty{2700}{\meter\per\second}$ in Ausdruck~\eqref{eqn:prisma} eingesetzt werden:
\begin{align*}
\alpha (\qty{15}{\degree}) = \qty{80.06}{\degree} &&
\alpha (\qty{30}{\degree}) = \qty{70.53}{\degree} &&
\alpha (\qty{60}{\degree}) = \qty{54.74}{\degree}
\end{align*}

Der Messappartur setzt sich aus einem Ultraschall-Doppler-Generator im Pulsbetrieb und einer daran angeschlossenen Ultraschallsonde zusammen,
welche $\qty{2}{\mega\hertz}$ als Arbeitsfrequenz verwendet. Zur Aufnahme und Analyse der gewonnenen Daten ist ein Rechner mit dem
entsprechenden Auswertungsprogramm \verb+FlowView+ angeschlossen. Damit wird ein Strömungsrohr mit Durchmesser $\qty{10}{\milli\meter}$
untersucht, das die Dopplerphantomflüssigkeit enthält. Dabei handelt es sich um eine Mischung aus Wasser, Glycerin und Glaskugeln,
dessen Viskosität so gewählt ist, dass sich bei den verwendeten Flussgeschwindigkeiten eine laminare Strömung ausbildet. Um die
Strömungsgeschwindigkeit zu variieren, \mbox{ist zudem} eine Zentrifugalpumpe mit maximaler Leistung von
$\qty[per-mode=symbol]{7.5}{\liter\per\minute}$ im Kreislauf verbaut. Zur Schonung der Gerätschaft werden $\qty{70}{\percent}$
dieser Spitzenleistung nicht überschritten.

Zunächst soll die Strömungsgeschwindigkeit als Funktion des Dopplerwinkels bestimmt werden. Dazu ist am Ultraschall-Generator
das \verb+SAMPLE-VOLUME+ auf \verb+LARGE+ gestellt. An der Pumpe wird die Leistung beginnend bei $\qty[per-mode=symbol]{2.0}{\liter\per\minute}$
in Stufen von $\qty[per-mode=symbol]{0.5}{\liter\per\minute}$ erhöht. Mithilfe von Doppler-Prisma und Kontakt-Gel werden dazu jeweils
$\qty{15}{\degree}$, $\qty{45}{\degree}$ und $\qty{60}{\degree}$ als Winkel bei gleichbleibendem Abstand zum Rohr betrachtet. 

Zur Untersuchung des Strömungsprofils am Schlauch wird bei \verb+SAMPLE-VOLUME+ auf \verb+SMALL+ und festem Ansetzwinkel gleich $\qty{15}{\degree}$
über den Regler \verb+DEPTH+ die Messtiefe in aufsteigenden Schritten von $\qty{0.5}{\micro\second}$ eingestellt. In Acryl entsprechen
$\qty{2}{\micro\second} = \qty{5}{\milli\meter}$, in der Flüssigkeit sind $\qty{2}{\micro\second} = \qty{3}{\milli\meter}$. Bei $\qty{45}{\percent}$
und $\qty{70}{\percent}$ der maximalen Pumpleistung wird je die gesamte Rohrtiefe abgetastet, Momentangeschwindigkeit und Streuintensität 
werden aufgezeichnet.

