% Theorie: Physikalische Grundlagen von Versuch/Messverfahren, Gleichungen ohne Herleitung knapp erklären
\section{Theorie}
\label{sec:theorie}

Ein Atom wird als stabil bezeichnet, wenn ein stabiles festgelegtes Verhältnis zwischen Neutronen und Protonen besteht.
Außerhalb dieser engen Grenze wandelt sich der Kern in einen stabilen oder instabilen Kern um.
Um die Zerfallswahrscheinlichkeit zu beschreiben wird die Halbwertszeit $T$ eines Nuklids bestimmt. Diese gibt bei einer 
großen Anzahl instabiler Kerne an, wann die Hälfte dieser zerfallen ist. Wenn die gesamte Nuklidkarte betrachtet wird, fällt auf, dass 
die verschiedenen Halbewertszeiten $T$ über 23 Zehnerpotenzen variieren können.
Im Folgenden Experiment werden Halbwertszeiten bestimmt. Um Nuklide mit Halbwertszeiten im Sekunden bis Stunden Bereich hezustellen,
werden stabile Kernen mit Neutronen beschossen.

\subsection{Kernreaktionen mit Neutronen}
\label{sec:Kernreaktionen mit Neutronen}

Der Begriff Kernreaktion beschreibt allgemein die Wechselwirkungen von Teilchen mit Atom Kernen.
Um die Halbwertszeiten bestimmen zu können, müssen zunächst Kernreaktionen bei denen ein Neutron in ein Teilchen
eindringt, untersucht werden. Wird ein Atomkern mit einem Neutron beschossen, so wird der Kern in ein angeregten Zustand überführt.
Diese Kernen werden als Zwischenkern oder Compoundkern bezeichnet.
Die Energie des Compoundkerns ist um die kinetische Energie und die Bindungsenergie des Neutrons höher als
vorher. Durch die zusätzliche Enrgie entsteht die Anregung der Nukleonen. Der Kern ist in den meisten Fällen ,wegen der
Verteilung der zusätzlichen Energie auf viele Nukleonen,nicht in der Lage, das Neutron oder ein Nukleon abzustoßen.
In diesem Falle wird ein $\gamma$-Quant emitiert, sodass der Zwischenkern wieder in den Grundzustand übergeht.
Für diese Reaktion gilt



