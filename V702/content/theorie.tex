% Theorie: Physikalische Grundlagen von Versuch/Messverfahren, Gleichungen ohne Herleitung knapp erklären
\section{Theorie}
\label{sec:theorie}

Ein Atom wird als stabil bezeichnet, wenn ein stabiles festgelegtes Verhältnis zwischen Neutronen und Protonen besteht.
Außerhalb dieser engen Grenze wandelt sich der Kern in einen stabilen oder instabilen Kern um.
Um die Zerfallswahrscheinlichkeit zu beschreiben wird die Halbwertszeit $T$ eines Nuklids bestimmt. Diese gibt bei einer 
großen Anzahl instabiler Kerne an, wann die Hälfte dieser zerfallen ist. Wenn die gesamte Nuklidkarte betrachtet wird, fällt auf, dass 
die verschiedenen Halbewertszeiten $T$ über 23 Zehnerpotenzen variieren können.
Im Folgenden Experiment werden Halbwertszeiten bestimmt. Um Nuklide mit Halbwertszeiten im Sekunden bis Stunden Bereich hezustellen,
werden stabile Kernen mit Neutronen beschossen.

\subsection{Kernreaktionen mit Neutronen}
\label{sec:Kernreaktionen mit Neutronen}

Der Begriff Kernreaktion beschreibt allgemein die Wechselwirkungen von Teilchen mit Atom Kernen.
Um die Halbwertszeiten bestimmen zu können, müssen zunächst Kernreaktionen bei denen ein Neutron in ein Teilchen
eindringt, untersucht werden. Wird ein Atomkern mit einem Neutron beschossen, so wird der Kern in ein angeregten Zustand überführt.
Diese Kernen werden als Zwischenkern oder Compoundkern bezeichnet.
Die Energie des Compoundkerns ist um die kinetische Energie und die Bindungsenergie des Neutrons höher als
vorher. Durch die zusätzliche Enrgie entsteht die Anregung der Nukleonen. Der Kern ist in den meisten Fällen ,wegen der
Verteilung der zusätzlichen Energie auf viele Nukleonen,nicht in der Lage, das Neutron oder ein Nukleon abzustoßen.
In diesem Falle wird ein $\gamma$-Quant emitiert, sodass der Zwischenkern wieder in den Grundzustand übergeht.
Für diese Reaktion gilt
\begin{equation}
    { }_z^m A+{ }_0^1 n \rightarrow{ }_z^{m+1} A^* \rightarrow{ }_z^{m+1} A+\gamma .
    \label{eqn:vorher}
\end{equation}
Der nun vorhandene Kern ist nicht stabil aufgrund der erhöhten Neutronenanzahl. Aufgrund des abgegebenen
Photons ist dieser langlebiger als der Compoundkern. Durch die Emission eines Elektrons geht dieser schließlich zu einem 
stabilen Kern über
\begin{equation}
    { }_z^{m+1} A \rightarrow \underset{z+1}{m+1} C+\beta^{-}+E_{k i n}+\bar{v}_e .
    \label{eqn:stabil}
\end{equation}
Die Masse von ${ }_z^{m+1} A$ ist größer als die Masse der Summe der Teilchen auf der rechten Seite der obigen Gleichung.
Gemäß der Einsteinschen Beziehung 
\begin{equation*}
    \increment E = \increment m c^2
\end{equation*}
wird die überschüssige Masse in kinetische Energie von Elektronen und Antineutrino umgewandelt.
Der Wirkungsquerschnitt $\sigma$ beschreibt die Wahrscheinlichkeit, dass ein Neutron durch einen stabilen Kern
eingefangen wird. Die durch den Wirkungsquerschnitt $\sigma$ beschriebene Fläche, die der Kern haben sollte,
sodass jedes Neutron, welches diese Fläche trifft eingefangen wird. Für den Wirkungsquerschnitt gilt
\begin{equation}
    \sigma = \frac{u}{nKd}.
    \label{eqn:wirkungs}
\end{equation}
Dabei ist $d$ die Dicke der gegebenen Probe, n die Aanzahl der auftreffenden Neutronen, $K$ die Anzahl der Kerne pro $\si{cm}^3$
und u ist die Anzahl der eingefangen Neutronen.
Es liegt eine strake Abhängigkeit des Wirkungsquerschnittes von der Geschwindigkeit der Neutronen und damit von der
kinetischen Energie der Neutronen vor. Dementsprechend wird zwischen schnellen und langsamen Neutronen differenziert.
Anhand der De-Broglie-Wellenlänge $\lambda = \frac{h}{m_n v}$ kann ein Rückschluss gezogen werden.
Wenn die Geschwindigkeit $v$ der Neutronen sher groß ist, sodass $\lambda$ klein gegen den Radius $R$ des gegebenen Kerns ist, dann
können für die Wechselwirkung der Neutronen mit dem Kern einfache geometrische Überlegeungen verwendet werden.
Für die langesamen Neutronen bei denen gilt $R \leq \lambda$, können die gemometrischen Überlegeungen nicht angewendet werden.
Es lässt sich der Wirkungsquerschnitt als Funktion der Neutroenenenergie $E$ darstellen. Die Formel von Breit und Wigner ist gegeben durch
\begin{equation}
    \sigma(E)=\sigma_0 \sqrt{\frac{E_{r_i}}{E}} \frac{\tilde{c}}{\left(E-E_{r_i}\right)^2+\tilde{c}} ,
    \label{eqn:funktion}
\end{equation}
wobei $\sigma_0$ und $\tilde{c}$ chrakteristische Konstanten der betreffenden Kernreaktion sind.