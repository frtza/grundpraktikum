% Messwerte: Alle gemessenen Größen tabellarisch darstellen
% Auswertung: Berechnung geforderter Ergebnisse mit Schritten/Fehlerformeln/Erläuterung/Grafik (Programme)
\section{Auswertung}
\label{sec:auswertung}

\subsection{Nulleffekt}

\begin{table}[H]
	\centering
	\caption{}
	\makebox[\linewidth][c]{\input{build/tab_1.tex}}
	\label{tab:1}
\end{table}

\begin{figure}[H]
	\centering
	\includegraphics[width=\linewidth]{build/plot_1.pdf}
	\caption{}
	\label{fig:1}
\end{figure}

\subsection{Vanadium}

\begin{table}[H]
	\centering
	\caption{}
	\makebox[\linewidth][c]{\input{build/tab_2.tex}}
	\label{tab:2}
\end{table}

\begin{figure}[H]
	\centering
	\includegraphics[width=\linewidth]{build/plot_2.pdf}
	\caption{}
	\label{fig:2}
\end{figure}

\subsection{Rhodium}

\begin{table}[H]
	\centering
	\caption{}
	\makebox[\linewidth][c]{\input{build/tab_3_a.tex}}
	\label{tab:3_a}
\end{table}

\begin{figure}[H]
	\centering
	\includegraphics[width=\linewidth]{build/plot_3_a.pdf}
	\caption{}
	\label{fig:3_a}
\end{figure}

\begin{figure}[H]
	\centering
	\includegraphics[width=\linewidth]{build/plot_3i_a.pdf}
	\caption{}
	\label{fig:3i_a}
\end{figure}

\begin{figure}[H]
	\centering
	\includegraphics[width=\linewidth]{build/plot_3_a_.pdf}
	\caption{}
	\label{fig:3_a_}
\end{figure}

\begin{table}[H]
	\centering
	\caption{}
	\makebox[\linewidth][c]{\input{build/tab_3_b.tex}}
	\label{tab:3_b}
\end{table}

\begin{figure}[H]
	\centering
	\includegraphics[width=\linewidth]{build/plot_3_b.pdf}
	\caption{}
	\label{fig:3_b}
\end{figure}

\begin{figure}[H]
	\centering
	\includegraphics[width=\linewidth]{build/plot_3i_b.pdf}
	\caption{}
	\label{fig:3i_b}
\end{figure}

\begin{figure}[H]
	\centering
	\includegraphics[width=\linewidth]{build/plot_3_b_.pdf}
	\caption{}
	\label{fig:3_b_}
\end{figure}
