% Messwerte: Alle gemessenen Größen tabellarisch darstellen
% Auswertung: Berechnung geforderter Ergebnisse mit Schritten/Fehlerformeln/Erläuterung/Grafik (Programme)
\newpage
\section{Auswertung}
\label{sec:auswertung}

Die beschriebenen Zusammenhänge werden nun angewendet und geprüft.

\subsection{Methodik}

Für eine Stichprobe $x$ aus $K$ Messpunkten $x_k$ ist der Mittelwert mit
\begin{align*}
	\mu(x) = \pfrac{1}{\hspace{-0.25ex}K\hspace{0.25ex}} \sum_{k\hspace{0.25ex} = 1}^K x_k
\end{align*}
gegeben. Über diesen ist auch die Standardabweichung
\begin{align*}
	\sigma(x) = \sqrt{\mu(x^2) - \mu(x)^2}
\end{align*}
als Streumaß definiert. Zum weiteren Vorgehen wird die Berechnung und Fortpflanzung von Abweichungen mithilfe der Bibliothek
\verb+uncertainties+ \cite{uncertainties} automatisiert. Lineare Ausgleichsrechnungen werden mit \verb+numpy+ \cite{numpy}
durchgeführt. Dabei ergeben sich die Fehler aus der jeweiligen Kovarianzmatrix. Zur Erstellung von Grafiken kommt
\verb+matplotlib+ \cite{matplotlib} zum Einsatz. Alle Programme werden unter \verb+python+ \cite{python} ausgeführt.
Passende Zeiten zur Aktivierung sowie günstige Messintervalle werden Abbildung~\ref{fig:liste} entnommen.

\subsection{Nulleffekt}

\begin{table}[H]
	\centering
	\caption{Messdaten zum Nulleffekt bei $\Delta t = \qty{10}{\second}$.}
	\makebox[\linewidth][c]{\input{build/tab_1.tex}}
	\label{tab:1}
\end{table}

Aus den Messwerten in Tabelle~\ref{tab:1} ergibt sich die gemittelte Größe
\begin{align*}
	N = \qty{0.392+-0.185}{\per\second}
\end{align*}
für die Hintergrundzählrate. Alle anschließenden Messreihen müssen durch Subtraktion von $N$ bereinigt werden, damit sich daraus
sinnvolle Werte ergeben. Zur Beurteilung des Streuverhaltens und der Annahme, dass der Nulleffekt über große Zeiträume konstant
ist, wird ein gleitender Mittelwert implementiert, der aus $n$ benachbarten Messpunkten zentriert das arithmetische Mittel
berechnet. Abbildung~\ref{fig:1} stellt dazu einige Fälle dar.

\begin{figure}[H]
	\centering
	\includegraphics[width=\linewidth]{build/plot_1.pdf}
	\caption{Streuverhalten der Hintergrundzählrate.}
	\label{fig:1}
\end{figure}

Der Bereich der Standardabweichung des Nulleffekts ist farblich hinterlegt. Es lässt sich eine gleichmäßige Konvergenz der geglätteten
Daten aus Tabelle~\ref{tab:1} gegen den Mittelwert erkennen. Die Vorraussetzung einer uniform verteilten Zufallsgröße scheint also
erfüllt zu sein, sodass mit der weiteren Auswertung fortgefahren werden kann.

\subsection{Vanadium}

Anhand des Isotops $\ce{^{52}V}$ als Aktivierungsprdukt von $\ce{^{51}V}$ wird exemplarisch ein relativ einfacher Zerfallsprozess
analysiert. Dazu werden die Werte in Tabelle~\ref{tab:2} logarithmiert, eine lineare Regression liefert dann die gesuchten Parameter.

\begin{table}[H]
	\centering
	\caption{Bereinigte Messdaten zu $\ce{^{52}V}$ bei $\Delta t = \qty{30}{\second}$.}
	\makebox[\linewidth][c]{\input{build/tab_2.tex}}
	\label{tab:2}
\end{table}

Mit dem konstanten Faktor $C = N_0 \hspace{0.25ex} \bigl(1 - \exp(-\lambda)\bigr)$ kann die Aktivität als
\begin{align*}
	N(t) = C \exp(-\lambda t)
\end{align*}
zusammengefasst werden. Entlang der logarithmischen Formulierung kann
\begin{align*}
	\ln(N(t)) = a\hspace{0.25ex}t + b
\end{align*}
als lineares Modell mit $\lambda = -a$ und $C = \exp(b)$ angesetzt werden.

\begin{figure}[H]
	\centering
	\includegraphics[width=\linewidth]{build/plot_2.pdf}
	\caption{Halblogarithmisches Zerfallsdiagramm für $\ce{^{52}V}$.}
	\label{fig:2}
\end{figure}

Die resultierende Ausgleichsgerade mit den Optimierungsparametern
\begin{align*}
	a = \qty[group-minimum-digits=6]{-0.00303+-0.00016}{\per\second} && b = \num{1.719+-0.087}
\end{align*}
ist in Abbildung~\ref{fig:2} eingetragen. Aus $a$ folgen die Zerfallskonstante
\begin{align*}
	\lambda = \qty[group-minimum-digits=6]{0.00303+-0.00016}{\per\second}
\end{align*}
und über den Zusammenhang $T = \ln (2) \hspace{0.25ex} \lambda^{-1}$ die Halbwertszeit
\begin{align*}
	T = \qty{228.4+-12.3}{\second}
\end{align*}
für $\ce{^{52}V}$. Der konstante Faktor $C$ berechnet sich aus $b$ zu
\begin{align*}
	C = \qty{5.580+-0.486}{\per\second}
\end{align*}
und gibt ein Maß für die initiale Aktivität an.

\subsection{Rhodium}

Neben dem instabilen Isotop $\ce{^{104}Rh}$ produziert die Neutronenaktivierung von $\ce{^{103}Rh}$ mit dem metastabilen $\ce{^{104m}Rh}$
auch dazu isomere Kerne. Dadurch tritt ein komplexer Zerfallsvorgang auf, bei dessen Untersuchung die Beiträge der gemischten Isotope
separiert werden müssen, bevor eine weitere Verarbeitung der Daten möglich ist.

\begin{table}[H]
	\centering
	\caption{Bereinigte Messdaten zum $\ce{^{104}Rh}$-Gemisch bei $\Delta t = \qty{8}{\second}$.}
	\makebox[\linewidth][c]{\input{build/tab_3_a.tex}}
	\label{tab:3_a}
\end{table}

\begin{figure}[H]
	\centering
	\includegraphics[width=\linewidth]{build/plot_3_a.pdf}
	\caption{Halblogarithmisches Zerfallsdiagramm für $\ce{^{104m}Rh}$ bei $\Delta t = \qty{8}{\second}$.}
	\label{fig:3_a}
\end{figure}

Anhand Abbildung~\ref{fig:3_a} wird ein Zeitpunkt $t_1 = \qty{410}{\second}$ geschätzt, ab dem die logarithmierten Daten aus
Tabelle~\ref{tab:3_a} um eine Gerade streuen, also nur noch das langlebigere $\ce{^{104m}Rh}$ zur Aktivität beiträgt. Die Messwerte
$t \geq t_1$ werden dann zur linearen Ausgleichsrechnung herangezogen. Der entsprechende Bereich ist farblich gekennzeichnet.


Die weitere Rechnung verläuft analog zum einfachen Zerfall. Mit den Koeffizienten
\begin{align*}
	a_1 = \qty[group-minimum-digits=6]{-0.00273+-0.00079}{\per\second} && b_1 = \num{1.476+-0.459}
\end{align*}
folgt aus $a_1$ die Zerfallskonstante
\begin{align*}
	\lambda_1 = \qty[group-minimum-digits=6]{0.00273+-0.00079}{\per\second}
\end{align*}
und damit die Halbwerszeit
\begin{align*}
	T_1 = \qty{254.3+-74.4}{\second}
\end{align*}
für $\ce{^{104m}Rh}$. Aus $b_1$ kann
\begin{align*}
	C_1 = \qty{4.376+-2.008}{\per\second}
\end{align*}
berechnet werden.
\\[5em]
Für den Zerfall von $\ce{^{104}Rh}$ muss die soeben bestimmte Zählrate $N_1(t) = C_1 \exp(-\lambda_1 t)$ von den in Tabelle~\ref{tab:3_a}
nachgehaltenen Messwerten abgezogen werden. Die Ergebnisse dieser Korrektur sind in Abbildung~\ref{fig:3i_a} visualisiert. Nun wird ein
Zeitpunkt $t_2 = \qty{80}{\second}$ gewählt, wobei $t_2 < t_1$ gefordert ist. Punkte $t \leq t_2$ verlaufen annähernd geradelinig und
erlauben erneut eine Regressionsrechnung. Wieder ist der verwendete Bereich farblich hervorgehoben.

\begin{figure}[H]
	\centering
	\includegraphics[width=\linewidth]{build/plot_3i_a.pdf}
	\caption{Halblogarithmisches Zerfallsdiagramm für $\ce{^{104}Rh}$ bei $\Delta t = \qty{8}{\second}$.}
	\label{fig:3i_a}
\end{figure}

Die geringste Abweichung zwischen Modell und Messung ist durch
\begin{align*}
	a_2 = \qty[group-minimum-digits=6]{-0.01703+-0.00153}{\per\second} && b_2 = \num{3.500+-0.076}
\end{align*}
gegeben. Aus $a_2$ folgt die Zerfallskonstante
\begin{align*}
	\lambda_2 = \qty[group-minimum-digits=6]{0.01703+-0.00153}{\per\second}
\end{align*}
und damit eine Halbwertzeit von
\begin{align*}
	T_2 = \qty{40.7+-3.7}{\second}
\end{align*}
für $\ce{^{104}Rh}$. Aus $b_2$ lässt sich
\begin{align*}
	C_2 = \qty{33.105+-2.517}{\per\second}
\end{align*}
bestimmen. 

\begin{figure}[H]
	\centering
	\includegraphics[width=\linewidth]{build/plot_3_a_.pdf}
	\caption{Halblogarithmisches Zerfallsdiagramm zum $\ce{^{104}Rh}$-Gemisch bei $\Delta t = \qty{8}{\second}$.}
	\label{fig:3_a_}
\end{figure}

Der Verlauf des Zerfalls der Isotopen-Mischung sollte der Superposition der Resultate
\begin{align*}
	N(t) = C_1 \exp(-\lambda_1 t) + C_2 \exp(-\lambda_2 t)
\end{align*}
gehorchen. Abbildung~\ref{fig:3_a_} erlaubt eine Beurteilung der Güte dieses Modells. Darin werden nun auch Fehlerbalken zu den
Messwerten eingetragen, welche in den Abbildungen~\ref{fig:3_a}~und~\ref{fig:3i_a} zur besseren Lesbarkeit ausgelassen sind.

Völlig analog wird eine zweite Messreihe mit größeren Zeitintervallen ausgewertet.

\begin{table}[H]
	\centering
	\caption{Bereinigte Messdaten zum $\ce{^{104}Rh}$-Gemisch bei $\Delta t = \qty{15}{\second}$.}
	\makebox[\linewidth][c]{\input{build/tab_3_b.tex}}
	\label{tab:3_b}
\end{table}

\begin{figure}[H]
	\centering
	\includegraphics[width=\linewidth]{build/plot_3_b.pdf}
	\caption{Halblogarithmisches Zerfallsdiagramm für $\ce{^{104m}Rh}$ bei $\Delta t = \qty{15}{\second}$.}
	\label{fig:3_b}
\end{figure}

Für $t_1 = \qty{420}{\second}$ ergeben sich aus der in Abbildung~\ref{fig:3_b} dargestellten linearen Regression
\begin{align*}
	a_1 = \qty[group-minimum-digits=6]{-0.00259+-0.00059}{\per\second} && b_1 = \num{1.516+-0.349}
\end{align*}
als Parameter. Mit $a_1$ folgen daraus die Zerfallskonstante
\begin{align*}
	\lambda_1 = \qty[group-minimum-digits=6]{0.00259+-0.00059}{\per\second}
\end{align*}
sowie die entsprechende Halbwertszeit
\begin{align*}
	T_1 = \qty{268.1+-60.9}{\second}
\end{align*}
für $\ce{^{104m}Rh}$. Mit $b_1$ kann der Faktor
\begin{align*}
	C_1 = \qty{4.555+-1.589}{\per\second}
\end{align*}
bestimmt werden.

\begin{figure}[H]
	\centering
	\includegraphics[width=\linewidth]{build/plot_3i_b.pdf}
	\caption{Halblogarithmisches Zerfallsdiagramm für $\ce{^{104}Rh}$ bei $\Delta t = \qty{15}{\second}$.}
	\label{fig:3i_b}
\end{figure}

Die bereinigten Werte in Abbildung~\ref{fig:3i_b} liefern mit $t_2 = \qty{210}{\second}$ auf gleiche Weise wie zuvor
\begin{align*}
	a_2 = \qty[group-minimum-digits=6]{-0.01702+-0.00051}{\per\second} && b_2 = \num{3.988+-0.065}
\end{align*}
als Regressionsparameter. Aus $a_2$ folgt die Zerfallskonstante
\begin{align*}
	\lambda_2 = \qty[group-minimum-digits=6]{0.01702+-0.00051}{\per\second}
\end{align*}
und damit die zugehörige Halbwertszeit
\begin{align*}
	T_2 = \qty{40.7+-1.2}{\second}
\end{align*}
für $\ce{^{104}Rh}$. Über $b_2$ lässt sich
\begin{align*}
	C_2 = \qty{53.941+-3.492}{\per\second}
\end{align*}
berechnen.

\begin{figure}[H]
	\centering
	\includegraphics[width=\linewidth]{build/plot_3_b_.pdf}
	\captionsetup{width=0.909\linewidth}
	\caption{Halblogarithmisches Zerfallsdiagramm zum $\ce{^{104}Rh}$-Gemisch bei $\Delta t = \qty{15}{\second}$.}
	\label{fig:3_b_}
\end{figure}

Abbildung~\ref{fig:3_b_} enthält wie Abbildung~\ref{fig:3_a_} die Superposition der Ergebnisse samt grafischer Darstellung der
fehlerbehafteten Messgrößen aus Tabelle~\ref{tab:3_b}.

Es lässt sich noch die Wahl der Zeitpunkte $t_1$ und $t_2$ überprüfen, indem in die Bedingung $N_2(t_2) \ll N_1(t_2)$ eingesetzt wird.

Für $\Delta t = \qty{8}{\second}$ ist diese Forderung mit $\qty{0.031+-0.019}{\per\second} \ll \qty{1.431+-0.807}{\per\second}$ erfüllt.

Mit $\qty{0.042+-0.009}{\per\second} \ll \qty{1.538+-0.657}{\per\second}$ genügt auch $\Delta t = \qty{15}{\second}$ der Voraussetzung.

Um die Ergebnisse zusammenzufassen, wird zuletzt für beide Messreihen der Mittelwert aufgestellt. Die Halbwertszeiten belaufen sich dann auf
$T = \qty{261.2+-48.1}{\second}$ für $\ce{^{104m}Rh}$ und $T = \qty{40.7+-1.9}{\second}$ für $\ce{^{104}Rh}$.
