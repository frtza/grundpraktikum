% Diskussion: Resultate mit Fehler/Genauigkeit zusammenstellen, Literaturwerte/Messmethoden/Ursachen vergleichen
% Literatur: Verwendete Literatur/Grafiken/Werte/Programme
% Anhang: Kopie der analog eingetragenen Messdaten
\section{Diskussion}
\label{sec:diskussion}

Als Literaturwert für $\ce{^{52}V}$ ist $T = \qty{224.6+-0.3}{s}$ gegeben \cite{52_v} und weist damit eine recht hohe Übereinstimmung
zum experimentell bestimmten Wert $T = \qty{228.4+-12.3}{\second}$ auf. Für $\ce{^{104m}Rh}$ sowie $\ce{^{104}Rh}$ gibt die Literatur
die Halbwertszeiten $T = \qty{260.4+-1.8}{s}$ sowie $T = \qty{42.3+-0.4}{s}$ an \cite{104_rh} und passt damit ebenfalls gut zu den
Auswertungsergebnissen $T = \qty{261.2+-48.1}{\second}$ sowie $T = \qty{40.7+-1.9}{\second}$.

Auffällig sind die vergleichsweise großen Fehlerintervalle für die Messung der Isotope $\ce{^{52}V}$ und $\ce{^{104m}Rh}$. Dies
ist in erster Linie darauf zurückzuführen, dass für Materialien mit längeren Lebensdauern zwingend geringere Zerfallsraten auftreten
und entsprechende Störungen der Hintergrundaktivität eine größere Streuung verursachen. Es handelt sich dabei um statistische
Abweichungen, eine größere Anzahl der Messwerte würde diese also kompensieren.

Weiterhin sei angemerkt, dass die Wahl der Zeitpunkte $t_1$ und $t_2$ mit einiger Willkür stattfindet und dementsprechend auch die
Ergebnisse nicht völlig rigoros reproduzierbar sind. Die visuelle Prüfung der Verläufe in den jeweiligen Abbildungen zeigt aber, dass
es sich trotzdem um eine relativ exakte Beschreibung des physikalischen Phänomens handelt.

Zuletzt lautet ein Nachteil für Isotope wie $\ce{^{104}Rh}$ mit Halbwertszeiten im Bereich einiger Sekunden, dass die Aktivität
schnell gegen Null geht und überdeckt wird. Dies ist anschaulich in Abbildung~\ref{fig:3i_a} zu sehen, in der nur innerhalb
$\qty{80}{\second}$ brauchbare Daten vorliegen, da hier eine längere Zeitspanne zwischen Entnahme aus der Quelle und Einsetzen
der Probe in den Detektor auftritt.

\vfill
