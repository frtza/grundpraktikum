% Messwerte: Alle gemessenen Größen tabellarisch darstellen
% Auswertung: Berechnung geforderter Ergebnisse mit Schritten/Fehlerformeln/Erläuterung/Grafik (Programme)
\section{Auswertung}
\label{sec:auswertung}

\subsection{Dampfdruck und mittlere freie Weglänge}

Mithilfe der Gleichungen \eqref{eqn:dampf1} und \eqref{eqn:dampf2} lassen sich die in Tabelle~\ref{tab:1} nachgehaltenen Ergebnisse
für die gegebenen Temperaturen bestimmen.

\begin{table}[H]
	\centering
	\caption{Zu Temperatur $T$ herrschender Dampfdruck $p$ und entsprechende mittlere freie Weglänge $\bar{w}$, sowie das Verhältnis $a / \bar{w}$ bei $a = \qty{1}{\centi\meter}$.}
	\makebox[\textwidth][c]{\input{build/table_1.tex}}
	\label{tab:1}
\end{table}

Es wird deutlich, dass nur Messung 1 die geforderte Bedingung $a / \bar{w} \sim [1000, 4000]$ deutlich verfehlt. Bei Zimmertemperatur ist die
Stoßwahrscheinlichkeit demnach nicht ausreichend, um den Franck-Hertz-Effekt zu beobachten. Alle weiteren Messungen sind dagegen gut dazu geeignet.

\subsection{Statistik zur graphischen Auswertung}

Für eine Messreihe $x_i$ mit $N$ Einzelmessungen beschreibt der Ausdruck
\begin{equation*}
	\overline{x} = \pfrac{1}{\hspace{-0.2ex} N \,} \sum_{i = 1}^{N} x_i
\end{equation*}
das arithmetische Mittel. Die zugehörige Standardabweichung folgt mit
\begin{equation*}
	\delta x = \sqrt{\hspace{0.2ex} \overline{x^2} - \overline{x}^2}
\end{equation*}
aus der Quadratwurzel der Varianz. Um den Fehler aus zwei unabhängigen Quellen zusammenzufassen, kann die idealisierte Annahme
\begin{equation*}
	\delta_{ab} x = \sqrt{\hspace{0.2ex} \delta_a x^2 + \delta_b x^2}
\end{equation*}
herangezogen werden. Auf diese Weise wird hier die statistische Streuung der Intervallbreite mit dem Ablesefehler vereint, um die Abweichung der
in Tabelle~\ref{tab:2} aufgeführten mittleren Breiten $\bar{n}$ der Skalenbereiche anzugeben. Für die abschließende Bestimmung der Anregungsenergie
wird aus fehlerbehafteten Größen das gewichtete Mittel nach
\begin{equation*}
	\overline{x}_\delta = \frac{\sum_i x_i / \delta x_i}{\sum_i 1/\delta x_i}
\end{equation*}
gebildet. Als Gewichtung dient dabei die reziproke Standardabweichung. Die zugehörige Streuung ist dann durch
\begin{equation*}
	\delta x_\delta = \sqrt{\frac{1}{\sum_i 1/\delta x_i}}
\end{equation*}
bemessen. Damit kann nun fortgefahren werden. In Tabelle~\ref{tab:2} ist vorbereitend die Skala der nachfolgenden Abbildungen ausgewertet.
Abbildungen \ref{fig:4} und \ref{fig:4*} entsprechen der vorherigen Bezeichnung mit Messung 1, Messung 2 wird in Abbildung \ref{fig:5} und
\ref{fig:5*} dargestellt. Die Messungen 3 und 4 werden je in den Abbildungen \ref{fig:6a} und \ref{fig:6b} sowie \ref{fig:7a} und \ref{fig:7b}
angezeigt. Analog werden Abbildung \ref{fig:8} und \ref{fig:9} mit Messung 5 und 6 bezeichnet. In den Darstellungen ohne * ist je die Kurve
des XY-Schreibers hinterlegt. Folglich werden die Achsen in der Horizontalen X und in der Vertikalen Y genannt, um die Beschreibung des Vorgehens zu
erleichtern. Die Einheit in Y-Richtung ist nicht relevant, da es hier nur auf eine korrekte Nullkalibrierung und den relativen Kurvenverlauf propotional
zum Auffängerstrom $I_{\hspace{-0.2ex} A}$ ankommt. Aus diesem Grund betrachtet Tabelle~\ref{tab:2} nur die X-Skalierung, für welche mittels Schreiberkopf
in regelmäßigen Stufen der Speisespannung ein Punkt markiert wird. Weiter lassen sich durch Abzählen der kleinsten Skaleneinheiten zwischen den Markern
auftretende Variationen der Schrittweite kompensieren. Wie oben beschrieben, wird aus dem Ablesefehler in der Größe einer X-Einheit und der
statistischen Standardabweichung die zusammengesetzte Ungenauigkeit der jeweiligen Skalen gebildet, die für die Entnahme der tatsächlichen Messpunkte
je als horizontale Toleranz dient. Außerdem sind in Tabelle~\ref{tab:2} noch die intervallspezifischen Einheiten der Speisespannung pro Skaleneinheit
angegeben. Diese werden zur Berechnung aus der X-Koordinate nach dem unten beschriebenen Algorithmus verwendet.

Die Skalierungspunkte entsprechen einer festen Schrittweite $\increment U$. Individuelle Bereiche zwischen den eingrenzenden Markierungen
haben die Spannungseinheit $r$ pro Kästchen. Nun wird die horizontale Kästchenzahl $x$ vom Messpunkt zum nächsten niedrigeren Grenzmarker gezählt. Indem
diese einzelnen Skalierungsmarkierungen aufsteigend mit $N = {0,1,2,...}$ bezeichnet werden, ergibt sich
\begin{equation*}
	U = N \cdot \increment U + x \cdot r
\end{equation*}
als tatsächlicher Spannungswert. Die entsprechenden $y$ Werte werden als Abstand zur Nulllinie aufgenommen, auf der die Skalenpunkte liegen,
und zur Veranschaulichung so normiert, dass sie in etwa dem Kurvenverlauf folgen.
\enlargethispage*{\parskip}
\newpage

\begin{table}[H]
	\centering
	\captionsetup{width=0.95\linewidth}
	\caption{Anzahl $n$ der Skaleneinheiten im jeweiligen Abschnitt $N$ zwischen benachbarten Skalierungspunkten. Zur Bewertung ist die
			 Spannung $r$ pro Einheit angezeigt.}
	\input{build/table_2.tex}
	\label{tab:2}
\end{table}

Da die Auswertung in erster Linie aus graphischem Ablesen von Messwerten besteht, sei an dieser Stelle die Python-Bibliothek Matplotlib \cite{matplotlib}
erwähnt, welche zur Darstellung der Daten zum Einsatz kommt. Die bei der Verarbeitung notwendige Fehlerfortpflanzung wird aufbauend auf NumPy \cite{numpy}
durch Uncertainties \cite{uncertainties} automatisiert. 

\subsection{Energiespektrum und Kontaktpotential}

\renewcommand{\thefigure}{5}
\begin{figure}[H]
	\includegraphics{build/plot1.pdf}
	\captionsetup{width=0.8\linewidth}
	\caption{Integrale Energieverteilung der mit $U_{\hspace{-0.2ex} B} = \qty{11}{\volt}$ beschleunigten Elektronen bei $T = \qty{297.45}{\kelvin}$.}
	\label{fig:4}
\end{figure}

Anhand Abbildung \ref{fig:4} lässt sich exemplarisch das Vorgehen für alle Aufzeichnungen des XY-Schreibers erklären. Das Bild wird
im Koordinatensystem wie markiert an der Null und einem weiteren Punkt fixiert. An der Zentrierung des Skalenpunktes lässt sich die Güte der
Ausrichtung beurteilen. Weiter werden die entnommenen Messdaten zur visuellen Verifikation über den Graphen gelegt. 

Um aus dem integralen das differentielle Energiespektrum zu bestimmen, werden die Beträge der Sekantensteigungen zwischen den Messwerten
\begin{equation*}
	n'_k = - \pfrac{n_{k+1} - n_k}{U_{k+1} - U_k}
\end{equation*}
ermittelt und in Abbildung~\ref{fig:4*} auf halber Strecke zwischen den ursprünglichen Punkten festgehalten. Da die Kurve um den Wendepunkt sehr steil verläuft,
ergeben sich für die Steigung extrem große Unsicherheiten, welche in der Grafik zur besseren Lesbarkeit um einen Faktor 40 reduziert sind. Die tatsächlichen
Fehler sind mit den originalen Ablesewerten in willkürlichen Einheiten $n \sim I_{\hspace{-0.2ex} A}$ aus Tabelle~\ref{tab:3} zu entnehmen.

\renewcommand{\thefigure}{5*}
\begin{figure}[H]
	\includegraphics{build/plot11.pdf}
	\captionsetup{width=0.85\linewidth}
	\caption{Differentielle Energieverteilung der mit $U_{\hspace{-0.2ex} B} = \qty{11}{\volt}$ beschleunigten Elektronen bei $T = \qty{297.45}{\kelvin}$.}
	\label{fig:4*}
\end{figure}

Damit die Lage des Wendepunktes der integralen beziehungsweise des Maximums der differentiellen Energieverteilung exakt bestimmt werden kann, 
wird ein Modell der Form
\begin{equation*}
	I_{\hspace{-0.2ex} A} \sim |U_{\hspace{-0.2ex} A} - a| \cdot b + c
\end{equation*}
angesetzt. Dieses beschreibt die Annahme, dass die Kurve aus Abbildung~\ref{fig:4*} um ihren Peak annähernd symmetrisch verläuft.
Die Parameter $b$ und $c$ liefern die nötigen Freiheitsgrade, sind sonst aber nicht relevant, sodass deren Ergebnisse ausgelassen werden.
Für die Verschiebung ergibt sich dann über eine mit SciPy \cite{scipy} durchgeführte und nach Ungenauigkeit gewichtete Regression
\begin{equation*}
	a = \input{build/a.tex}
\end{equation*}
als Wert, deren verhältnismäßig geringe Abweichung aus der Quadratwurzel der Diagonale der Kovarianzmatrix folgt. Demnach besitzt ein Großteil der
bei $U_{\hspace{-0.2ex} B} = \qty{11}{\volt}$ emittierten Elektronen eine Energie im Bereich $E = \qty{8.724+-0.007}{\electronvolt}$, die resultierende
Differenz zur Beschleunigungsspannung bemisst nach \eqref{eq:Kontaktpotential} das Kontaktpotential der Apparatur mit $K = \qty{2.276+-0.007}{\volt}$.

\begin{table}[H]
	\centering
	\caption{}
	\makebox[\textwidth][c]{\input{build/table_3.tex}}
	\label{tab:3}
\end{table}

\renewcommand{\thefigure}{6}
\begin{figure}[H]
	\includegraphics{build/plot2.pdf}
	\captionsetup{width=0.8\linewidth}
	\caption{Integrale Energieverteilung der mit $U_{\hspace{-0.2ex} B} = \qty{11}{\volt}$ beschleunigten Elektronen bei $T = \qty{418.15(5.00)}{\kelvin}$.}
	\label{fig:5}
\end{figure}

\renewcommand{\thefigure}{6*}
\begin{figure}[H]
	\includegraphics{build/plot22.pdf}
	\captionsetup{width=0.85\linewidth}
	\caption{Differentielle Energieverteilung der mit $U_{\hspace{-0.2ex} B} = \qty{11}{\volt}$ beschleunigten Elektronen bei $T = \qty{418.15(5.00)}{\kelvin}$.}
	\label{fig:5*}
\end{figure}

\begin{table}[H]
	\centering
	\caption{}
	\makebox[\textwidth][c]{\input{build/table_4.tex}}
	\label{tab:4}
\end{table}

\begin{equation*}
	s = \input{build/s.tex}
\end{equation*}

\begin{equation*}
	\input{build/z.tex}
\end{equation*}

\subsection{Anregungsenergie und Emission}

\renewcommand{\thefigure}{7a}
\begin{figure}[H]
	\includegraphics{build/plot3a.pdf}
	\caption{Franck-Hertz-Kurve bei $U_{\hspace{-0.2ex} A} = \qty{1}{\volt}$ und $T = \qty{433.15(5.00)}{\kelvin}$.}
	\label{fig:6a}
\end{figure}

\renewcommand{\thefigure}{7b}
\begin{figure}[H]
	\includegraphics{build/plot3b.pdf}
	\captionsetup{width=0.85\linewidth}
	\caption{Franck-Hertz-Kurve bei $U_{\hspace{-0.2ex} A} = \qty{1}{\volt}$ und $T = \qty{433.15(5.00)}{\kelvin}$. Vergrößerter Ausschnitt.}
	\label{fig:6b}
\end{figure}

\renewcommand{\thefigure}{8a}
\begin{figure}[H]
	\includegraphics{build/plot4a.pdf}
	\caption{Franck-Hertz-Kurve bei $U_{\hspace{-0.2ex} A} = \qty{2}{\volt}$ und $T = \qty{433.15(5.00)}{\kelvin}$.}
	\label{fig:7a}
\end{figure}

\renewcommand{\thefigure}{8b}
\begin{figure}[H]
	\includegraphics{build/plot4b.pdf}
	\captionsetup{width=0.85\linewidth}
	\caption{Franck-Hertz-Kurve bei $U_{\hspace{-0.2ex} A} = \qty{2}{\volt}$ und $T = \qty{433.15(5.00)}{\kelvin}$. Vergrößerter Ausschnitt.}
	\label{fig:7b}
\end{figure}

\renewcommand{\thefigure}{9}
\begin{figure}[H]
	\includegraphics{build/plot5.pdf}
	\caption{Franck-Hertz-Kurve bei $U_{\hspace{-0.2ex} A} = \qty{2}{\volt}$ und $T = \qty{453.15(5.00)}{\kelvin}$.}
	\label{fig:8}
\end{figure}

\renewcommand{\thefigure}{10}
\begin{figure}[H]
	\includegraphics{build/plot6.pdf}
	\caption{Franck-Hertz-Kurve bei $U_{\hspace{-0.2ex} A} = \qty{1}{\volt}$ und $T = \qty{453.15(5.00)}{\kelvin}$.}
	\label{fig:9}
\end{figure}

\begin{table}[H]
	\centering
	\caption{}
	\input{build/table_5.tex}
	\label{tab:5}
\end{table}

\begin{equation*}
	\increment \hspace{0.5ex}\overline{\hspace{-0.5ex} U}_{\hspace{-0.2ex} k} = \input{build/d.tex}
\end{equation*}
