% Messwerte: Alle gemessenen Größen tabellarisch darstellen
% Auswertung: Berechnung geforderter Ergebnisse mit Schritten/Fehlerformeln/Erläuterung/Grafik (Programme)
\section{Auswertung}
\label{sec:auswertung}

\subsection{Dampfdruck und mittlere freie Weglänge}

\begin{table}[H]
	\centering
	\caption{Zu Temperatur $T$ herrschender Dampfdruck $p$ und entsprechende mittlere freie Weglänge $\bar{w}$ sowie das Verhältnis $a / \bar{w}$ bei $a = \qty{1}{\centi\meter}$.}
	\makebox[\textwidth][c]{\input{build/table_1.tex}}
	\label{tab:1}
\end{table}

\subsection{Statistik zur graphischen Auswertung}

\subsection{Energiespektrum und Kontaktpotential}

\renewcommand{\thefigure}{4a}
\begin{figure}[H]
	\includegraphics{build/plot1.pdf}
	\captionsetup{width=0.8\linewidth}
	\caption{Integrale Energieverteilung der mit $U_{\hspace{-0.2ex} B} = \qty{11}{\volt}$ beschleunigten Elektronen bei $T = \qty{297.45}{\kelvin}$.}
	\label{fig:4a}
\end{figure}

\renewcommand{\thefigure}{4b}
\begin{figure}[H]
	\includegraphics{build/plot11.pdf}
	\captionsetup{width=0.85\linewidth}
	\caption{Differentielle Energieverteilung der mit $U_{\hspace{-0.2ex} B} = \qty{11}{\volt}$ beschleunigten Elektronen bei $T = \qty{297.45}{\kelvin}$.}
	\label{fig:4b}
\end{figure}

\begin{equation*}
	I_{\hspace{-0.2ex} A} \sim |U_{\hspace{-0.2ex} A} - a| \cdot b + c
\end{equation*}

\begin{equation*}
	a = \input{build/a.tex}
\end{equation*}

\renewcommand{\thefigure}{5a}
\begin{figure}[H]
	\includegraphics{build/plot2.pdf}
	\captionsetup{width=0.8\linewidth}
	\caption{Integrale Energieverteilung der mit $U_{\hspace{-0.2ex} B} = \qty{11}{\volt}$ beschleunigten Elektronen bei $T = \qty{418.15(5.00)}{\kelvin}$.}
	\label{fig:5a}
\end{figure}

\renewcommand{\thefigure}{5b}
\begin{figure}[H]
	\includegraphics{build/plot22.pdf}
	\captionsetup{width=0.85\linewidth}
	\caption{Differentielle Energieverteilung der mit $U_{\hspace{-0.2ex} B} = \qty{11}{\volt}$ beschleunigten Elektronen bei $T = \qty{297.45}{\kelvin}$.}
	\label{fig:5b}
\end{figure}

\begin{equation*}
	s = \input{build/s.tex}
\end{equation*}

\subsection{Anregungsenergie und Emission}

\renewcommand{\thefigure}{6a}
\begin{figure}[H]
	\includegraphics{build/plot3a.pdf}
	\label{fig:6a}
\end{figure}

\renewcommand{\thefigure}{6b}
\begin{figure}[H]
	\includegraphics{build/plot3b.pdf}
	\label{fig:6b}
\end{figure}

\renewcommand{\thefigure}{7a}
\begin{figure}[H]
	\includegraphics{build/plot4a.pdf}
	\label{fig:7a}
\end{figure}

\renewcommand{\thefigure}{7b}
\begin{figure}[H]
	\includegraphics{build/plot4b.pdf}
	\label{fig:7b}
\end{figure}

\renewcommand{\thefigure}{8}
\begin{figure}[H]
	\includegraphics{build/plot5.pdf}
	\label{fig:8}
\end{figure}

\renewcommand{\thefigure}{9}
\begin{figure}[H]
	\includegraphics{build/plot6.pdf}
	\label{fig:9}
\end{figure}

