% Messwerte: Alle gemessenen Größen tabellarisch darstellen
% Auswertung: Berechnung geforderter Ergebnisse mit Schritten/Fehlerformeln/Erläuterung/Grafik (Programme)
\section{Auswertung}
\label{sec:auswertung}

Unter Verwendung der bis zu dieser Stelle beschriebenen Zusammenhänge und Methoden werden die gewonnenen Messdaten nun untersucht.

\subsection{Dampfdruck und mittlere freie Weglänge}

Mithilfe der Gleichungen \eqref{eqn:dampf1} und \eqref{eqn:dampf2} lassen sich die in Tabelle~\ref{tab:1} nachgehaltenen Ergebnisse
für die gegebenen Temperaturen bestimmen.

\begin{table}[H]
	\centering
	\caption{Zu Temperatur $T$ herrschender Dampfdruck $p$ mit entsprechender mittleren freien Weglänge $\bar{w}$  und Verhältnis $a / \bar{w}$ bei $a = \qty{1}{\centi\meter}$.}
	\makebox[\textwidth][c]{\input{build/table_1.tex}}
	\label{tab:1}
\end{table}

Es wird deutlich, dass nur Messung 1 die geforderte Bedingung an $a / \bar{w} \sim [1000, 4000]$ deutlich verfehlt. Bei Zimmertemperatur ist die
Stoßwahrscheinlichkeit demnach nicht ausreichend, um den Franck-Hertz-Effekt zu beobachten. Alle übrigen Messungen sind dagegen gut dafür geeignet.

\subsection{Statistik zur graphischen Auswertung}

Für eine Messreihe $x_k$ mit $N$ Einzelmessungen beschreibt der Ausdruck
\begin{equation*}
	\overline{x} = \pfrac{1}{\hspace{-0.2ex} N \,} \sum_{k = 1}^{N} x_k
\end{equation*}
das arithmetische Mittel. Die zugehörige Standardabweichung folgt mit
\begin{equation*}
	\delta x = \sqrt{\hspace{0.2ex} \overline{x^2} - \overline{x}^2}
\end{equation*}
aus der Quadratwurzel der Varianz.

Um den Fehler aus zwei unabhängigen Quellen zusammenzufassen, kann die idealisierte Annahme
\begin{equation*}
	\delta_{ab} x = \sqrt{\hspace{0.2ex} \delta_a x^2 + \delta_b x^2}
\end{equation*}
herangezogen werden. Auf diese Weise wird im weiteren Verfahren die statistische Streuung der Intervallbreite mit dem Ablesefehler vereint, um die Unsicherheit
der in Tabelle~\ref{tab:2} aufgeführten mittleren Breiten $\bar{n}$ der Skalenbereiche anzugeben. Für die abschließende Bestimmung der Anregungsenergie
wird aus fehlerbehafteten Größen das gewichtete Mittel nach
\begin{equation*}
	\overline{x}_\delta = \frac{\sum_k x_k / \delta x_k}{\sum_k 1/\delta x_k}
\end{equation*}
gebildet. Als Gewichtung dient dabei die reziproke Standardabweichung. Die zugehörige Streuung ist dann durch
\begin{equation*}
	\delta x_\delta = \sqrt{\frac{1}{\sum_k 1/\delta x_k}}
\end{equation*}
bemessen. Zur Vorbereitung der eigentlichen Auswertung ist in Tabelle~\ref{tab:2} zunächst die Skala der nachfolgenden Abbildungen ausgewertet.
Abbildungen \ref{fig:4} und \ref{fig:4*} entsprechen der vorherigen Bezeichnung mit Messung 1, Messung 2 wird in Abbildung \ref{fig:5} und
\ref{fig:5*} dargestellt. Die Messungen 3 und 4 werden je in den Abbildungen \ref{fig:6a} und \ref{fig:6b} sowie \ref{fig:7a} und \ref{fig:7b}
angezeigt. Analog werden Abbildung \ref{fig:8} und \ref{fig:9} mit Messung 5 und 6 bezeichnet. In den Darstellungen ohne * ist je die Kurve
des XY-Schreibers hinterlegt. Um das nachfolgende Vorgehen zu beschreiben, werden die Achsen werden in der Horizontalen X und in der Vertikalen Y genannt.
Die Dimension der Y-Richtung ist nicht relevant, da es hier nur auf eine korrekte Nullkalibrierung und den relativen Kurvenverlauf proportional
zum Auffängerstrom $I_{\hspace{-0.2ex} A}$ ankommt. Aus diesem Grund betrachtet Tabelle~\ref{tab:2} ausschließlich die X-Skalierung, für welche mittels
Schreiberkopf in regelmäßigen Stufen der Speisespannung ein Punkt markiert wird. Durch Abzählen der Skaleneinheiten zwischen benachbarten Markern lässt sich die
auftretende Variationen der Schrittweite kompensieren. Wie oben beschrieben, wird aus dem Ablesefehler, welcher auf eine Einheit geschätzt wird, und
der statistischen Standardabweichung die zusammengesetzte Ungenauigkeit der jeweiligen Skalen gebildet, welche für die Entnahme der tatsächlichen Messpunkte
je als horizontale Toleranz Verwendung findet. Außerdem sind in Tabelle~\ref{tab:2} noch die intervallspezifischen Spannungseinheiten pro Skaleneinheit
angegeben. Diese werden weiterhin zur Berechnung der Speisespannung aus der abgelesenen Koordinate nach dem unten beschriebenen Algorithmus verwendet.

Die Skalierungspunkte entsprechen einer festen Schrittweite $\increment U$, die so eingegrenzten Bereiche werden aufsteigend mit $N = {0,1,2,...}$ bezeichnet.
Sie besitzen jeweils die Spannungseinheit $r_N$ pro Kästchen. Nun wird die horizontale Kästchenzahl $x$ vom Messpunkt zum nächstniedrigeren Grenzmarker gezählt.
Es ergibt sich dann
\begin{equation*}
	U = N \cdot \increment U + x \cdot r
\end{equation*}
als tatsächlicher Spannungswert.

Die entsprechenden $y$ Werte werden als Anzahl der vertikalen Kästchen zur Nulllinie aufgenommen, auf der die Skalenpunkte liegen,
und zur Veranschaulichung so normiert, dass sie in etwa dem Kurvenverlauf folgen.

\begin{table}[H]
	\centering
	\captionsetup{width=0.95\linewidth}
	\caption{Anzahl $n$ der Skaleneinheiten im jeweiligen Abschnitt $N$ zwischen benachbarten Skalierungspunkten. Zur Bewertung ist die
			 Spannung $r$ pro Einheit angezeigt.}
	\input{build/table_2.tex}
	\label{tab:2}
\end{table}

Da die Auswertung in erster Linie aus graphischem Ablesen von Messwerten besteht, sei an dieser Stelle die Python-Bibliothek Matplotlib \cite{matplotlib}
erwähnt, welche zur Darstellung der Daten zum Einsatz kommt. Die bei der Verarbeitung notwendige Fehlerfortpflanzung wird aufbauend auf NumPy \cite{numpy}
durch Uncertainties \cite{uncertainties} automatisiert. 

\subsection{Energiespektrum und Kontaktpotential}

\renewcommand{\thefigure}{5}
\begin{figure}[H]
	\includegraphics{build/plot1.pdf}
	\captionsetup{width=0.8\linewidth}
	\caption{Integrale Energieverteilung der mit $U_{\hspace{-0.2ex} B} = \qty{11}{\volt}$ beschleunigten Elektronen bei $T = \qty{297.45}{\kelvin}$.}
	\label{fig:4}
\end{figure}

Anhand Abbildung \ref{fig:4} lässt sich exemplarisch das Vorgehen für alle Aufzeichnungen des XY-Schreibers erklären. Das Bild wird
im Koordinatensystem wie markiert an der Null und einem weiteren Punkt fixiert. An der Zentrierung des Skalenpunktes lässt sich die Güte der
Ausrichtung beurteilen. Weiter werden die entnommenen Messdaten zur visuellen Verifikation über den Graphen gelegt. 

Um aus dem integralen das differentielle Energiespektrum zu bestimmen, werden die Beträge der Sekantensteigungen zwischen den Messwerten
\begin{equation*}
	n'_k = - \pfrac{n_{k+1} - n_k}{U_{k+1} - U_k}
\end{equation*}
ermittelt und in Abbildung~\ref{fig:4*} auf halber Strecke zwischen den ursprünglichen Punkten festgehalten. Da die Kurve um den Wendepunkt sehr steil verläuft,
ergeben sich für die Steigung extrem große Unsicherheiten, welche in der Grafik zur besseren Lesbarkeit um einen Faktor 40 reduziert sind. Die tatsächlichen
Fehler sind mit den originalen Ablesewerten in willkürlichen Einheiten $n \sim I_{\hspace{-0.2ex} A}$ aus Tabelle~\ref{tab:3} zu entnehmen.

\renewcommand{\thefigure}{5*}
\begin{figure}[H]
	\includegraphics{build/plot11.pdf}
	\captionsetup{width=0.85\linewidth}
	\caption{Differentielle Energieverteilung der mit $U_{\hspace{-0.2ex} B} = \qty{11}{\volt}$ beschleunigten Elektronen bei $T = \qty{297.45}{\kelvin}$.}
	\label{fig:4*}
\end{figure}

Damit die Lage des Wendepunktes der integralen oder äquivalent des Maximums der differentiellen Energieverteilung exakt bestimmt werden kann, 
wird ein Modell der Form
\begin{equation*}
	I_{\hspace{-0.2ex} A} \sim |U_{\hspace{-0.2ex} A} - a| \cdot b + c
\end{equation*}
angesetzt. Dieses beschreibt die Annahme, dass die Kurve aus Abbildung~\ref{fig:4*} um ihren Peak annähernd symmetrisch verläuft.
Die Parameter $b$ und $c$ liefern die nötigen Freiheitsgrade, sind sonst aber nicht relevant, sodass deren Ergebnisse ausgelassen werden.
Für die Verschiebung ergibt sich dann über eine mit SciPy \cite{scipy} durchgeführte und nach Ungenauigkeit gewichtete Regression
\begin{equation*}
	a = \input{build/a.tex}
\end{equation*}
als Wert, dessen verhältnismäßig geringe Abweichung aus der Quadratwurzel der Diagonale der Kovarianzmatrix folgt. Demnach besitzt ein Großteil der
bei $U_{\hspace{-0.2ex} B} = \qty{11}{\volt}$ emittierten Elektronen eine Energie im Bereich $E = \qty{8.724+-0.007}{\electronvolt}$, die resultierende
Differenz zur Beschleunigungsspannung bemisst nach \eqref{eq:Kontaktpotential} das Kontaktpotential der Apparatur mit $K = \qty{2.276+-0.007}{\volt}$.

\begin{table}[H]
	\centering
	\captionsetup{width=\linewidth}
	\caption{Abgelesene Punkte aus Abbildung~\ref{fig:4} mit den abgeleiteten Werten, die in Abbildung~\ref{fig:4*} aufgetragen sind. Entsprechen einer
			 Messung bei $U_{\hspace{-0.2ex} B} = \qty{11}{\volt}$ und $T = \qty{297.45}{\kelvin}$.}
	\makebox[\textwidth][c]{\scalebox{0.99}{\input{build/table_3.tex}}}
	\label{tab:3}
\end{table}

\renewcommand{\thefigure}{6}
\begin{figure}[H]
	\includegraphics{build/plot2.pdf}
	\captionsetup{width=0.8\linewidth}
	\caption{Integrale Energieverteilung der mit $U_{\hspace{-0.2ex} B} = \qty{11}{\volt}$ beschleunigten Elektronen bei $T = \qty{418.15(5.00)}{\kelvin}$.}
	\label{fig:5}
\end{figure}

\renewcommand{\thefigure}{6*}
\begin{figure}[H]
	\includegraphics{build/plot22.pdf}
	\captionsetup{width=0.85\linewidth}
	\caption{Differentielle Energieverteilung der mit $U_{\hspace{-0.2ex} B} = \qty{11}{\volt}$ beschleunigten Elektronen bei $T = \qty{418.15(5.00)}{\kelvin}$.}
	\label{fig:5*}
\end{figure}

\begin{table}[H]
	\centering
	\captionsetup{width=1.05\linewidth}
	\caption{Abgelesene Punkte aus Abbildung~\ref{fig:5} mit den abgeleiteten Werten, die in Abbildung~\ref{fig:5*} aufgetragen sind. Entsprechen einer
			 Messung bei $U_{\hspace{-0.2ex} B} = \qty{11}{\volt}$ und $T = \qty{418.15+-5.00}{\kelvin}$.}
	\makebox[\textwidth][c]{\scalebox{0.99}{\input{build/table_4.tex}}}
	\label{tab:4}
\end{table}

Auf analoge Weise wird für Messung 2 die differentielle Energieverteilung in Abbildung~\ref{fig:5*} bestimmt. Die zugehörigen Daten können aus
Tabelle \ref{tab:4} entnommen werden, wobei die vertikalen Fehlerbalken in der Grafik der Klarheit wegen um einen Faktor 4 reduziert sind. Der
grundlegende Verlauf der differentiellen Darstellung ist stark vereinfacht hinter die Datenpunkte gelegt und setzt sich aus zwei Plateaus sowie
deren Verbindungsgerade zusammen. Damit können einige charakteristische Phänomene der angewendeten Messmethode beschrieben werden. Einerseits ist
der Sättigungsdampfdruck bei der auf $\qty{145}{\celsius}$ erhöhten Temperatur schon so groß, dass die mittlere freie Weglänge $\bar{w}$ laut
Tabelle~\ref{tab:1} bereits um den Faktor $\num{1.37(0.27)e3}$ kleiner als der Abstand $a = \qty{1}{\centi\meter}$ ist. Die Elektronen erfahren
daher auf dem Weg von Kathode zu Elektrode vielfach elastische Stöße. Ihr Energiespektrum flacht sich bei dieser Streuung derartig ab, dass es
innerhalb der konstanten Bereiche als annähernd uniform beobachtet wird. Der Sprung zwischen den beiden Stufen lässt sich auf die Anregung der
Quecksilber-Atome zurückführen. An dieser Stelle wird auf den nächsten Abschnitt vorgegriffen, um das dazu ermittelte Energieniveau
$\increment E = \qty{5.06(0.13)}{\electronvolt}$ anzugeben. Die Differenz
\begin{equation*}
	\increment \hspace{0.5ex}\overline{\hspace{-0.5ex} U}_{\hspace{-0.2ex} k} - K = \input{build/z.tex}
\end{equation*}
liegt in ungefährer Umgebung der grob genäherten Sprungspannung
\begin{equation*}
	s = \input{build/s.tex}
\end{equation*}
und entspricht dann der durch das Kontaktpotential gesenkten Anregungsspannung. Die meisten Elektronen mit Austrittsenergien oberhalb dieses Gebiets stoßen also
inelastisch mit den Atomen, geben dabei genau das Energiequant $\increment \hspace{0.5ex}\overline{\hspace{-0.5ex} E}_{\hspace{-0.2ex} k}$ ab und
werden nach weiteren elastischen Stößen Teil des niederenergetischen Plateaus des Auffängerstroms. Auf diese Weise entsteht ein idealerweise diskreter Einbruch
des Spektrums, der hier aufgrund weiterer Fehlerquellen nur näherungsweise realisiert wird.

\subsection{Anregungsenergie und Emission}

Zur Bestimmung der Anregungsenergie werden nun die für diesen Versuch typischen Franck-Hertz-Kurven aufgenommen. Indem die Analyse verschiedene
Arbeitsbereiche der Apparatur bezüglich Temperatur und Bremsspannung berücksichtigt, lassen sich etwaige Störeinflüsse reduzieren. Aus den
unten abgebildeten Aufzeichnungen des XY-Schreibers werden die den jeweiligen Maxima entsprechenden Werte der Beschleunigungsspannung entnommen.
Aus deren Differenzen berechnet sich die Anregungsenergie. Die Ergebnisse sind in Tabelle~\ref{tab:5} festgehalten.

\renewcommand{\thefigure}{7a}
\begin{figure}[H]
	\includegraphics{build/plot3a.pdf}
	\caption{Franck-Hertz-Kurve bei $U_{\hspace{-0.2ex} A} = \qty{1}{\volt}$ und $T = \qty{433.15(5.00)}{\kelvin}$.}
	\label{fig:6a}
\end{figure}

\renewcommand{\thefigure}{7b}
\begin{figure}[H]
	\includegraphics{build/plot3b.pdf}
	\captionsetup{width=0.85\linewidth}
	\caption{Franck-Hertz-Kurve bei $U_{\hspace{-0.2ex} A} = \qty{1}{\volt}$ und $T = \qty{433.15(5.00)}{\kelvin}$. Vergrößerter Ausschnitt.}
	\label{fig:6b}
\end{figure}

\renewcommand{\thefigure}{8a}
\begin{figure}[H]
	\includegraphics{build/plot4a.pdf}
	\caption{Franck-Hertz-Kurve bei $U_{\hspace{-0.2ex} A} = \qty{2}{\volt}$ und $T = \qty{433.15(5.00)}{\kelvin}$.}
	\label{fig:7a}
\end{figure}

\renewcommand{\thefigure}{8b}
\begin{figure}[H]
	\includegraphics{build/plot4b.pdf}
	\captionsetup{width=0.85\linewidth}
	\caption{Franck-Hertz-Kurve bei $U_{\hspace{-0.2ex} A} = \qty{2}{\volt}$ und $T = \qty{433.15(5.00)}{\kelvin}$. Vergrößerter Ausschnitt.}
	\label{fig:7b}
\end{figure}

\renewcommand{\thefigure}{9}
\begin{figure}[H]
	\includegraphics{build/plot5.pdf}
	\caption{Franck-Hertz-Kurve bei $U_{\hspace{-0.2ex} A} = \qty{2}{\volt}$ und $T = \qty{453.15(5.00)}{\kelvin}$.}
	\label{fig:8}
\end{figure}

\renewcommand{\thefigure}{10}
\begin{figure}[H]
	\includegraphics{build/plot6.pdf}
	\caption{Franck-Hertz-Kurve bei $U_{\hspace{-0.2ex} A} = \qty{1}{\volt}$ und $T = \qty{453.15(5.00)}{\kelvin}$.}
	\label{fig:9}
\end{figure}

\begin{table}[H]
	\centering
	\vspace{-3ex}
	\caption{Aus den genannten Abbildungen entnommene Peaks mit entsprechenden Differenzen der jeweils benachbarten Spannungswerte.}
	\input{build/table_5.tex}
	\label{tab:5}
\end{table}

Der über die Standardabweichung gewichtete Mittelwert der Spannungsdifferenzen ist mit
$\increment \hspace{0.5ex}\overline{\hspace{-0.5ex} U}_{\hspace{-0.2ex} k} = \input{build/d.tex}$ angegeben, folglich bemisst
\begin{equation*}
	\increment E = \qty{5.06(0.13)}{\electronvolt}
\end{equation*}
die erste Anregungsenergie der Quecksilberatome. Nach Formel~\eqref{eqn:freq} ergibt sich
\begin{equation*}
	\nu = \input{build/nu.tex}
\end{equation*}
als zugehörige Frequenz, wobei $h = \input{build/h.tex}$ die Planck-Konstante bezeichnet. Einsetzen der Lichtgeschwindigkeit
$c = \input{build/c.tex}$ in den Term $c = \lambda \nu$ liefert
\begin{equation*}
	\lambda = \input{build/lam.tex}
\end{equation*}
für die Wellenlänge. Die Naturkonstanten werden hierzu aus SciPy \cite{scipy} abgegriffen.

