% Diskussion: Resultate mit Fehler/Genauigkeit zusammenstellen, Literaturwerte/Messmethoden/Ursachen vergleichen
% Literatur: Verwendete Literatur/Grafiken/Werte/Programme
% Anhang: Kopie der analog eingetragenen Messdaten
\section{Diskussion}
\label{sec:diskussion}

Aus der originalen Literatur \cite{franck_hertz} können die Werte $\increment E = \qty{4.9}{\electronvolt}$ sowie die dazugehörige Quecksilberresonanzlinie
$\lambda = \qty{253.6}{\nano\meter}$ entnommen werden. Im Vergleich der Ergebnisse dieses Versuchs zeigen $\increment E = \qty{5.06(0.13)}{\electronvolt}$
als Energiedifferenz sowie daraus folgend $\lambda = \input{build/lam.tex}$ als Wellenlänge der beim Übergang in den Grundzustand emittierten Strahlung eine
gute Übereinstimmung. Anders als etwa bei Neon tritt im verwendeten Quecksilberdampf keine sichtbare Bandenstruktur für die erste Anregungsenergie
$\increment E$ auf, da $\lambda$ im ultravioletten Bereich liegt. 

Der Energieverlust durch den zentralen elastischen Stoß der Elektronen kann für die Auswertung weitestgehend vernachlässigt werden. Durch auftretende
Verfälschung der Geschwindigkeitskomponente in Richtung der Auffängerelektrode sorgen elastische Stöße im Gegenfeldbereich allerdings für eine Abflachung und
Verbreiterung des Kurvenverlaufs. Ersterer Einfluss erschwert zunächst nur das Ablesen und sollte statistisch ausgleichbar sein. Eine Verzerrung
mit einhergehender Vergrößerung der Maximumsabstände bietet aber eine mögliche Erklärung für eine leichte systematische Abweichung nach oben, die sich
hier zwischen Ergebnis und Literaturwert andeutet.

Weitere Feherquellen könnten durch den XY-Schreiber eingeführt werden, da dieser die Aufzeichnung der Graphen durch mechanische Bewegung vornimmt.
Entsprechend ist der Kurvenverlauf anfällig für plötzliche Vibrationen. Solche sollten aber in der Auswertung als eindeutig unstetige Stellen auffallen.
Bei der Justierung des Schreibers fällt außerdem ein leichter Drift auf, der Kopf bleibt bei konstanter Spannung also nicht direkt stehen. 

Auch die manuelle Einstellung der jeweiligen Speisespannung ist fehleranfällig, da diese automatisch mit konstanter Rate ansteigt und beim gewünschten
Wert angehalten werden muss. Dies ist zwar hilfreich zur Erstellung gut lesbarer Kurven am XY-Schreiber, hat aber zur Folge, dass die Skalierungsschritte
teilweise deutliche Abweichungen aufweisen. 

Zuletzt sei noch erwähnt, dass die Temperatur aufgrund fehlender Skaleneinheiten nur sehr ungenau und mit anhaltender Schwankung einstellbar ist. Dies
sollte für die relevanten Schritte mit den großzügig abgeschätzten Fehlern berücksichtigt sein, für die Bestimmung von $\increment E$ ist dies aber
größtenteils irrelevant.
\newpage
