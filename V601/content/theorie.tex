% Theorie: Physikalische Grundlagen von Versuch/Messverfahren, Gleichungen ohne Herleitung knapp erklären
\section{Theorie}
\label{sec:theorie}

Der Frank-Hertz Versuch zählt zu dem Elektronenstoßexperimenten, welcher zur Untersuchung von elektronenhüllen dient.
Es werden Quecksilber-Atome mit Elektronen beschossen, sodass elastische und inelastische Wechselwirkungen entstehen.
Wenn es zu einem inelastischen Stoß kommt wird das Quecksilber-Atom aus seinem Grundzustand $E_0$ in den 
ersten Zustand $E_1$ gehoben. Für die Differenzen lässt sich das Verhältnis 
\begin{equation*}
    \frac{m_0 \cdot v_{\taxt{vor}}^2}{2} - \frac{m_0 \cdot v_{\taxt{nach}}^2}{2} = E_1 - E_0
\end{equation*}
aufstellen.