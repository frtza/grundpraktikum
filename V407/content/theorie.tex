% Theorie: Physikalische Grundlagen von Versuch/Messverfahren, Gleichungen ohne Herleitung knapp erklären
\section{Theorie}
\label{sec:theorie}

Als Grundlage des Versuches dient die elektromagnetische Wellentheorie, wobei die Ausbreitung von Licht 
mit Hilfe der Maxwellschen Gleichungen 
\begin{align}
    \nabla \times \vec{H}&=\vec{j}+\varepsilon \varepsilon_{0} \partial_t \vec{E} \quad \text{und} \\
    \nabla \times \vec{E}&=-\mu \mu_{0} \partial_t \vec{H}
    \label{eqn:maxwell}
\end{align}
beschrieben wird.
Im Folgenden werden nicht-ferromagnetische und nicht elektrisch leitende Materialien betrachtet, somit gilt $\mu \approx 1$
und $\vec{j} =0$.
Die elektrische und magnetische Arbeit 
\begin{align*}
    W_\text{{elektrisch}} &\coloneq \frac{1}{2} \varepsilon \varepsilon_0 \vec{E}^2 \quad \text{und}\\
    W_{\text{magnetisch}} &\coloneq \frac{1}{2} \mu_0 \vec{H}^2
\end{align*}
stellen den Zusammenhang zwischen Energie pro Volumeneinheit eines elektrischen beziehungsweise magnetischen Feldes dar.
Der Poynting Vektor 
\begin{align}
    \vec{S} &= \vec{E} \times \vec{H} \quad  \text{und}\\
    |\vec{S}| &= v \varepsilon \varepsilon_0 \vec{E}^2
    \label{eqn:poynting}
\end{align}
besitzt die Dimension Leistung/Fläche und stellt die Strahlungsleistung pro Flächeneinheit eines 
elektromagnetischen Feldes dar.
Beim Einfallen einer Welle aus dem Vakuum auf eine Grenzfläche unter einem Winkel $\alpha$, wird ein Bruchteil dieser
refelktiert und der andere dringt in das Medium ein. Der Lichtstrahl, welcher in das Medium eindringt erfährt eine Richtungsänderung 
und wird so gebrochen, dass der Beugungswinkel $\beta < \alpha$ ist. Es werden nur nicht absorbierende Medien verwendet und es gilt somit
\begin{align*}
    \symup{S}_e \symup{F}_e &= \symup{S}_r \symup{F}_e + \symup{S}_d \symup{F}_d \quad \text{oder}\\
    \symup{S}_e \cos \alpha &= \symup{S}_r \cos \alpha + \symup{S}_d \cos \beta .
\end{align*}
Diese Gleichung kann umgeschrieben werden zu 
\begin{equation}
        c \varepsilon_0 \vec{E}_e^2 \cos \alpha=c \varepsilon_0 \vec{E}_r^2 \cos \alpha+v \varepsilon \varepsilon_0 \vec{E}_d^2 \cos \beta.
        \label{eqn:strahlung}
\end{equation}
Für den Brechungsindex ergibt sich das Verhältnis
\begin{equation}
    n = \frac{c}{v}.
    \label{eqn:brechungsindex}
\end{equation}
Aus den Maxwellschen Gleichungen \eqref{eqn:maxwell} ergibt sich die Maxwellsche Relation
\begin{equation}
    n = \varepsilon^2 .
    \label{eqn:relation}
\end{equation}
Aus der Maxwellschen Relation \eqref{eqn:relation} und der \autoref{eqn:strahlung} ergibt sich 
\begin{equation}
    \left(\vec{E}_e^2-\vec{E}_r^2\right) \cos \alpha=\mathrm{n} \vec{E}_d^2 \cos \beta .
\end{equation}

Die Polarisationsrichtung der einfallenden Welle $\vec{E}_e$ relativ zur Einfallsebene ist entweder senkrecht polarisiert oder parallel polarisiert,
sodass
\begin{equation}
        \vec{E}_e=\vec{E}_{\perp}+\vec{E}_{\|}
\end{equation}
gegeben ist.
Zunächst wird die Polarisation senkrecht zur Einfallsebene betrachtet. Für den parallel polarisierten Teil $\vec{E}_{\|}$ geht hervor, dass 
dieser tangential zur Grenzfläche schwingt. In der \autoref{fig:bild1} wird die Reflexion eines Lichtstrahls an einer Grenzfläche 
dargestellt.

\begin{figure}[H]
	\centering
	\includegraphics[width=0.6\linewidth]{content/grafik/bild1.png}
	\caption{Reflexion und Brechung des senkrecht polarisierten Lichtstrahls. \cite{fresnel}}
	\label{fig:bild1}
\end{figure}

Da die Beträge der $\vec{E}_{\perp}$ gleich ihren Tangentialkomponenten sind und keine Normalkomponente vorhanden ist kann aus den
Stetigkeitsbedingungen die Beziehung 
\begin{equation*}
    \vec{E}_{e\perp} + \vec{E}_{r \perp} = \vec{E}_{d\perp}
\end{equation*}
aufgestellt werden.  
Zusammen mit dem Snellius Brechungsgesetz
\begin{equation}
    n = \frac{\sin \alpha}{\sin \beta}
    \label{eqn:snellius}
\end{equation}
ergeben sich die Fresnel Formeln
\begin{equation}
    \begin{aligned}
    \vec{E}_{\mathrm{r}_{\perp}}&=-\vec{E}_{\mathrm{e}_{\perp}} \frac{\sin (\alpha-\beta)}{\sin (\alpha+\beta)} \quad \text{und}\\
    \vec{E}_{\mathrm{r}_{\perp}}&=-\vec{E}_{\mathrm{e}_{\perp}} \frac{\left(\sqrt{\mathrm{n}^2-\sin ^2 \alpha}-\cos \alpha\right)^2}{\mathrm{n}^2-1} .
    \label{eqn:fresnel1}
    \end{aligned}
\end{equation}

Für den streifenden Einfall $\alpha = \frac{\pi}{2}$ gilt
\begin{equation*}
    \vec{E}_{r\perp}(\frac{\pi}{2}) = - \vec{E}_{r\perp}.
\end{equation*}
Wenn der Lichtstrahl senkrecht einfällt, also bei $\alpha = 0$ gilt
\begin{equation*}
    \vec{E}_{r\perp}(0) = - \vec{E}_{r\perp}\frac{n - 1}{n + 1}.
\end{equation*}


Die Reflexion und Brechung des parallel zur Einfallsebene einfallenden Strahls ist in \autoref{fig:bild2} dargestellt.

\begin{figure}[H]
	\centering
	\includegraphics[width=0.9\linewidth]{content/grafik/bild2.png}
	\caption{Reflexion und Brechung des parallel polarisierten Lichtstrahls. \cite{fresnel}}
	\label{fig:bild2}
\end{figure}

Die parallel polarisierte Komponente $\vec{E}_{\|}$ setzt sich zusammen aus einer tangentialen Komponente $\vec{E}_{\|tg}$
und eine Komponente, welche normal zu Grenzfläche ist.

Aus den Stetigkeitsbedingungen und den Tangentialkomponenten der Vektoren $\vec{E}_{e\|}$, $\vec{E}_{r\|}$ und $\vec{E}_{d\|}$
ergibt sich die Gleichung
\begin{equation}
    \vec{E}_{r\|} = \vec{E}_{e\|} \frac{n \cos \alpha - \cos \beta}{n \cos \alpha + \cos \beta}.
    \label{eqn:keinname}
\end{equation}
Für das parallel polarisierte Licht lassen sich ebenfalls die Fresnelschen Gleichungen aufstellen
\begin{equation}
    \begin{aligned}
    \vec{E}_{r \|}&=\vec{E}_e \| \frac{\tan (\alpha-\beta)}{\tan (\alpha+\beta)} \quad \text{und}\\
    \vec{E}_{r \|}(\alpha)&=\vec{E}_e \| \frac{n^2 \cos \alpha-\sqrt{n^2-\sin ^2 \alpha}}{n^2 \cos \alpha+\sqrt{n^2-\sin ^2 \alpha}} .
    \label{eqn:fresnel2}
    \end{aligned}
\end{equation}
Für den senkrechten Einfall $ \alpha = 0$ gilt
\begin{equation*}
    \vec{E}_{r\|}(0) = \vec{E}_{e\|} \frac{n - 1}{n + 1}
\end{equation*}
und für den streifenden Fall $\alpha= \frac{\pi}{2}$ gilt
\begin{equation*}
    \vec{E}_{r\|}(\frac{\pi}{2}) = -\vec{E}_{e\|}.
\end{equation*}

Fällt Licht unter einem Winkel $\alpha_0$, dem sogenannten Brewsterschen Winkel, auf die Grenzfläche, so wird dieses
nicht mehr reflektiert sondern dringt ganz in das brechende Medium ein.
