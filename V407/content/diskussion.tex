% Diskussion: Resultate mit Fehler/Genauigkeit zusammenstellen, Literaturwerte/Messmethoden/Ursachen vergleichen
% Literatur: Verwendete Literatur/Grafiken/Werte/Programme
% Anhang: Kopie der analog eingetragenen Messdaten
\section{Diskussion}
\label{sec:diskussion}

Als Literaturwert kann für $\ce{Si}$ ein Brechungsindex von $n = \num{3.805}$ bei einer Wellenlänge
$\lambda = \qty{681.20}{\nano\meter}$ angenommen werden \cite{asp_the_silicon}.

Bei der Messung senkrechter Polarisation besitzt der skalierte Optimierungsparameter $n = \input{build/nnn_s.tex}$ die beste Übereinstimmung,
der einfache Fit wie auch der mittlere Brechungsindex liegen mit $n = \input{build/f_nnn_s.tex}$ und $n = \input{build/nn_s.tex}$ deutlich darunter.
Als Konsequenz der nach Fehlergröße gewichteten Regressionseingabe zeigt der einfache Fit in Abbildung \ref{fig:plot_comp_s} für kleine Winkel
eine gute Übereinstimmung mit den Messwerten, da in diesem Bereich durch Wahl der Skala geringere Abweichungen auftreten. Bei größeren Winkeln
divergiert der Fit von der Messverteilung, nähert sich aber der Theoriekurve an, welche zuvor unterhalb der approximierten Messung liegt.

Für parallel polarisiertes Licht verträgt sich die Optimierung $n = \input{build/f_nnn_p.tex}$ am besten mit dem Literaturwert, dicht
gefolgt vom gewichteten Mittelwert $n = \input{build/nn_p.tex}$ der Messreihe. Der skalierte Fit liegt mit $n = \input{build/nnn_p.tex}$ oberhalb
der erwarteten Größe. Entsprechend ist das Minimum der durch die Regression über Skalierungsfaktor genäherten Messreihe in
Abbildung \ref{fig:plot_comp_p} zu etwas größeren Winkeln hin verschoben, mit
\begin{equation*}
	\alpha_0 = \text{arctan} \, n
\end{equation*}
entspricht es dem Brewsterwinkel, bei dem der parallel polarisierte Lichtanteil vollständig absorbiert wird. Die Kurven aus Theorie und einfacher
Ausgleichsrechnung stimmen wegen ihrer ähnlichen Ergbnisse gut miteinander überein, nähern sich den Messergebnissen aber nur im Bereich des
Intensitätsminimums lokal an. 

Es fällt auf, dass die gemessenen Amplitudenverhältnisse beider Polarisationsrichtungen Stauchungsfaktoren von $\input{build/s_s.tex}$
beziehungsweise $\input{build/s_p.tex}$ aufweisen. Diese Tatsache weist auf die naheliegende Korrektur hin, statt der ungefilterten
Gesamtintensität jeweils $I_0$ auf die direkte Intensität der betrachteten Polarisationsrichtung zu setzen. Damit ist dann auch die
konkrete Gültigkeit der Fresnelschen Formeln erfüllt, welche sich auf die Verhältnisse gleich polarisierter Wellen beschränkt, und
erfordert in jedem Fall nur eine zusätzliche Messung. Es ist davon auszugehen, dass die Berücksichtigung dieses Vorgehens die wesentlichen
Abweichungen der Auswertungsergebnisse aufheben würde.

Eine weitere mögliche Fehlerquelle ist durch den Einfluss oxidierter Oberflächenschichten gegeben, da $\ce{SiO2}$ bei
$\lambda = \qty{680.00}{\nano\meter}$ einen geringeren Brechungsindex von $n = \num{1.456}$ besitzt \cite{mal_fused_silica}.
Weiter kann spekuliert werden, dass eine anisotrope Anordnung der Kristall-Struktur polarisationsspezifische Abweichungen hervorruft.
Das gewichtete Mittel aller Resultate beträgt $n = \input{nn.tex}$ und liegt damit zwischen den bekannten Werten für $\ce{SiO2}$ und $\ce{Si}$.
Es ist ebenso nicht auszuschließen, dass andere Verunreinigungen des Spiegels zu verfälschten Ergebnissen beitragen.

Zuletzt können bei Justierung von Goniometer und Detektor Fehlereinflüsse auftreten, indem der Winkel der Spiegelfläche oder die Detektoröffnung
unpräzise eingestellt und positioniert werden. So entstehende Abweichungen sollten aber generell durch mehrfache Messung kompensierbar sein.

\enlargethispage*{\baselineskip}
\newpage

