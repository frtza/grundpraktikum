% Messwerte: Alle gemessenen Größen tabellarisch darstellen
% Auswertung: Berechnung geforderter Ergebnisse mit Schritten/Fehlerformeln/Erläuterung/Grafik (Programme)
\section{Auswertung}
\label{sec:auswertung}

\subsection{Senkrechte Polarisation}

\begin{figure}[H]
	\includegraphics{build/plot_is.pdf}
	\caption{}
	\label{fig:plot_is}
\end{figure}

\begin{table}[H]
	\caption{}
	\makebox[\textwidth]{
		\centering
		\input{build/table_s.tex}}
	\label{tab:s}
\end{table}

$n = \input{nn_s.tex}$

\begin{figure}[H]
	\includegraphics{build/plot_s.pdf}
	\caption{}
	\label{fig:plot_s}
\end{figure}

$n = \input{nnn_s.tex}$ $s = \input{s_s.tex}$

$n = \input{f_nnn_s.tex}$

\subsection{Parallele Polarisation}

\begin{figure}[H]
	\includegraphics{build/plot_ip.pdf}
	\caption{}
	\label{fig:plot_ip}
\end{figure}

\begin{table}[H]
	\caption{}
	\makebox[\textwidth]{
		\centering
		\input{build/table_p.tex}}
	\label{tab:p}
\end{table}

$n = \input{nn_p.tex}$

\begin{figure}[H]
	\includegraphics{build/plot_p.pdf}
	\caption{}
	\label{fig:plot_p}
\end{figure}

$n = \input{nnn_p.tex}$ $s = \input{s_p.tex}$

$n = \input{f_nnn_p.tex}$

$n = \input{nn.tex}$
