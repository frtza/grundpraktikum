% Messwerte: Alle gemessenen Größen tabellarisch darstellen
% Auswertung: Berechnung geforderter Ergebnisse mit Schritten/Fehlerformeln/Erläuterung/Grafik (Programme)
\section{Auswertung}
\label{sec:auswertung}
Im Folgenden wird die Kennlinie des Geiger-Müller Zählrohrs bestimmt. Die Totzeit wird zunächst über die Zwei-Quellen-Methode
und im Anschluss über das Osziloskop bestimmt.

\subsection{Fehlerrechnung}
\label{sec:Fehlerrechnung}
Die Fehlerrechnung, für die Bestimmung der Messunsicherheiten, wird mit Uncertainties \cite{uncertainties} gemacht.
Für die Formel der Gauß Fehlerfortpflanzung ist gegeben durch
\begin{equation}
    \Delta f=\sqrt{\sum_{i=1}^N\left(\frac{\partial f}{\partial x_i}\right)^2 \cdot\left(\Delta x_i\right)^2}.
    \label{eqn:gauss}
\end{equation}
Für den Mittelwert gilt 
\begin{equation}
    \bar{x} = \frac{1}{N}\sum\limits_{i = 1}^N x_i .
    \label{eqn:mittelwert}
\end{equation}
Der Fehler des Mittelwertes ist gegeben durch 
\begin{equation}
    \Delta \bar{x}=\frac{1}{\sqrt{N}} \sqrt{\frac{1}{N-1} \sum_{i=1}^N\left(x_i-\bar{x}\right)^2}.
    \label{eqn:mittelwertfehler}
\end{equation}

\subsection{Kennlinie des Geiger-Müller-Zählrohrs}
\label{sec:Kennlinie des Geiger-Müller-Zählrohrs}

Die aufgenommenen Messwerte zur Bestimmung der Kennlinie des Geiger-Müller-Zählrohrs sind in der \autoref{tab:kennlinie}
dargestellt. Zudem wurde mit Hilfe der \autoref{eqn:} der statistische Fehler $\lambda$ bestimmt und ebenfalls aufgelistet.

\begin{table}
    \centering
    \caption{Messdaten zur Bestimmung der Kennlinie des Geiger-Müller-Zählrohrs}
    \label{tab:kennlinie}
\begin{tabular}{c c c c}
    \toprule
    $U \mathbin{/} \mathrm{V}$ & $N$ &  $I \mathbin{/} \unit{\micro\ampere}$ & $\lambda$\\
    \midrule
         330 &    17211 &   0.2 & 131\\
         350 &    18298 &   0.2 & 135\\
         370 &    18392 &   0.3 & 136\\
         390 &    18818 &   0.4 & 137\\
         410 &    18653 &   0.4 & 137\\
         430 &    18946 &   0.5 & 138\\
         450 &    18915 &   0.6 & 138\\
         470 &    18905 &   0.7 & 137\\
         490 &    18934 &   0.8 & 138\\
         510 &    18970 &   0.8 & 138\\
         530 &    19015 &   0.8 & 138\\
         550 &    19336 &   0.9 & 139\\
         570 &    19235 &     1 & 139\\
         590 &    19174 &     1 & 138\\
         610 &    19224 &   1.1 & 139\\
         630 &    18991 &   1.2 & 138\\
         650 &    19082 &   1.2 & 138\\
         670 &    19548 &   1.3 & 140\\
         690 &    19505 &   1.3 & 140\\
         710 &    20031 &   1.4 & 142\\
         730 &    20429 &   1.5 & 143\\
         750 &    21666 &   1.6 & 147\\
    \bottomrule
    \end{tabular}
\end{table}

Die Messdaten der Detektorspannung $U$ wurden gegen die Zählraten auf
