% Messwerte: Alle gemessenen Größen tabellarisch darstellen
% Auswertung: Berechnung geforderter Ergebnisse mit Schritten/Fehlerformeln/Erläuterung/Grafik (Programme)
\section{Auswertung}

\label{sec:auswertung}
Im Folgenden wird die Kennlinie des Geiger-Müller Zählrohrs bestimmt. Die Totzeit wird zunächst über die Zwei-Quellen-Methode
und im Anschluss über das Osziloskop bestimmt.

\subsection{Fehlerrechnung}
\label{sec:Fehlerrechnung}
Die Fehlerrechnung, für die Bestimmung der Messunsicherheiten, wird mit Uncertainties \cite{uncertainties} gemacht.
Für die Formel der Gauß Fehlerfortpflanzung ist gegeben durch
\begin{equation}
    \Delta f=\sqrt{\sum_{i=1}^N\left(\frac{\partial f}{\partial x_i}\right)^2 \cdot\left(\Delta x_i\right)^2}.
    \label{eqn:gauss}
\end{equation}
Für den Mittelwert gilt 
\begin{equation}
    \bar{x} = \frac{1}{N}\sum\limits_{i = 1}^N x_i .
    \label{eqn:mittelwert}
\end{equation}
Der Fehler des Mittelwertes ist gegeben durch 
\begin{equation}
    \Delta \bar{x}=\frac{1}{\sqrt{N}} \sqrt{\frac{1}{N-1} \sum_{i=1}^N\left(x_i-\bar{x}\right)^2}.
    \label{eqn:mittelwertfehler}
\end{equation}
