% Diskussion: Resultate mit Fehler/Genauigkeit zusammenstellen, Literaturwerte/Messmethoden/Ursachen vergleichen
% Literatur: Verwendete Literatur/Grafiken/Werte/Programme
% Anhang: Kopie der analog eingetragenen Messdaten
\section{Diskussion}
\label{sec:diskussion}

Beim Vergleich der theorethischen Kennlinie des Geiger-Müller Zählrohrs (\autoref{fig:diagramm}) und der gemessenen Kennlinie
fällt auf, dass sich der Verlauf der experimentellen Kennlinie der theorethischen vom Verlauf annähert.
Die Plateau-Länge von $\SI{340}{\volt}$ und die lineare Plateau-Steigung von $ s= \qty{0.018(3)}{\per\volt}$ passen ebenfalls
zu den erwarteten Werte für das Zählrohr. Der Verlauf der detektierten Teilchen in Abhängigkeit zur Spannung 
weist einen ähnlichen Verlauf auf wie die gemessene Kennlinie des Geiger-Müller Zählrohrs.
Bei der Spannungsquelle des Geiger-Müller Zählrohrs ist in der Anleitung angegeben, dass diese eine halbe Stunde
vor Versuchs Anfang angestellt werden sollte. Bei der Durchführung des Versuches wurde dieser Zeitraum um ungefähr 15 Minuten 
gekürzt. Dies kann die Abweichungen erklären.

Für die Bestimmung der detektierten Teilchen wurde für den Strom $I$ ein Fehler von $0.05 \si{\micro\second}$ angenommen.
Bei der berechnung der detektierten Ladungsträger war dieser Fehler irrelevant , da er in einer zu kleinen Größenordnung ist.

Die Totzeit $\tau$ wurde über zwei Verfahren bestimmt. Beim Ablesen am Osziloskop wurde für $\tau$ ein Wert von
$\left(150 \pm  10\right) \si{\micro\second}$ bestimmt. Über die Zwei-Quellen-Methode wurde ein Wert von $\tau =\left(2.03 \pm 0.07\right) \si{\micro\second}$
berechnet. Die Werte befinden sich nicht in einer Größenordnung und weichen daher zu stark von einander ab, als das dies annehmbar ist.
Diese große Abweichung lässt sich durch einen Fehler des Zählrohrs erklären. Während der Messung der Zählraten der Proben
hat die Messung bei 67233 staatt 99999 abgebrochen. Die genau Ursache dieses Fehlers konnte nicht bestimmt werden.