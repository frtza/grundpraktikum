% Vorbereitung: Vorbereitungsaufgaben bearbeiten
% Versuchsaufbau: Verwendete Apparatur, Beschreibung Funktionsweise/Nutzen mit Skizze/Foto
\section{Durchführung}
\label{sec:durchführung}

Für den Versuch fungiert nun eine Spektrallampe mit Quecksilberdampf als Lichtquelle, alle übrigen Streulichteinflüsse werden abgeschirmt.
Zur Untersuchung der verschiedenen auftretenden Lichtfarben wird die Photozelle entsprechend der jeweiligen Spektrallinie ausgerichtet
und die Bremsspannung monoton so eingeregelt, dass der Photostrom verschwindet. In Abhängigkeit der Spannung am Voltmeter wird bei jedem
Schritt der Strom am Picoamperemeter abgelesen und notiert. Gegebenenfalls verbessert das Anlegen einer kleinen Beschleunigungsspannung
die Messung bei geringen Frequenzen. Weiter wird anhand einer Messung für gelbes Licht bei $\lambda = \qty{578}{\nano\meter}$ mit
beschleunigenden und verzögernden Spannungen zwischen $\qty{-20}{\volt}$ und $\qty{20}{\volt}$ sowie einer konstanten Intensität
die Funktion der Photozelle betrachtet.

