% Diskussion: Resultate mit Fehler/Genauigkeit zusammenstellen, Literaturwerte/Messmethoden/Ursachen vergleichen
% Literatur: Verwendete Literatur/Grafiken/Werte/Programme
% Anhang: Kopie der analog eingetragenen Messdaten
\section{Diskussion}
\label{sec:diskussion}

Bei der Filterkurve des Selektivverstärkers ist zu erkennen, dass der Verlauf der Messwerte sich an einer glockenförmigen Verteilung
anpasst. Dennoch wurde nicht wie eigentlich gewollt eine Güte von $Q = 100$ eingestellt, sondern $Q = 20$. 
Das hat dazu geführt, dass viel mehr Störspannungen nicht gefiltert werden konnten und daher konnte die minimale Ausgangsspannung
nicht eindeutig bestimmt werden. Als angegebene Durchlassfrequenz war $\SI{35.5}{kHz}$ angegeben. Bei der Messung wurde aber
als Durchlassfrequenz ein Wert von $\SI{21.8}{kHz}$ gemessen.

Für die experimentell bestimmte Suszeptibilität des Stoffes \ce{Dy2O3} ergab sich ein Wert von $\chi_\text{exp} = 4.288$. 
Bei der Berechnung des theoretischen Wertes der Suszeptibilität ergab sich der Wert $\chi_\text{theo} = 0.0248$. Bei den Werten
liegt eine Abweichung weit über $100\%$ vor.
Die experimentelle Suszeptibilität des Stoffes \ce{Gd2O3} beträgt $\chi_\text{exp} = 4.613$. Der theoretische Wert liegt bei
$\chi_\text{theo} = 0.0135$ , demnach liegt auch hier eine zu große Abweichung vor, als dass der experimentelle Wert noch annehmbar ist.
$\chi_\text{exp} = 0.748$ ist der experimentelle Wert der Suszeptibilität des Stoffes \ce{Nd2O3}. Der theoretische Wert lag bei
$\chi_\text{theo} = 0.03$. Die Werte haben eine ähnlich große Abweichung wie bei den anderen Stoffen.

Für die Ursache der großen Abweichungen lassen sich nur Vermutungen anstellen. Bei der Untersuchung der Filterkurve treten bereits
unerwartete Phänomene auf. So ergibt sich für die maximale Ausgangsspannung ein Wert von \qty{4.4}{\volt}, obwohl durch mehrfache
Messung sichergestellt wird, dass die Eingangsspannung bei \qty{1}{\volt} liegt. Es scheint also trotz abweichender Einstellung auf
$A_V = 1$ eine Verstärkung um den Faktor $A_V = \num{4.4}$ aufzutreten. Ebenso deutet die vom Sollwert stark abweichende
Durchlassfrequenz auf eine möglicherweise unzuverlässige Messapparatur hin. Da die experimentellen Suszeptibilitäten aber
durch die Widerstandsänderung an der Brückenschaltung ermittelt werden, scheint der relevante Fehlereinfluss des Selektivverstärkers
auf die Endergebnisse eher gering auszufallen. Eine weitere Schwierigkeit zeigt sich am Ausschlag der
abgegriffenen Spannung am Voltmeter nach Einführen des Proberöhrchens. Dieser fällt teilweise so gering aus, dass das erneute Abgleichen
der Brückenspannung kaum möglich ist und potentiell in Messfehlern der passenden Größenordnung resultiert. Zuletzt ist nicht
auszuschließen, dass innerhalb der Messspule oder den Proben selbst Unreinheiten oder Abschirmeffekte die gemessenen Werte verfälschen.
Eine Abweichung des Theoriemodells um mehrere Größenordnungen ist nicht plausibel und kann mit hoher Sicherheit ausgeschlossen werden.

