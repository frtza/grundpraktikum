% Diskussion: Resultate mit Fehler/Genauigkeit zusammenstellen, Literaturwerte/Messmethoden/Ursachen vergleichen
% Literatur: Verwendete Literatur/Grafiken/Werte/Programme
% Anhang: Kopie der analog eingetragenen Messdaten
\section{Diskussion}
\label{sec:diskussion}

Bei der Filterkurve des Selektivverstärkers ist zu erkennen, dass der Verlauf der Messwerte sich an einer Gaußverteilung
anpasst. Dennoch wurde nicht wie eigentlich gewollt eine Güte von $Q = 100$ eingestellt, sondern $Q = 20$. 
Das hat dazu geführt, dass viel mehr Störspannungen nicht gefiltert werden konnten und daher konnte die monimale Ausgangsspannung
nicht eindeutig bestimmt werden. Als angegebene Durchlassfrequenz war $\SI{35.5}{kHz}$ angegeben. Bei der Messung wurde aber
als Durchlassfrequenz ein Wert von $\SI{21.8}{kHz}$ gemessen.

Für die experimentell bestimmte Suszeptibilität des Stoffes \ce{Dy2O3} ergab sich ein Wert von $\chi_{exp} = 4.288$. 
Bei der Berechnung des theoretischen Wertes der Suszeptibilität ergab sich der Wert $\chi_{theo} = 0.0248$. Bei den Werten
liegt eine Abweichung weit über $100\%$ vor.
Die experimentelle Suszeptibilität des Stoffes \ce{Gd2O3} beträgt $\chi_{exp} = 4.613$. Der theorethische Wert liegt bei
$\chi_{theo} = 0.0135$ , demnach liegt auch hier eine zu große Abweichung vor als , dass der experimentelle Wert noch annehmbar ist.
$\chi_{exp} = 0.748$ ist der experimentelle Wert der Suszeptibilität des Stoffes \ce{Nd2O3}. Der theorethische Wert lag bei
$\chi_{theo} = 0.03$. Die Werte haben eine ähnlich große Abweichung wie bei den anderen Stoffen.

Die Großen Abweichungen lassen sich durch die vorliegende Verstärkung der Ausgangspannung erklären.
Diese wurde eigentlich auf $\SI{1}{\volt}$ eingestellt, bei der Messung der Filterkurve wurde jedoch eine Spannung von $\SI{4.4}{\volt}$ gemessen.