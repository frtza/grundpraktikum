% Messwerte: Alle gemessenen Größen tabellarisch darstellen
% Auswertung: Berechnung geforderter Ergebnisse mit Schritten/Fehlerformeln/Erläuterung/Grafik (Programme)
\section{Auswertung}
\label{sec:auswertung}
 Im Folgenden werden die aufgenommenen Messwerte ausgewertet. Zunächst werden für den Selektivverstärker 
 die Werte in einem Diagramm dargestellt. Aus den Messergebnissen der einzelen Stoffe werden die 
 experimentellen Suszeptibilitäten bestimmt.

\subsection{Fehlerrechnung}
\label{sec:Fehlerrechnung}
Für die Fehlerrechnung in diesem Versuch werden die folgenden Gleichungen verwendet, dabei wird mit den Programmen
SciPy \cite{scipy}, NumPy \cite{numpy} und Matplotlib \cite{matplotlib} gearbeitet.
%\subsection{Formeln für den Mittelwert, den Gauß-Fehler und die Standardabweichung }
%\label{sec:Formeln fuer den Mittelwert, den Gauss-Fehler und die Standardabweichung}
    Für den Mittelwert gilt
    \begin{equation}
    \bar{x} = \frac{1}{N}\sum\limits_{i = 1}^N x_i .
    \end{equation}

    Für den Fehler des Mittelwertes gilt
    \begin{equation}
        \Delta \bar{x}=\frac{1}{\sqrt{N}} \sqrt{\frac{1}{N-1} \sum_{i=1}^N\left(x_i-\bar{x}\right)^2}.
        \label{eqn:mittelwert}
        \end{equation}

    Für die Gaußsche Fehlerfortpflanzung gilt
    \begin{equation}
        \Delta f=\sqrt{\sum_{i=1}^N\left(\frac{\partial f}{\partial x_i}\right)^2 \cdot\left(\Delta x_i\right)^2}.
    \end{equation}

    Diese Formeln werden für sämtliche Fehlerrechnungen in diesem Versuch verwendet, ohne sie für die 
    jeweiligen Rechnungen explizit anzugeben. Die Rechnungen selbst werden dabei mithilfe von
    Uncertainties \cite{uncertainties} durchgeführt.

\subsection{Durchlasskurve}
\label{sec:Durchlasskurve}

Zunächst wird die Filterkurve eines Selektivverstärkers untersucht, wobei eine effektive Spannung $U_E$ in Höhe von $ \SI{1}{\volt}$ 
Verwendung findet. Aufgenommen wird dabei die Ausgangspannung $U_A$ in Anhängigkeit von der Frequenz $\nu$. Die Frequenz wurde von
$\SI{2}{\kHz}$ auf $\SI{31}{\kHz}$ hochgedreht. In Tabelle \ref{tab:filterkurve} sind die aufgenommen Messwerte aufgetragen.

\begin{table}[H]
    \centering
    \caption{Messwerte für die Filterkurve.}
    \label{tab:filterkurve}
\begin{tabular}{c c}
    \toprule
          $v \, /\,\si{kHz}$ & $U \,/\,\si{v}$ \\
    \midrule
           2 &    0.01 \\
           4 &    0.02 \\
           6 &   0.025 \\
           8 &    0.03 \\
          10 &    0.04 \\
          11 &    0.05 \\
          12 &    0.06 \\
          13 &    0.07 \\
          14 &    0.08 \\
          15 &   0.095 \\
          16 &   0.115 \\
          17 &   0.145 \\
          18 &    0.19 \\
          19 &    0.26 \\
          20 &    0.42 \\
        20.5 &     0.6 \\
          21 &    0.98 \\
        21.1 &     1.2 \\
        21.2 &     1.3 \\
        21.3 &     1.5 \\
        21.4 &    1.85 \\
        21.5 &    2.35 \\
        21.6 &     3.1 \\
        21.7 &       4 \\
        21.8 &     4.4 \\
        21.9 &     3.8 \\
          22 &    2.95 \\
        22.1 &    2.45 \\
        22.2 &    1.85 \\
        22.3 &    1.55 \\
        22.4 &     1.3 \\
        22.5 &    1.15 \\
          23 &    0.68 \\
          25 &   0.265 \\
          27 &    0.17 \\
          29 &   0.125 \\
          31 &    0.1 \\
    \bottomrule
    \end{tabular}
\end{table}

In Abbildung \ref{fig:plot} wird die Durchlasskurve der aufgenommenen Messwerte abgebildet.
Dabei ist der Quotient $\frac{U_A}{U_E}$ gegen die Frequenz $\nu$ aufgetragen. Anhand des Graphen lässt sich ablesen, dass das 
Maximum bei $\SI{21.8}{kHz}$ liegt mit einer Spannung von $\SI{4.4}{\volt}$. Dieses Maximum ist dann die Durchlassfrequenz.

\begin{figure}[H]
\includegraphics{build/plot1.pdf}
	\caption{Filterkurve des Selektivverstärkers mit einer Güte $Q = 20$ und die Ausgleichsrechnung in Form einer
			 symmetrischen Glockenkurve.}
	\label{fig:plot}
\end{figure}

\subsection{Effektiver Querschnitt}
\label{sec:Effektiver Querschnitt}

Der effektive Querschnitt wird im Folgenden von vier unterschiedlichen Stoffen bestimmt.
Für die Berechnung des realen Querschnitts gilt 
\begin{equation}
  Q_{real}=\frac{m}{l \cdot \rho_w}.
\end{equation}

In Tabelle \ref{tab:quer} stehen die gemessenen Werte für die Stoffe, sowie die berechneten effektiven Querschnitte.

\begin{table}
  \centering
  \caption{Maße der Stoffe und der daraus berechnete effektive Querschnitt.}
  \label{tab:quer}
\begin{tabular}{c c c c c}
  \toprule
  Stoff &  $m / \si{g}$ &  $l / \si{cm}$ &  $\rho / \unit{\gram\per\cubic\centi\meter}$ & $Q / \si{cm}^2$ \\
  \midrule
  \ce{Dy2O3} & 14,38 &  16,3 &           7,80 & 0,113104 \\
  \ce{Gd2O3} & 14,08 &  17,3 &           7,40 & 0,109983 \\
  \ce{Nd2O3} & 18,48 &  14,5 &           7,24 & 0,176034 \\
  \bottomrule
  \end{tabular}
\end{table}


\subsection{Suszeptibilität}
Im Folgenden wird die Suszeptibilität $\chi$ unterschiedlicher Stoffe untersucht.

In der Tabelle \eqref{tab:dy} sind die Messwerte der Probe \ce{Dy2O3} angegeben. Es sind die 
Werte ohne die Probe für die Spannung und den Widerstand angegeben und für den Fall, dass die Probe
verwendet wurde. Daraus wurde die Differenz der Widerstände $\Delta R\mathbin{/}\si{\ohm}$ berechnet.

\begin{table}
  \centering
  \caption{Messwerte der Probe \ce{Dy2O3} sowie die Differenz $\Delta R$.}
  \label{tab:dy}
\begin{tabular}{c c | c c | c}
  \hline
  \multicolumn{2}{c}{ohne Probe} & \multicolumn{3}{c}{mit Probe} \\
  \hline
  $U\mathbin{/} \si{\mV}$ & $R_{3/4}\mathbin{/} \si{\ohm}$ & $U\mathbin{/} \si{\mV}$ & $R_{3/4}\mathbin{/} \si{\ohm}$ & $\Delta R\mathbin{/}\si{\ohm}$ \\
  \hline
  15.5  & 2.62 & 15    & 3.265 & 0.645\\
  16  & 2.565 & 15.5   & 3.245 & 0.68\\
  16.5 & 2.445 & 15   & 3.245 & 0.8\\
  16  & 2.525 & 14.3   & 3.245 & 0.72\\
  \bottomrule
  \end{tabular}
\end{table}

Aus den brechneten Werten der Differenz wird anschließend der Mittelwert \eqref{eqn:mittelwert} bestimmt
\begin{equation*}
  \bar{\Delta R} \approx 0.711 \si{\ohm}.
\end{equation*}
Ebenfalls wird der Mittelwert für den Abgleichwiderstand $R_{3/4}$ ohne Probe gebildet
\begin{equation*}
  \bar{R_{3/4}} \approx 2.539 \si{\ohm}.
\end{equation*}

Die Tabelle \ref{tab:gd} beinhaltet die Messwerte für den Stoff \ce{Gd2O3}. Die Werte für die Spannung und die Widerstände
vor dem Einführen und nach dem Einführen sind angegeben. Daraus wurde dann die Differenz der Widerstände berechnet. 
\begin{table}
  \centering
  \caption{Messwerte der Probe \ce{Gd2O3} sowie die Differenz $\Delta R$.}
  \label{tab:gd}
\begin{tabular}{c c | c c | c}
  \hline
  \multicolumn{2}{c}{ohne Probe} & \multicolumn{3}{c}{mit Probe} \\
  \hline
  $U\mathbin{/} \si{\mV}$ & $R_{3/4}\mathbin{/} \si{\ohm}$ & $U\mathbin{/} \si{\mV}$ & $R_{3/4}\mathbin{/} \si{\ohm}$ & $\Delta R\mathbin{/}\si{\ohm}$ \\
  \hline
  15.4 & 2.45 & 15.4  & 1.765 & 0.685\\
  16  & 2.48 & 15  & 1.73 & 0.75\\
  16 & 2.525& 15.4  & 1.775 & 0.75\\
  \bottomrule
  \end{tabular}
\end{table}

Für den Mittelwert der Widerstandsdifferenzen für den Stoff \ce{Gd2O3} ergibt sich
\begin{equation*}
  \bar{\Delta R} \approx 0.728 \si{\ohm}.
\end{equation*}
Ebenfalls wird der Mittelwert für den Abgleichwiderstand $R_{3/4}$ ohne Probe gebildet
\begin{equation*}
  \bar{R_{3/4}} \approx 2.485\si{\ohm}.
\end{equation*}

Für den dritten Stoff wurde die gleiche Messung durchgeführt. Die Ergebnisse werden in Tabelle \ref{tab:nd}
abgebildet. Die Differenz wurde ebenfalls berechnet und in der Tabelle \ref{tab:nd} abgebildet.
\begin{table}
  \centering
  \caption{Messwerte der Probe \ce{Nd2O3} sowie die Differenz $\Delta R$.}
  \label{tab:nd}
\begin{tabular}{c c | c c | c}
  \hline
  \multicolumn{2}{c}{ohne Probe} & \multicolumn{3}{c}{mit Probe} \\
  \hline
  $U\mathbin{/} \si{\mV}$ & $R_{3/4}\mathbin{/} \si{\ohm}$ & $U\mathbin{/} \si{\mV}$ & $R_{3/4}\mathbin{/} \si{\ohm}$ & $\Delta R\mathbin{/}\si{\ohm}$ \\
  \hline
  15.5 & 2.495 & 16.5  & 2.365 & 0.13\\
  17  & 2.535 & 16.5  & 2.265 & 0.27\\
  16 & 2.47& 16 & 2.30 & 0.17\\
  \bottomrule
  \end{tabular}
\end{table}

Für den Mittelwert der Widerstandsdifferenzen für den Stoff \ce{Nd2O3} ergibt sich
\begin{equation*}
  \bar{\Delta R} \approx 0.19 \si{\ohm}.
\end{equation*}
Ebenfalls wird der Mittelwert für den Abgleichwiderstand $R_{3/4}$ ohne Probe gebildet
\begin{equation*}
  \bar{R_{3/4}} \approx 2.5\si{\ohm}.
\end{equation*}

Mit Hilfe der Formel \eqref{eqn:chi_exp}
lässt sich für die einzelnen Stoffe der experimentelle Wert der Suszeptibilität bestimmen.
Die berechneten Werte der einzelen Stoffe stehen in der Tabelle \ref{tab:sus}.
\begin{table}
  \centering
  \caption{Die experimentell bestimmten Suszeptibilitäten.}
  \label{tab:sus}
\begin{tabular}{c|c}
  \toprule
  Stoff & $\chi$ \\
  \midrule
  \ce{Dy2O3} & 0.0109\\
  \ce{Gd2O3} & 0.0019\\
  \ce{Nd2O3} & 0.0114\\
  \bottomrule
 \end{tabular}
\end{table}

Zur Bestimmung der theoretischen Suszeptibilität müssen die Hundschen Regeln auf die Ionen der untersuchten Substanzen angewendet
werden: In der 4f-Hülle von \ce{Dy^{3+}} befinden sich neun Elektronen. Das Pauli-Prinzip gestattet, dass sieben der neun Spins
nach Vorschrift der 1. Hundschen Regel parallel den Wert $\symbf s = \num{0.5}$ annehmen, während für die übrigen zwei Elektronen
der Spin mit $\symbf s = \num{-0.5}$ antiparallel ausgerichtet ist. Dies ist eine Konsequenz der begrenzten Orientierungszahl
$l \leq 3$ für 4f-Elektronen, welche nur Zustände $\symbf l \in \{-3,-2,-1,0,+1,+2,+3\}$ zulässt. Es ergibt sich
$S = 7 (\num{0.5}) - 2 (\num{0.5}) = \num{2.5}$ für den Gesamtspin. Da die sieben Elektronen mit parallelen Spins alle Werte
für $\symbf l$ ausschöpfen, verschwindet ihr Anteil am Bahndrehimpuls. Nach der 2. Hundschen Regel nehmen die Elektronen mit
negativem Spin dann die maximal möglichen Werte an, sodass $L = 3 + 2 = 5$ ist. Da mit $9 > 7$ mehr als die Hälfte der 4f-Schale
besetzt ist, lässt sich laut 3. Hundscher Regel der Gesamtdrehimpuls zu $J = L + S = 5 + \num{2.5} = \num{7.5}$ berechnen. Analog
wird mit den weiteren Stoffen verfahren. \ce{Gd^{3+}} besitzt sieben 4f-Elektronen und füllt die Schale damit zur Hälfte. Alle
Spins sind parallel zu $S = 7(\num{0.5}) = \num{3.5}$ gerichtet, während für den Gesamtbahndrehimpuls $L = 0$ und daher
$J = S = \num{3.5}$ gilt. Zuletzt weist \ce{Nd^{3+}} drei Elektronen in der 4f-Schale auf, sodass der maximale Gesamtspin
$S = 3(\num{0.5}) = \num{1.5}$ mit dem Bahndrehimpuls $L = 3 + 2 + 1 = 6$ auftritt. Wegen $3 < 7$ ist die maximale Anzahl 4f-Elektronen
weniger als zur Hälfte ausgereizt, demnach kann $J = L - S = 6 - \num{1.5} = \num{4.5}$ gefolgert werden. Die Gültigkeit dieser
Quantenzahlen wird auch für die Oxidverbindungen angenommen und in Tabelle \ref{tab:zahlen} mit dem entsprechenden Landé-Faktor
$g_J$ nach \eqref{eqn:lande} zusammengefasst. Für die Suszeptibilität $\chi$ wird Formel \eqref{eqn:chi_theo} herangezogen, wobei
\begin{equation*}
	N = 2 \frac{N_{\! A} \, \rho}{M_\text{mol}}
\end{equation*}
mit Avogadro-Konstante $N_{\! A} = \qty{6.02e23}{\per\mole}$ und Molmasse $M_\text{mol}$ gilt. Der Vorfaktor begründet sich dadurch,
dass alle Moleküle je zwei der Seltenen-Erd-Ionen beinhalten. Unter Verwendung der Naturkonstanten
$e_0 = \qty{1.60e-19}{\coulomb}$, $m_0 = \qty{9.11e-31}{\kilo\gram}$ und $\hbar = \qty{1.05e-34}{\joule\second}$ ergibt sich nach
\eqref{eqn:magneton} weiter $\mu_B = \qty{9.27e-24}{\joule\per\tesla}$ für das Bohrsche Magneton. Mit der magnetischen Feldkonstante
$\mu_0 = \qty{1.26e-6}{\newton\per\ampere\squared}$ und der Boltzmann-Konstante $\qty{1.38e-23}{\joule\per\kelvin}$ sind nun
alle Parameter zur Rechnung bekannt.

\begin{table}
	\centering
	\caption{Ergebnisse für eine Temperatur $T = \qty{300}{\kelvin}$.}
	\label{tab:zahlen}
	\begin{tabular}{c |
			S[table-format=1.0]
			S[table-format=1.1]
			S[table-format=1.0]
			S[table-format=1.1]
			S[table-format=1.2]
			S[table-format=3.2]
			S[table-format=1.2]
			S[table-format=1.4]}
		\toprule
		Verbindung & \ce{e-} & $S$ & $L$ & $J$ & $g_J$ &
		{$M_\text{mol} \mathbin{/} \unit{\gram\per\mole}$} &
		{$N \mathbin{/} \qty{e28}{\per\cubic\meter}$} & $\chi$ \\
		\midrule
		\ce{Dy2O3} & 9 & 2.5 & 5 & 7.5 & 1.33 & 373.00 & 2.52 & 0.0248 \\
		\ce{Gd2O3} & 7 & 3.5 & 0 & 3.5 & 2.00 & 362.49 & 2.46 & 0.0135 \\
		\ce{Nd2O3} & 3 & 1.5 & 6 & 4.5 & 0.73 & 336.48 & 2.59 & 0.0030 \\
		\bottomrule
	\end{tabular}
\end{table}
