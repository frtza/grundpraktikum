% Messwerte: Alle gemessenen Größen tabellarisch darstellen
% Auswertung: Berechnung geforderter Ergebnisse mit Schritten/Fehlerformeln/Erläuterung/Grafik (Programme)
\section{Auswertung}
\label{sec:auswertung}


\subsection{Fehlerrechnung}
\label{sec:Fehlerrechnung}
Die Fehlerrechnung für die Bestimmung der Messunsicherheiten, wird mit Uncertainties \cite{uncertainties} gemacht.
Die Formel der Gauß Fehlerfortpflanzung ist gegeben durch
\begin{equation}
    \Delta f=\sqrt{\sum_{i=1}^N\left(\frac{\partial f}{\partial x_i}\right)^2 \cdot\left(\Delta x_i\right)^2}.
    \label{eqn:gauss}
\end{equation}
Für den Mittelwert bei $N$ Messwerten gilt 
\begin{equation}
    \bar{x} = \frac{1}{N}\sum\limits_{i = 1}^N x_i .
    \label{eqn:mittelwert}
\end{equation}
Der Fehler des Mittelwertes lässt sich berechnen mit
\begin{equation}
    \Delta \bar{x}=\frac{1}{\sqrt{N}} \sqrt{\frac{1}{N-1} \sum_{i=1}^N\left(x_i-\bar{x}\right)^2}.
    \label{eqn:mittelwertfehler}
\end{equation}
Zu Bestimmung der relativen Abweichung von den experimentellen Werten zu den theoretischen Werten wird die
Relation 
\begin{equation}
    \increment x = \frac{x_{exp}-x_{theo}}{x_{theo}}
\end{equation} 
genutzt.

Zunächst wird überprüft, ob die einzelenen Messwerte die Bedingung $2v_0 = v_{ab}-v_{auf}$ erfüllen.
Wenn die Messwerte die Relation nicht erfüllen und außerhalb des Rahmens der Messgenauigkeit sind, können 
diese verworfen werden. Verwendet werden die Werte, bei denen die prozentuale Abweichung vom Sollwert unter
$50 \%$ liegt. Die verwendbaren Messwerte werden in \autoref{} dargestellt.

\begin{table}[H]
	\centering
	\captionsetup{width=0.95\linewidth}
	\caption{Ergebnisse für 1. Tröpfchen. Felder entsprechen jeweils primären, sekundären und tertiären Messgrößen. Hervorgehoben
			 ist der Mittelwert der Laufzeiten. Relative Abweichung zur Bedingung an Geschwindigkeit: $\protect\input{build/bed_1.tex}$.}
	\input{build/table_1.tex}
	\label{tab:tropfen_1}
\end{table}

\begin{table}[H]
	\centering
	\captionsetup{width=0.95\linewidth}
	\caption{Ergebnisse für 3. Tröpfchen. Felder entsprechen jeweils primären, sekundären und tertiären Messgrößen. Hervorgehoben
			 ist der Mittelwert der Laufzeiten. Relative Abweichung zur Bedingung an Geschwindigkeit: $\protect\input{build/bed_3.tex}$.}
	\input{build/table_3.tex}
	\label{tab:tropfen_3}
\end{table}

\begin{table}[H]
	\centering
	\captionsetup{width=0.95\linewidth}
	\caption{Ergebnisse für 4. Tröpfchen. Felder entsprechen jeweils primären, sekundären und tertiären Messgrößen. Hervorgehoben
			 ist der Mittelwert der Laufzeiten. Relative Abweichung zur Bedingung an Geschwindigkeit: $\protect\input{build/bed_4.tex}$.}
	\input{build/table_4.tex}
	\label{tab:tropfen_4}
\end{table}

\begin{table}[H]
	\centering
	\captionsetup{width=0.95\linewidth}
	\caption{Ergebnisse für 5. Tröpfchen. Felder entsprechen jeweils primären, sekundären und tertiären Messgrößen. Hervorgehoben
			 ist der Mittelwert der Laufzeiten. Relative Abweichung zur Bedingung an Geschwindigkeit: $\protect\input{build/bed_5.tex}$.}
	\input{build/table_5.tex}
	\label{tab:tropfen_5}
\end{table}

\begin{table}[H]
	\centering
	\captionsetup{width=0.95\linewidth}
	\caption{Ergebnisse für 6. Tröpfchen. Felder entsprechen jeweils primären, sekundären und tertiären Messgrößen. Hervorgehoben
			 ist der Mittelwert der Laufzeiten. Relative Abweichung zur Bedingung an Geschwindigkeit: $\protect\input{build/bed_6.tex}$.}
	\input{build/table_6.tex}
	\label{tab:tropfen_6}
\end{table}

\begin{table}[H]
	\centering
	\captionsetup{width=0.95\linewidth}
	\caption{Ergebnisse für 7. Tröpfchen. Felder entsprechen jeweils primären, sekundären und tertiären Messgrößen. Hervorgehoben
			 ist der Mittelwert der Laufzeiten. Relative Abweichung zur Bedingung an Geschwindigkeit: $\protect\input{build/bed_7.tex}$.}
	\input{build/table_7.tex}
	\label{tab:tropfen_7}
\end{table}

\begin{table}[H]
	\centering
	\captionsetup{width=0.95\linewidth}
	\caption{Ergebnisse für 8. Tröpfchen. Felder entsprechen jeweils primären, sekundären und tertiären Messgrößen. Hervorgehoben
			 ist der Mittelwert der Laufzeiten. Relative Abweichung zur Bedingung an Geschwindigkeit: $\protect\input{build/bed_8.tex}$.}
	\input{build/table_8.tex}
	\label{tab:tropfen_8}
\end{table}

\begin{table}[H]
	\centering
	\captionsetup{width=0.95\linewidth}
	\caption{Ergebnisse für 10. Tröpfchen. Felder entsprechen jeweils primären, sekundären und tertiären Messgrößen. Hervorgehoben
			 ist der Mittelwert der Laufzeiten. Relative Abweichung zur Bedingung an Geschwindigkeit: $\protect\input{build/bed_10.tex}$.}
	\input{build/table_10.tex}
	\label{tab:tropfen_10}
\end{table}

\begin{table}[H]
	\centering
	\captionsetup{width=0.95\linewidth}
	\caption{Ergebnisse für 12. Tröpfchen. Felder entsprechen jeweils primären, sekundären und tertiären Messgrößen. Hervorgehoben
			 ist der Mittelwert der Laufzeiten. Relative Abweichung zur Bedingung an Geschwindigkeit: $\protect\input{build/bed_12.tex}$.}
	\input{build/table_12.tex}
	\label{tab:tropfen_12}
\end{table}

\begin{table}[H]
	\centering
	\captionsetup{width=0.95\linewidth}
	\caption{Ergebnisse für 13. Tröpfchen. Felder entsprechen jeweils primären, sekundären und tertiären Messgrößen. Hervorgehoben
			 ist der Mittelwert der Laufzeiten. Relative Abweichung zur Bedingung an Geschwindigkeit: $\protect\input{build/bed_13.tex}$.}
	\input{build/table_13.tex}
	\label{tab:tropfen_13}
\end{table}

\begin{table}[H]
	\centering
	\captionsetup{width=0.95\linewidth}
	\caption{Ergebnisse für 14. Tröpfchen. Felder entsprechen jeweils primären, sekundären und tertiären Messgrößen. Hervorgehoben
			 ist der Mittelwert der Laufzeiten. Relative Abweichung zur Bedingung an Geschwindigkeit: $\protect\input{build/bed_14.tex}$.}
	\input{build/table_14.tex}
	\label{tab:tropfen_14}
\end{table}

\begin{table}[H]
	\centering
	\captionsetup{width=0.95\linewidth}
	\caption{Ergebnisse für 16. Tröpfchen. Felder entsprechen jeweils primären, sekundären und tertiären Messgrößen. Hervorgehoben
			 ist der Mittelwert der Laufzeiten. Relative Abweichung zur Bedingung an Geschwindigkeit: $\protect\input{build/bed_16.tex}$.}
	\input{build/table_16.tex}
	\label{tab:tropfen_16}
\end{table}

\begin{table}[H]
	\centering
	\captionsetup{width=0.95\linewidth}
	\caption{Ergebnisse für 17. Tröpfchen. Felder entsprechen jeweils primären, sekundären und tertiären Messgrößen. Hervorgehoben
			 ist der Mittelwert der Laufzeiten. Relative Abweichung zur Bedingung an Geschwindigkeit: $\protect\input{build/bed_17.tex}$.}
	\input{build/table_17.tex}
	\label{tab:tropfen_17}
\end{table}
