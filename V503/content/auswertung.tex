% Messwerte: Alle gemessenen Größen tabellarisch darstellen
% Auswertung: Berechnung geforderter Ergebnisse mit Schritten/Fehlerformeln/Erläuterung/Grafik (Programme)
\section{Auswertung}
\label{sec:auswertung}


\subsection{Fehlerrechnung}
\label{sec:Fehlerrechnung}
Die Fehlerrechnung für die Bestimmung der Messunsicherheiten, wird mit Uncertainties \cite{uncertainties} gemacht.
Die Formel der Gauß Fehlerfortpflanzung ist gegeben durch
\begin{equation}
    \Delta f=\sqrt{\sum_{i=1}^N\left(\frac{\partial f}{\partial x_i}\right)^2 \cdot\left(\Delta x_i\right)^2}.
    \label{eqn:gauss}
\end{equation}
Für den Mittelwert bei $N$ Messwerten gilt 
\begin{equation}
    \bar{x} = \frac{1}{N}\sum\limits_{i = 1}^N x_i .
    \label{eqn:mittelwert}
\end{equation}
Der Fehler des Mittelwertes lässt sich berechnen mit
\begin{equation}
    \Delta \bar{x}=\frac{1}{\sqrt{N}} \sqrt{\frac{1}{N-1} \sum_{i=1}^N\left(x_i-\bar{x}\right)^2}.
    \label{eqn:mittelwertfehler}
\end{equation}
Zu Bestimmung der relativen Abweichung von den experimentellen Werten zu den theoretischen Werten wird die
Relation 
\begin{equation}
    \increment x = \frac{x_{exp}-x_{theo}}{x_{theo}}
\end{equation} 
genutzt.

Zunächst wird überprüft, ob die einzelenen Messwerte die Bedingung $2v_0 = v_{ab}-v_{auf}$ erfüllen.
Wenn die Messwerte die Relation nicht erfüllen und außerhalb des Rahmens der Messgenauigkeit sind, können 
diese verworfen werden. Verwendet werden die Werte, bei denen die prozentuale Abweichung vom Sollwert unter
$50 \%$ liegt.
