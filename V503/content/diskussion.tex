% Diskussion: Resultate mit Fehler/Genauigkeit zusammenstellen, Literaturwerte/Messmethoden/Ursachen vergleichen
% Literatur: Verwendete Literatur/Grafiken/Werte/Programme
% Anhang: Kopie der analog eingetragenen Messdaten
\newpage
\section{Diskussion}
\label{sec:diskussion}


% Fehlerquellen die mir einfallen...


Die experimentell bestimmte Ladung entspricht $e_{\text{exp}} = (1.6 \pm 1.0) \cdot 10^{-19} \, \mathrm{C}$, der
theoretische Wert enspricht $e_{\text{theo}} = 1.602176634 \cdot 10^{-19} \, \mathrm{C}$ . Der berechnete
Mittelwert stimmt mit dem Literaturwert bei einer Unsicherheit von $60 \%$ überein. 
Für die experimentell bestimmte Avogadro-Konstante gilt $N_{\! A} = (6 \pm 4) \cdot 10^{23} \mathrm{mol}^{-1}$.
Der theoretische Wert stimmt mit dem Experimentellen bei einer Unsicherheit von $70\%$ überein.
Die zum Vergleich verwendeten Konstanten werden der Datenbank \verb+scipy.constants+~\cite{scipy} entnommen.

Auffallend bei der Bestimmung der Ladung ist, dass die Fehler der einzelnen Ladungen hoch sind. Bei den Messwerten wurde
zunächst gemittelt, um dann die Rechnung fortzuführen. Dadurch, dass die Messwerte für $t_{\text{ab}}$ und $t_{\text{auf}}$ eines einzelnen Teilchens
sich teilweise stark unterscheiden, wird der Fehler des jeweiligen Mittelwertes groß. 
Der große Fehler des Mittelwertes lässt sich dadurch begründen, dass die Messmethode nicht effektiv war. Die Zeiten wurden
händisch über eine Stoppuhr gemessen, wobei ein Experimentator die Tropfen beobachtet hat und der andere
die Zeit gemessen hat. Die Strecke für die die Zeit gemessen wurde, hat im Idealfall $s = 0.5 \mathrm{mm}$ betragen, jedoch
war dies eine grobe Abschätzung, da es durch das Mikroskop sehr ungenau war diese Strecke genau für die Tropfen ablesen
zu können. Wenn die Teilchen sehr schnell waren und demnach eine hohe Geschwindigkeit hatten, wurde der Fehler des
Ablesens auch groß. 

Eine weitere Fehlerquelle war die Kondensator-Kammer, da diese nicht dicht war. Es kam während der Versuchsdurchführung mehrmals
dazu, dass die Tropfen durch Luftzüge bewegt worden waren. Demnach wurde die Bewegung der Tröpfchen beeinflusst und somit 
die Ergebnisse verfälscht. Die Libelle war ebenfalls leicht ungenau ausgerichtet und demnach war die Gewichtskraft nicht
exakt parallel zum Feld.

Die Methode zur Ladungsbestimmung beruht auf einer graphischen Abschätzung und ist somit grundsätzlich für Fehler der menschlichen Kognition
anfällig. Da der Literaturwert der Elementarladung bereits im Voraus bekannt ist, kann also nicht ausgeschlossen werden, dass die Wahl der
Ladungsstufen von einer Bestätigungstendez beeinflusst ist, obwohl das Theorieergebnis nie explizit in der Rechnung Verwendung findet. Um
dies weiter zu untersuchen muss die Argumentation der Auswahl betrachtet werden. Diese stützt sich im Wesentlichen auf die Annahme, dass
beim Zerstäuben der Öltröpfchen nur vielfache Elementarladungen im Bereich der niedrigen einstelligen Zahlen auftreten. Mit der experimentellen
Erfahrung, dass die Umgebungsluft größtenteils elektrisch neutral ist und es demnach nur wenige freie Ladungsträger gibt, scheint damit
eine plausible Näherung getroffen zu sein, die resultierenden Ergebnisse können also angenommen werden.

\newpage
