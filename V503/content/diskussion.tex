% Diskussion: Resultate mit Fehler/Genauigkeit zusammenstellen, Literaturwerte/Messmethoden/Ursachen vergleichen
% Literatur: Verwendete Literatur/Grafiken/Werte/Programme
% Anhang: Kopie der analog eingetragenen Messdaten
\section{Diskussion}
\label{sec:diskussion}


% Fehlerquellen die mir einfallen...


Die experimentell bestimmte Ladung entspricht $e_{\text{exp}} = (1.6 \pm 1.0) \cdot 10^{-19} \, \mathrm{C}$, der
theorethische Wert enspricht $e_{\text{theo}} = 1.602176634 \cdot 10^{-19} \, \mathrm{C}$ . Der berechnete
Mittelwert stimmt mit dem Literauturwert und einer Unsicherheit von $60 \%$ überein. 
Für die experimentell bestimmte Avogadro-Konstante gilt $N_{\! A} = (6 \pm 4) \cdot 10^{23} \mathrm{mol}^{-1}$.
Der theorethische Wert stimmt mit dem Experimentellen mit einer Unsicherheit von $70\%$ überein.
Auffallend bei der bestimmung der Ladung ist, dass die Fehler der einzelenen Ladungen hoch ist. Bei den Messwerten wurde
zunächst gemittel, um dann die Rechnung fortzuführen. Dadurch, dass die Messwerte für $t_{\text{ab}}$ und $t_{\text{auf}}$ eines einzelenen Teilchens
sich teilweise stark unterscheiden, wird der Fehler des jeweiligen Mittelwertes groß. 
Der große Fehler des Mittelwertes lässt sich dadurch begründen, dass die Messmethode nicht effektiv war. Die Zeiten wurden
händisch über eine Stoppuhr gemessen, wobei ein Experimentator die Tropfen beobachtet hat und der andere
die Zeit gemessen hat. Die Strecke für die die Zeit gemessen wurde, hat im Idealfall $s = 0.5 \mathrm{mm}$ betragen, jedoch
war dies eine grobe Abschätzung, da es durch das Mikroskop sehr ungenau war diese Strecke genau für die Tropfen abgelesen
zu können. Wenn die Teilchen sehr schnell waren und demnach eine hohe Geschwindigkeit hatten, wurde der Fehler des
Ablesens auch groß. 

Eine weitere Fehlerquelle war die Kondensator-Kammmer, da diese nicht dicht war. Es kam während der Versuchdurchführung mehrmals
dazu, dass die Tropfen durch Luftzüge bewegt worden waren. Demnach wurde die Bewegung der Tröpfchen beeinflusst und somit 
die Ergebnisse verfälscht. Die Luftzüge wurden bevorzugt durch vorbei gehende Menschen erzeugt, daher ist es nicht 
möglich nachzuvollziehen, welche der Messwerte betroffen sind.

Die Libelle war ungenau ausgerichtet und demnach war die Gewichtskraft nicht exakt parallel zum Feld.


