% Diskussion: Resultate mit Fehler/Genauigkeit zusammenstellen, Literaturwerte/Messmethoden/Ursachen vergleichen
% Literatur: Verwendete Literatur/Grafiken/Werte/Programme
% Anhang: Kopie der analog eingetragenen Messdaten
\section{Diskussion}
\label{sec:diskussion}

Die Kennlinien der Abbildungen \ref{fig:plot_1}, \ref{fig:plot_2} und \ref{fig:plot_3} zeigen eine hohe Übereinstimmung mit dem
erwarteten Verlauf nach Abbildung~\ref{fig:kennlinie} auf. Im Gültigkeitsbereich des Langmuir-Schottkyschen
Raumladungsgesetzes~\eqref{eqn:langmuir} ergeben sich dazu nach Ausgleichsrechnung mit steigender Heizleistung die Werte
\input{build/b_1}, \input{build/b_2} und \input{build/b_3} für den Exponenten, der laut Vorhersage $b = 3/2$ erfüllt. Diese
Resultate fallen damit zwar geringer als in der Theorie aus, lassen sich aber dennoch mit dieser vereinbaren: Es ist davon
auszugehen, dass genauere Wahl des Übergangspunktes vom Raumladungs- zum Sättigungsstromgebiet die Regressionskurve
steiler verlaufen lassen und den Exponenten so an \num{1.5} angleichen würde. Die gleiche Störquelle betrifft auch die
berechneten Sättigungsströme $I_S$. Hinzu kommt weiter, dass Ansatz~\eqref{eqn:fit_2} nicht auf konkreten physikalischen
Zusammenhängen aufbaut, sondern ausschließlich zur Bestimmung der Asymptote dient. Trotzdem wird die Erwartung eindeutig erfüllt:
Mit wachsendem Heizstrom steigt auch $I_S$ mit den Werten $\input{build/u_1.tex}$, \mbox{$\input{build/u_2.tex}$ und
$\input{build/u_3.tex}$ stetig an.}

Auch der exponentielle Zusammenhang des Anlaufstromgebietes in Abbildung \ref{fig:plot_4} folgt der Vorhersage. Die Messung des
Anodenstroms gestaltet sich für größere Gegenfeldspannungen jedoch schwierig, da die Empfindlichkeit der Messapparatur so hoch
eingestellt werden muss, dass das Signal vom Hintergrundrauschen verdeckt wird. Die daraus per nonlineare Regression errechnete
Kathodentemperatur $T = \input{build/T.tex}$ liegt in der passenden Größenordnung, steht jedoch im Konflikt zu den Werten in
Tabelle~\ref{tab:table_3}, die sich aus der Leistungsbilanz ergeben und für geringere Heizleistungen bereits höher ausfallen.
Eine mögliche Ursache dafür könnte eine unbekannte Verlustquelle darstellen, für die bei der Berechnung aus der Gesamtleistung
nicht kontrolliert wird. Die vorgenommene Korrektur um den Innenwiderstand des Nanoamperemeters fällt an dieser Stelle
geringer als die Genauigkeit der inititalen Messung am Konstantspannungsgerät aus, ihr Effekt ist demnach größtenteils
vernachlässigbar.

Der Literaturwert für polykristalline Wolfram-Aggregate wird mit \qty{4.54(0.05)}{\electronvolt}~\cite{hop_riv_work_function}
angegeben, Monokristalle weisen je nach Oberflächenstruktur Werte von \qty{4.47(0.02)}{\electronvolt} über
\qty{4.63(0.02)}{\electronvolt} bis \qty{5.25(0.02)}{\electronvolt} \cite{str_mac_swa_work_function} auf.
Die hier ermittelte Austrittsarbeit von $W = \input{build/W.tex}$ verträgt sich demnach gut mit den geläufigen
Ergebnissen verschiedener Messmethoden. Dabei findet die geringere Genauigkeit in den systematischen
Fehlerquellen ihre Begründung. Abweichungen aus Strom- und Spannungsmessung sowie aus dem Abschätzen des Wendepunkts der
Kennlinien führt bereits vor jeglicher rechnerischer Abweichung signifikante Störfaktoren in den verwendeten Datensatz ein.

\vfill
