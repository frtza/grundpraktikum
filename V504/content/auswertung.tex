% Messwerte: Alle gemessenen Größen tabellarisch darstellen
% Auswertung: Berechnung geforderter Ergebnisse mit Schritten/Fehlerformeln/Erläuterung/Grafik (Programme)
\section{Auswertung}
\label{sec:auswertung}

\subsection{Kennlinien}

\begin{table}
	\caption{}
	\centering
	\input{build/table_1.tex}
	\label{tab:table_1}
\end{table}

\begin{equation*}
    I = a \frac{4}{9} \xi_0 \sqrt{2 e_0/m_0} U^b ,
\end{equation*}

\begin{equation*}
	I = u + \frac{v}{U^w}
\end{equation*}

\begin{figure}[H]
	\includegraphics{build/plot_1.pdf}
	\caption{}
	\label{fig:plot_1}
\end{figure}

\begin{align*}
	a = \input{build/a_1.tex} && b = \input{b_1.tex}
\end{align*}
\begin{align*}
	u = \input{build/u_1.tex} && v = \input{v_1.tex} && w = \input{build/w_1.tex}
\end{align*}

\begin{figure}[H]
	\includegraphics{build/plot_2.pdf}
	\caption{}
	\label{fig:plot_2}
\end{figure}

\begin{align*}
	a = \input{build/a_2.tex} && b = \input{b_2.tex}
\end{align*}
\begin{align*}
	u = \input{build/u_2.tex} && v = \input{v_2.tex} && w = \input{build/w_2.tex}
\end{align*}

\begin{figure}[H]
	\includegraphics{build/plot_3.pdf}
	\caption{}
	\label{fig:plot_3}
\end{figure}

\begin{align*}
	a = \input{build/a_3.tex} && b = \input{b_3.tex}
\end{align*}
\begin{align*}
	u = \input{build/u_3.tex} && v = \input{v_3.tex} && w = \input{build/w_3.tex}
\end{align*}

\begin{figure}[H]
	\includegraphics{build/plot_4.pdf}
	\caption{}
	\label{fig:plot_4}
\end{figure}

\begin{equation*}
	I = p \exp(qU)
\end{equation*}

\begin{align*}
	p = \input{build/p.tex} && q = \input{q.tex}
\end{align*}
\begin{equation*}
	T = \input{build/T.tex}
\end{equation*}
