% Messwerte: Alle gemessenen Größen tabellarisch darstellen
% Auswertung: Berechnung geforderter Ergebnisse mit Schritten/Fehlerformeln/Erläuterung/Grafik (Programme)
\section{Auswertung}
\label{sec:auswertung}

\subsection{Sättigungsstrom und Gültigkeit des Raumladungsgesetzes}

Wie beschrieben wird die in Tabelle \ref{tab:table_1} nachgehaltene Kennlinienschar aufgenommen. Anhand dieser Daten
wird der Wendepunkt als Übergang vom Raumladungs- zum Sättigungsstrombereich abgeschätzt. Die so aufgeteilten Messwerte, deren
Schnittmenge nur noch diesen approximierten Wendepunkt enthält, werden weiter separat unter Python~\cite{python} ausgewertet. Zur
Untersuchung des Exponenten $b$ der Strom-Spannungs-Beziehung bietet sich eine Funktion der Form
\begin{equation}
    I = a \frac{4}{9} \xi_0 \sqrt{2 e_0/m_0} U^b ,
	\label{eqn:fit_1}
\end{equation}
nach dem Langmuir-Schottkyschen Raumladungsgesetz~\eqref{eqn:langmuir} an, wobei der hier dimensionslose Skalierungsfaktor $a$
als zusätzlicher Freiheitsgrad dient. Zur Modellierung des Sättigungsstroms dient eine Beziehung
\begin{equation}
	I = u + \frac{v}{U^w}
	\label{eqn:fit_2}
\end{equation}
mit deren Asymptote $u$, wobei $v < 0$ und $w > 0$ gilt. Die Parameter der passenden Ausgleichrechnungen lassen sich mittels
\verb+scipy.optimize.curve_fit+~\cite{scipy} bestimmen, wobei die Streuung über die Quadratwurzel der Diagonalelemente der
Kovarianzmatrix gegeben ist. Zur grafischen Darstellung von Messdaten und Regression wird Matplotlib~\cite{matplotlib} verwendet,
die Bibliothek Uncertainties~\cite{uncertainties} erlaubt eine automatisierte Fehlerfortpflanzung.

\begin{table}[H]
	\caption{Anodenstrom $I_A$ einer Hochvakuumdiode bei entsprechender Saugspannung $U_A$ unter Variation von
			 Heizstrom $I_H$ und -spannung $U_H$ an der Glühkathode. Hervorgehobene Werte stammen aus nachträglicher
			 Messung zur genaueren Untersuchung des Raumladungsstromgebiets.}
	\centering
	\input{build/table_1.tex}
	\label{tab:table_1}
\end{table}

\begin{figure}[H]
	\includegraphics{build/plot_1.pdf}
	\caption{Messwerte und Regressionskurven für $I_H = \qty{2.0}{\ampere}$ und $U_H = \qty{3.5}{\volt}$.}
	\label{fig:plot_1}
\end{figure}

Der Wendepunkt lässt sich grob bei $U_A = \qty{20}{\volt}$ einordnen. Für den Gültigkeitsbereich des Raumladungsgesetzes
ergeben sich damit
\begin{align*}
	a = \input{build/a_1.tex} && b = \input{b_1.tex}
\end{align*}
als Parameter nach \eqref{eqn:fit_1}. Weiter folgen aus \eqref{eqn:fit_2}
\begin{align*}
	u = \input{build/u_1.tex} && v = \input{v_1.tex} && w = \input{build/w_1.tex}
\end{align*}
und liefern einen Sättigungsstrom von $I_S = u = \input{build/u_1.tex}$. 

\begin{figure}[H]
	\includegraphics{build/plot_2.pdf}
	\caption{Messwerte und Regressionskurven für $I_H = \qty{2.2}{\ampere}$ und $U_H = \qty{4.5}{\volt}$.}
	\label{fig:plot_2}
\end{figure}

Der Wendepunkt lässt sich grob bei $U_A = \qty{50}{\volt}$ einordnen. Für den Gültigkeitsbereich des Raumladungsgesetzes
ergeben sich damit
\begin{align*}
	a = \input{build/a_2.tex} && b = \input{b_2.tex}
\end{align*}
als Parameter nach \eqref{eqn:fit_1}. Weiter folgen aus \eqref{eqn:fit_2}
\begin{align*}
	u = \input{build/u_2.tex} && v = \input{v_2.tex} && w = \input{build/w_2.tex}
\end{align*}
und liefern einen Sättigungsstrom von $I_S = u = \input{build/u_2.tex}$. 

\begin{figure}[H]
	\includegraphics{build/plot_3.pdf}
	\caption{Messwerte und Regressionskurven für $I_H = \qty{2.4}{\ampere}$ und $U_H = \qty{5.0}{\volt}$.}
	\label{fig:plot_3}
\end{figure}

Der Wendepunkt lässt sich grob bei $U_A = \qty{100}{\volt}$ einordnen. Für den Gültigkeitsbereich des Raumladungsgesetzes
ergeben sich damit
\begin{align*}
	a = \input{build/a_3.tex} && b = \input{b_3.tex}
\end{align*}
als Parameter nach \eqref{eqn:fit_1}. Weiter folgen aus \eqref{eqn:fit_2}
\begin{align*}
	u = \input{build/u_3.tex} && v = \input{v_3.tex} && w = \input{build/w_3.tex}
\end{align*}
und liefern einen Sättigungsstrom von $I_S = u = \input{build/u_3.tex}$. 

\subsection{Kathodentemperatur unter maximaler Heizleistung}

\begin{table}[H]
	\caption{Anodenstrom $I_A$ zur Gegenfeldspannung $U_{-A}$ an der Anode der Hochvakuumdiode bei maximaler Heizleistung mit
			 $I_H = \qty{2.5}{\ampere}$ und $U_H = \qty{5.5}{\volt}$. Hervorgehobene Werte werden wegen unzuverlässiger Anzeige
			 für weitere Rechnungen ausgeschlossen.}
	\centering
	\input{build/table_2.tex}
	\label{tab:table_2}
\end{table}

\begin{figure}[H]
	\includegraphics{build/plot_4.pdf}
	\caption{}
	\label{fig:plot_4}
\end{figure}

\begin{equation*}
	I = p \exp(qU)
\end{equation*}

\begin{align*}
	p = \input{build/p.tex} && q = \input{q.tex}
\end{align*}
\begin{equation*}
	T = \input{build/T.tex}
\end{equation*}

\subsection{Austrittsarbeit des Kathodenmaterials}

