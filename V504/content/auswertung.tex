% Messwerte: Alle gemessenen Größen tabellarisch darstellen
% Auswertung: Berechnung geforderter Ergebnisse mit Schritten/Fehlerformeln/Erläuterung/Grafik (Programme)
\section{Auswertung}
\label{sec:auswertung}

\subsection{Sättigungsstrom und Gültigkeit des Raumladungsgesetzes}

Wie beschrieben wird die in Tabelle \ref{tab:table_1} nachgehaltene Kennlinienschar aufgenommen. Anhand dieser Daten
wird der Wendepunkt als Übergang vom Raumladungs- zum Sättigungsstrombereich abgeschätzt. Die so aufgeteilten Messwerte, deren
Schnittmenge nur noch diesen approximierten Wendepunkt enthält, werden weiter separat unter Python~\cite{python} ausgewertet. Zur
Untersuchung des Exponenten $b$ der Strom-Spannungs-Beziehung bietet sich eine Funktion der Form
\begin{equation}
    I = a \frac{4}{9} \xi_0 \sqrt{2 e_0/m_0} U^b ,
	\label{eqn:fit_1}
\end{equation}
nach dem Langmuir-Schottkyschen Raumladungsgesetz~\eqref{eqn:langmuir} an, wobei der hier dimensionslose Skalierungsfaktor $a$
als zusätzlicher Freiheitsgrad dient. Zur Modellierung des Sättigungsstroms wird eine Beziehung
\begin{equation}
	I = u + \frac{v}{U^w}
	\label{eqn:fit_2}
\end{equation}
mit deren Asymptote $u$ genutzt, wobei $v < 0$ und $w > 0$ gilt. Die Parameter der passenden Ausgleichrechnungen lassen sich mittels
\verb+scipy.optimize.curve_fit+ bestimmen, wobei die Streuung über die Quadratwurzel der Diagonalelemente der
Kovarianzmatrix gegeben ist. Zur grafischen Darstellung von Messdaten und Regression wird Matplotlib~\cite{matplotlib} verwendet,
die Bibliothek Uncertainties~\cite{uncertainties} erlaubt eine automatisierte Fehlerfortpflanzung.

\begin{table}[H]
	\caption{Anodenstrom $I_A$ einer Hochvakuumdiode bei entsprechender Saugspannung $U_A$ unter Variation von
			 Heizstrom $I_H$ und -spannung $U_H$ an der Glühkathode. Hervorgehobene Werte stammen aus nachträglicher
			 Messung zur genaueren Untersuchung des Raumladungsstromgebiets.}
	\centering
	\input{build/table_1.tex}
	\label{tab:table_1}
\end{table}

\begin{figure}[H]
	\includegraphics{build/plot_1.pdf}
	\caption{Messwerte und Regressionskurven der Kennlinie für $I_H = \qty{2.0}{\ampere}$ und $U_H = \qty{3.5}{\volt}$.}
	\label{fig:plot_1}
\end{figure}

Der Wendepunkt lässt sich grob bei $U_A = \qty{20}{\volt}$ einordnen. Für den Gültigkeitsbereich des Raumladungsgesetzes
ergeben sich damit
\begin{align*}
	a = \input{build/a_1.tex} && b = \input{b_1.tex}
\end{align*}
als Parameter nach \eqref{eqn:fit_1}. Weiter folgen aus \eqref{eqn:fit_2}
\begin{align*}
	u = \input{build/u_1.tex} && v = \input{v_1.tex} && w = \input{build/w_1.tex}
\end{align*}
und liefern einen Sättigungsstrom von $I_S = u = \input{build/u_1.tex}$. 

\begin{figure}[H]
	\includegraphics{build/plot_2.pdf}
	\caption{Messwerte und Regressionskurven der Kennlinie für $I_H = \qty{2.2}{\ampere}$ und $U_H = \qty{4.5}{\volt}$.}
	\label{fig:plot_2}
\end{figure}

Der Wendepunkt lässt sich grob bei $U_A = \qty{50}{\volt}$ einordnen. Für den Gültigkeitsbereich des Raumladungsgesetzes
ergeben sich damit
\begin{align*}
	a = \input{build/a_2.tex} && b = \input{b_2.tex}
\end{align*}
als Parameter nach \eqref{eqn:fit_1}. Weiter folgen aus \eqref{eqn:fit_2}
\begin{align*}
	u = \input{build/u_2.tex} && v = \input{v_2.tex} && w = \input{build/w_2.tex}
\end{align*}
und liefern einen Sättigungsstrom von $I_S = u = \input{build/u_2.tex}$. 

\begin{figure}[H]
	\includegraphics{build/plot_3.pdf}
	\caption{Messwerte und Regressionskurven der Kennlinie für $I_H = \qty{2.4}{\ampere}$ und $U_H = \qty{5.0}{\volt}$.}
	\label{fig:plot_3}
\end{figure}

Der Wendepunkt lässt sich grob bei $U_A = \qty{100}{\volt}$ einordnen. Für den Gültigkeitsbereich des Raumladungsgesetzes
ergeben sich damit
\begin{align*}
	a = \input{build/a_3.tex} && b = \input{b_3.tex}
\end{align*}
als Parameter nach \eqref{eqn:fit_1}. Weiter folgen aus \eqref{eqn:fit_2}
\begin{align*}
	u = \input{build/u_3.tex} && v = \input{v_3.tex} && w = \input{build/w_3.tex}
\end{align*}
und liefern einen Sättigungsstrom von $I_S = u = \input{build/u_3.tex}$. 

\newpage
\subsection{Kathodentemperatur unter maximaler Heizleistung}

Die Messwerte für den Anlaufstrom werden in Tabelle \ref{tab:table_2} aufgetragen. 

\begin{table}[H]
	\caption{Anodenstrom $I_A$ zur Gegenfeldspannung $U_{-A}$ an der Anode der Hochvakuumdiode bei maximaler Heizleistung mit
			 $I_H = \qty{2.5}{\ampere}$ und $U_H = \qty{5.5}{\volt}$. Die unten beschriebene Korrektur wird für $\hat{U}_{-A}$
			 vorgenommen. \mbox{Hervorgehobene} Werte werden wegen unzuverlässiger Anzeige für weitere Rechnungen ausgeschlossen. }
	\centering
	\input{build/table_2.tex}
	\label{tab:table_2}
\end{table}

Zur Ausgleichsrechnung muss die Anodenspannung zunächst um den Einfluss des Innenwiderstands $R = \qty{1}{\mega\ohm}$ im
Nanoamperemeter bereinigt werden, es ist also $\hat U_{-A} = U_{-A} - RI_A$ als Korrektur zu berücksichtigen. 

\begin{figure}[H]
	\includegraphics{build/plot_4.pdf}
	\caption{Messwerte und Regressionskurve für den Anlaufstrom bei $I_H = \qty{2.5}{\ampere}$ und $U_H = \qty{5.5}{\volt}$. Es gilt
			 $U_A = -\hat{U}_{-A}$ für Saug- und Gegenspannung.}
	\label{fig:plot_4}
\end{figure}

Wie zuvor wird mittels SciPy \cite{scipy} eine Regressionsrechnung durchgeführt. Dazu dient
\begin{equation*}
	I = p \exp(q U_{-A})
\end{equation*}
als Modell, entlang dem die Daten verlaufen. Es ergeben sich
\begin{align*}
	p = \input{build/p.tex} && q = \input{q.tex}
\end{align*}
als fehlerbehaftete Koeffizienten. Nach \eqref{eqn:anlauf_exp} ist dann $q = -\frac{e_0}{kT}$, sodass
\begin{equation*}
	T = -\frac{e_0}{qk} = \input{build/T.tex}
\end{equation*}
die Temperatur der Kathode bemisst, wobei Elementarladung $e_0 = \qty{1.6e-19}{\coulomb}$ und Boltzmann-Konstante
$k = \qty{1.38e-23}{\joule\per\kelvin}$ aus der Datenbank \verb+scipy.constants+ entnommen werden.

\subsection{Austrittsarbeit des Kathodenmaterials}

Aus einer Leistungsbilanz des Heizstromfadens lässt sich die Kathodentemperatur errechnen. Die zugeführte Leistung
$N_Z = I_H U_H$ wird von der Apparatur über Wärmeleitung mit $N_{WL} \approx \qty{1}{\watt}$ und nach dem
Stefan-Boltzmannschen Gesetz über Wärmestrahlung mit $N_{WS} = f \eta \sigma T^4$ abgegeben, wobei
$\sigma = \qty{5.7e-12}{\watt\per\square\centi\meter\kelvin\tothe{-4}}$ die Stefan-Boltzmannsche
Strahlungskonstante bezeichnet. Apparatspezifisch ist mit $f = \qty{0.32}{\centi\meter\squared}$ die Oberfläche und mit
$\eta = \num{0.28}$ der Emissionsgrad der Kathode gegeben. Damit folgt
\begin{equation*}
	N_Z = I_H U_H = f \eta \sigma T^4 + N_{WL} = N_{WS} + N_{WL}
\end{equation*}
aus dem Energiesatz, womit über
\begin{equation*}
	T = \left( \frac{I_H U_H - N_{WL}}{f \eta \sigma} \right)^{\!\! 1/4}
\end{equation*}
die Kathodentemperatur in Tabelle \ref{tab:table_3} bestimmt werden kann. Weiter lässt sich durch Umstellen der
Richardson-Gleichung~\eqref{eqn:richard} nach $e_0 \phi$ die Austrittsarbeit
\begin{equation*}
	W = e_0 \phi = -kT \, \ln \!\left( \frac{j_S h^3}{4\pi e_0 m_0 k^2 T^2} \right)
	= -kT \, \ln \!\left( \frac{I_S h^3}{4\pi e_0 m_0 f k^2 T^2} \right)
\end{equation*}
berechnen. Dabei ist $h = \qty{6.626e-34}{\joule\second}$ die Planck-Konstante,
\begin{equation*}
	j_S = \frac{I_S}{f}
\end{equation*}
drückt die Sättigungsstromdichte durch den Sättigungsstrom aus. Die Ergebnisse sind ebenfalls in Tabelle \ref{tab:table_3} eingetragen.

\begin{table}[H]
	\caption{Die Temperatur $T$ der Kathode in der Hochvakuumdiode wird aus dem zugehörigem Heizstrom $I_H$ und der passenden
			 Heizspannung $U_H$ berechnet. Mit dem entsprechenden Sättigungsstrom $I_S$ ergibt sich die Austrittsarbeit~$W$ für
			 das verwendete Kathodenmaterial Wolfram.}
	\centering
	\input{build/table_3.tex}
	\label{tab:table_3}
\end{table}

Um Fehlereinflüsse zu kompensieren wird für die Austrittsarbeit der nach der inversen Varianz gewichtete Mittelwert
\begin{equation*}
	x_w = \frac{\sum_i w_i x_i}{\sum_i w_i} = \frac{\sum_i x_i / \sigma_i^2}{\sum_i 1/\sigma_i^2}
\end{equation*}
mit der dazugehörigen Abweichung
\begin{equation*}
	\increment x_w = \sqrt{\frac{1}{\sum_i w_i}} = \sqrt{\frac{1}{\sum_i 1/\sigma_i^2}}
\end{equation*}
aufgestellt. Dieser ist schließlich mit $W = \input{build/W.tex}$ gegeben.

