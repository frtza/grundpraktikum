% Theorie: Physikalische Grundlagen von Versuch/Messverfahren, Gleichungen ohne Herleitung knapp erklären
\section{Theorie}
\label{sec:theorie}

Im Folgenden werden grundlegende Begriffe des Versuches, wie die Austrittsarbeit, die Hochvakuum-Diode und dessen Kennlinie und Sättigungsstromdichte erläutert.
Zudem wird beschrieben und teilweise hergeleitet, wie rechnerisch die Austrittsarbeit und die Kathodentemperatur bestimmt werden kann.

\subsection{Austrittsarbeit und Energieverteilung der Leitungselektronen}
\label{sec:Begriffe der Austrittsarbeit und die Energieverteilung der Leitungselektronen}

Eine große Anzahl der Metalle sind kristalline Festkörper, welche hohe elektrische Leitfähigkeit besitzen.
Diese Tatsache lässt sich damit erklären, dass die Atome, welche auf den Kristallgitterplätzen sitzen, alle ionisiert sind.
Somit bilden die Ionen ein periodisches Gitter, welches von freigesetzten Elektronen eingehüllt ist.
Diese Elektronen befinden sich im Kraftfeld sämtlicher Ionen und werden als Leitungselektronen bezeichnet.
Das Gitterpotential ist eine vom Ort abhängige periodische Funktion. Diese nimmt an den Gitterpunkten einen hohen positiven Wert an,
weiter entfernt von den Punkten ist der Wert des Gitterpotentials nur wenig veränderlich. Somit lässt sich durch eine
Näherung sagen, dass das Gitterpotential konstant ist. Zusammenfassend kann gesagt werden, dass im Metallinneren ein konstantes 
positvives Potential, welches um $\phi$ verschieden zum Außenraum ist, herrscht. Die Elektronen können sich daher 
frei bewegen und demnach die elektrische Leitfähigkeit erzeugen.

Wenn ein Elektron das Metallinnere verlassen will, muss dieses die Austrittsarbeit zu dem gegebenen Potential $\xi$ leisten.
In \autoref{fig:potentialtopf} wird dies anhand des Potentialtopf-Modells gezeigt.
% hier abbildung nutzen
\begin{figure}[H]
    \centering
    \includegraphics[width=0.5\linewidth]{content/grafik/potential.png}
    \caption{Darstellung des Potentialtopf-Modells eines Metalls.\cite{elektron}}
    \label{fig:potentialtopf}
\end{figure}

Die Quantentheorie beantwortet die Frage, ob das Elektron die benötigte Energie aufbringen kann.
Elektronen können ausschließlich diskrete Energiewerte annehmen. Das Elektron hat einen halbzahligen Spin und 
unterliegt demnach dem Pauli-Verbot. Dieses besagt, dass jeder mögliche Zustand mit der vorausgesetzten Energie $E$ nur von zwei
Elektronen eingenommen werden kann, wenn diese entgegengesetzten Spin haben. Somit besitzen die Elektronen auch am Nullpunkt
eine endlich Energie. Diese ist abhängig von den Elektronen pro Volumeneinheit im Metall. Der Begriff für diese Energie bei $T = 0$
wird Fermische Grenzenergie $\zeta $ genannt. Für Zimmertemperatur gilt für alle Metalle $\zeta \gg \symup{k}\symup{T}$.
Durch die Fermi-Diracsche Verteilungs Funktion 
\begin{equation}
    f(E)= \frac{1}{\exp{\frac{E -\zeta}{\symup{k}T}}+1} ,
    \label{eqn:fermi-dis}
\end{equation}
wird die Wahrscheinlichkeit angegeben, dass im thermischen Gleichgewicht der Zustand
mit der Energie $E$ besetzt ist.
Der Verlauf des Graphen der Fermi-Diracschen Verteilungsfunktion ist in \autoref{fig:fermi} zu sehen.

\begin{figure}[H]
    \centering
    \includegraphics[width=0.5\linewidth]{content/grafik/fermi.png}
    \caption{Der Verlauf der Fermi-Diracschen Verteilungsfunktion am absoluten Nullpunkt \qty{0}{\kelvin} und bei einer Temperatur
			 $T \gg \qty{0}{\kelvin}$.\cite{elektron}}
    \label{fig:fermi}
\end{figure}

Es kann abgelesen werden, dass ein Elektron mindestens eine Energie von $\zeta + e_0 \phi$ vorweisen muss, damit 
es die Metalloberfläche verlassen kann. Für den Fall, dass das gegebene Metall Wolfram ist, kann eine
Näherung getroffen werden
\begin{equation}
    f(E) \approx \exp{\left(\frac{\zeta - E}{\symup{k}T}\right)} .
\label{eqn:naehrung}
\end{equation}

\subsection{Sättigungsstromdichte bei thermischer Elektronenemission}
\label{sec:Berechnung der Sättigungsstromdichte bei der thermischen Elektonenemission}

Mithilfe der \autoref{eqn:naehrung} lässt sich die Sättigungsstromdichte in Abhängigkeit zur Temperatur errechnen.
Schlussendlich folgt für die Sättigungsstromdichte $j_S(T)$ die Richardson-Gleichung
\begin{equation}
    j_S(T) = 4\symup{\pi} \frac{e_0\cdot m_0 \cdot \symup{k}^2}{h^3} T^2 \exp{\frac{-e_0 \phi}{\symup{k}T}}.
	\label{eqn:richard}
\end{equation}

\subsection{Die Hochvakuum-Diode}
\label{sec:Die Hochvakuum-Diode}

Um den Sättigungsstrom einer emittierenden Metalloberfläche zu messen, muss ein Hochvakuum vorliegen, da sonst die
Elektronen mit den Gasmolekülen wechselwirken würden. Weiter wird ein elektrisches Feld benötigt, welches die ausgetretenen Elektronen
absaugt. Diese dafür vorgesehene Apperatur heißt Hochvakuum-Diode. Die Schaltskizze einer solchen Apperatur ist in \autoref{fig:diode} 
dargestellt.

\begin{figure}[H]
    \centering
    \includegraphics[width=0.5\linewidth]{content/grafik/diode.png}
    \caption{Die grundlegende Schaltskizze einer Hochvakuum-Diode.\cite{elektron}}
    \label{fig:diode}
\end{figure}

In der dargestellten Konfiguration fließt durch die emittierten Elektronen ein Strom von Kathode zu Anode. Bei umgekehrter Polung
wird dies durch das Gegenfeld verhindert, im Vakuum sind keine sonstigen Ladungsträger vorhanden. Entsprechend ist
die Bezeichnung als Diode gerechtfertigt, die je nach Betriebsrichtung eine leitende oder sperrende Funktion erfüllt.

\subsection{Die Langmuir-Schottkysche Raumladungsgleichung}
\label{sec:Die Langmuir-Schottkysche Raumladungsgleichung}

Die Elektronen führen eine beschleunigte Bewegung in Richtung der Anode aus, haben also keine konstante Geschwindigkeit.
Dementsprechend ist die Raumladungsdichte $\rho$ der Elektronen eine Funktion des Ortes, welche zur Anode hin abnimmt. Diese
Tatsache lässt sich aus der Kontinuitätsgleichung, dass $j$ überall konstant ist, ableiten. Die Stromdichte ist gegeben durch
\begin{equation}
    j = - \rho v .
\label{eqn:konti}
\end{equation}
Die Raumladungsdichte $\rho$ beeinflusst daher den Verlauf der Feldstärke zwischen Anode und Kathode.
Sie schirmt das Feld von der Kathode ab. Die emittierten Elektronen werden dann nicht mehr alle von dem
Anodenfeld erfasst. Darauf folgt, dass der zu messende Kathodenstrom kleiner als der zu erwartende Sättigungsstrom ist.
In der \autoref{eqn:poisson} ist der Zusammenhang von Anodenspannung und -strom in der Poissongleichung dargestellt
\begin{equation}
    \increment V = - \frac{1}{\varepsilon_0}\rho .
    \label{eqn:poisson}
\end{equation} 
Angenommen wird, dass die Anode und Kathode unendlich ausgedehnte ebene Oberflächen sind, welche
mit einem Abstand $a$ zueinander ausgerichtet sind.
Zusammen aus \autoref{eqn:konti} und \autoref{eqn:poisson} folgt der Zusammenhang
zwischen der Stromdichte $j$ und der Anodenspannung $V$
\begin{equation}
    j = \frac{4}{9} \xi_0 \sqrt{2 e_0/m_0} \frac{V^{\frac{3}{2}}}{a^2} ,
\label{eqn:langmuir}
\end{equation}
welcher als Langmuir-Schottkysches Raumladungsgesetz bezeichnet wird. Der Gültigkeitsbereich 
in einem $j$-$V$-Diagramm einer Hochvakuum-Diode wird Raumladungsgebiet genannt.

In \autoref{fig:raumladung} ist die Ortsabhängigkeit des Potentials, der Feldstärke und der Ladungsdichte im Raumladungsgebiet
eine Hochvakuumdiodenkennlinie aufgetragen.

\begin{figure}[H]
    \centering
    \includegraphics[width=0.5\linewidth]{content/grafik/raumladung.png}
    \caption{Die Darstellung der Ortsabhängigkeit des Potentials $V$, der Feldstärke $E$ und der Ladungsdichte $\rho$ im
    Raumladungsgebiet einer Hochvakuumdiodenkennlinie.\cite{elektron}}
    \label{fig:raumladung}
\end{figure}

\subsection{Das Anlaufstromgebiet einer Hochvakuum-Diode}
\label{sec:Das Anlaufstromgebite einer Hochvakuum-Diode}

Aus \autoref{eqn:langmuir} kann abgelesen werden, dass für $j = 0$ auch $V = 0$ gilt.
Durch die Eigengeschwindigkeit, die die Elektronen bei Verlassen der Kathode haben, kann bei $V = 0$
ein geringer Anodenstrom gemessen werden. Für $T > 0$ existieren endlich viele Elektronen, deren Energie größer ist 
als die Austrittsarbeit. Diese Energie 
\begin{equation}
    \increment E = E - \left(\zeta + e_0\phi\right),
\end{equation}    
ist dann die kinetische Energie der Elektronen. Dieser Strom wird als Anlaufstrom bezeichnet.
Das Energieverhältnis im Anlaufstromgebiet ist in \autoref{fig:anlauf} dargestellt.

\begin{figure}[H]
    \centering
    \includegraphics[width=0.42\linewidth]{content/grafik/anlauf.png}
    \caption{Die Potentialverhältnisse in einer Hochvakuum-Diode im Bereich ihres Anlaufstromgebietes.\cite{elektron}}
    \label{fig:anlauf}
\end{figure}

\enlargethispage*{\baselineskip}
\newpage

Da die Anzahl der Leitungselektronen mit Energien zwischen $E$ und $\symup dE$ nach \autoref{eqn:naehrung} näherungsweise
exponentiell von $E$ abhängig ist, lässt sich mit
\begin{equation}
	j(V) = j_0 \exp \left( -\frac{e_0 \phi_A + e_0 V}{kT} \right) = \text{const} \exp \left( -\frac{e_0 V}{kT} \right)
	\label{eqn:anlauf_exp}
\end{equation}
eine analoge Beziehung für Anlaufstromstärke und äußeres Potential $V$ formulieren.

\subsection{Die Kennlinie der Hochvakuumdiode}
\label{sec:Die Kennlinie der Hochvakuumdiode}

Die Kennlinie beschreibt den Zusammenhang der Stromdichte $j$ beziehungsweise dem Anodenstrom $I_A$ 
und dem außen angelegten Potential.

\begin{figure}[H]
    \centering
    \includegraphics[width=0.5\linewidth]{content/grafik/kennlinie.png}
    \caption{Die Kennlinie einer Hochvakuum-Diode.\cite{elektron}}
    \label{fig:kennlinie}
\end{figure}

Die \autoref{fig:kennlinie} zeigt eine solche Kennlinie einer Hochvakuum-Diode.  Diese lässt sich in drei Abschnitte 
unterteilen: Dem Anlaufstromgebiet, dem Raumladungsgebiet und dem Sättigungsstromgebiet. Das Anlaufstromgebiet ist durch einen
exponentiellen Anstieg gekennzeichnet, dieses Gebiet liegt im Bereich $V < 0$. Bei dem Raumladungsgebiet liegt eine
$\sqrt{V^3}$- Abhängigkeit vor. Schlussendlich wird das Raumladungsgebiet langsam von dem Sättigungsstromgebiet abgelöst. 
Mithilfe dieser Kennlinie kann die Kathodentemperatur und die Austrittsarbeit der Kathode bestimmt werden.
