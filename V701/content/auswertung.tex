% Messwerte: Alle gemessenen Größen tabellarisch darstellen
% Auswertung: Berechnung geforderter Ergebnisse mit Schritten/Fehlerformeln/Erläuterung/Grafik (Programme)
\newpage
\section{Auswertung}
\label{sec:auswertung}

\subsection{Reichweite und Energie}

Die Ergebnisse der ersten Messreihe, für die das verwendete Präparat in einem Abstand von \qty{6}{\centi\meter} zum
Halbleiter-Sperrschichtzähler befestigt wird, lassen sich anhand Tabelle~\ref{tab:6} nachvollziehen. Ein entsprechendes
Ventil an der Vakuumpumpe dient zur Variation des Zylinderdrucks $p$, der mithilfe des integrierten Manometers eingestellt
wird. Daraus ergibt sich nach Formel~\eqref{eqn:effektiv} die effektive Distanz $x$. Mit $N_\text{tot}$ ist die Anzahl aller
aufgezeichneten Strompulse innerhalb eines Intervalls von \qty{120}{\second} bezeichnet. Für eine gegebene Einstellung zählt
$N_\text{max}$ über denselben Zeitraum die maximale Anzahl der Impulse innerhalb eines Kanals. Dieser die meisten Signale
erhaltende Kanal wird unter $\text{CH}$ aufgeführt. Im Vielkanalanalysator kann davon ausgangen werden, dass ein linearer
Zusammenhang zwischen Kanalnummer und entsprechender Energie des Alphateilchens existiert. So wird dann der Modus $E$ der
Energieverteilung ermittelt, da $E = \qty{4}{\mega\electronvolt}$ für $p = 0$ oder äquivalent für $x = 0$ bekannt ist. 
Der statistische Modus, auch Modalwert gennant, gibt den Wert an, der am häufigsten in einer Stichprobe vorkommt.

\begin{table}[H]
	\centering
	\caption{Messdaten bei festem Abstand $x_0 = \qty{6}{\centi\meter}$ zwischen Probe und Detektor.}
	\input{build/table_6.tex}
	\label{tab:6}
\end{table}

Zur weiteren Auswertung des Verlaufs der Zählrate $N_\text{tot}$ in Abhängigkeit zur effektiven Länge $x$ bietet sich die
Verwendung einer Sigmoidfunktion
\begin{equation}
	\symup{sig}(t) = \pfrac{a}{1 + \exp \left( b(t - c) \right)} + d
	\label{eqn:sig}
\end{equation}
an, welche hier mit vier Freiheitsgraden formuliert ist. Ihr einziger Wendepunkt liegt bei $t = c$, dort nimmt sie den Funktionswert
$\symup{sig}(c) = \raisebox{0.2ex}{\scalebox{0.8}{\(\pfrac{1}{2}\)}} \hspace{0.1ex} a + d$ an. Dieser liegt genau mittig zwischen
den Asymptoten $d$ und $a + d$.

Aus dieser Tatsache folgt speziell für das Modell der Zählrate, dass bei einer effektiven Länge $x = c$ noch genau die Hälfte der
maximalen Pulszahl den Detektor erreicht, womit $R = c$ ein Maß für die mittlere Reichweite von Alphastrahlung in Luft ist.

\enlargethispage*{\baselineskip}
\newpage

Alternativ lässt sich $R$ über die tatsächlich gemessene maximale Zählrate $\hat{N}_\text{tot}$ bestimmen, indem
$t = \raisebox{0.2ex}{\scalebox{0.8}{\(\pfrac{1}{2}\)}} \hspace{0.1ex} \hat{N}_\text{tot}$ in die Inverse von \eqref{eqn:sig}
eingesetzt wird. Diese ist mit
\begin{equation}
	\symup{sig}^{-1}(t) = \pfrac{\ln \left( \displaystyle{\pfrac{a}{t - d}} - 1 \right)}{b} + c
	\label{eqn:inv}
\end{equation}
gegeben. In Abbildung~\ref{fig:qty_6} werden die zuvor aufgetragenen Messwerte mit der beschriebenen Ausgleichskurve
durch die Bibliothek Matplotlib~\cite{matplotlib} unter Python~\cite{python} dargestellt.

\begin{figure}[H]
	\includegraphics{build/plot_qty_6.pdf}
	\caption{Gesamtzählrate $N_\text{tot}$ der über einen Zeitraum von \qty{120}{\second} gemessenen Impulse in Abhängigkeit zur
			 effektiven Länge $x$ bei $x_0 = \qty{6}{\centi\meter}$.}
	\label{fig:qty_6}
\end{figure}

Die optimalen Parameter liefert die numerische Methode \verb+scipy.optimize.curve_fit+~\cite{scipy}, wobei die angegebenen Abweichungen
der Wurzel der Elemente auf der Hauptdiagonalen der Kovarianzmatrix entsprechen. So ergeben sich
\begin{align*}
	a &= \input{build/a_6.tex} & b &= \input{build/b_6.tex} \\
	c &= \input{build/c_6.tex} & d &= \input{build/d_6.tex}
\end{align*}
zur Minimierung der Fehlerquadrate. $R = \input{build/c_6.tex}$ beschreibt die mittlere Reichweite über den Wendepunkt $c$ mit
$E = \input{build/E_val_6.tex}$ als die dazugehörige Energie. Um letztere zu berechnen, wird $R = \num{3.1} E^{\, 3/2}$ zur
Näherung mit Gültigkeit für $E \leq \qty{2.5}{\mega\electronvolt}$ ausgenutzt, wobei $R$ in \unit{\milli\meter} anzugeben ist.
Mit Uncertainties~\cite{uncertainties} erfolgt dazu eine automatisierte Fehlerfortpflanzung.

Die Hälfte der maximalen Zählrate innerhalb von \qty{120}{\second} lautet
$\raisebox{0.2ex}{\scalebox{0.8}{\(\pfrac{1}{2}\)}} \hspace{0.1ex} \hat{N}_\text{tot} = \input{build/h_6.tex}$
und liefert über die Umkehrfunktion \eqref{eqn:inv} den Wert $R = \input{build/R_6.tex}$ für die mittlere Reichweite bei
einer Energie von $E = \input{build/E_R_6.tex}$. 

Zur Regression entlang der Modalwerte $E$ bei Länge $x$ wird ein linearer Zusammenhang
\begin{equation}
	E(x) = w - v x
	\label{eqn:lin}
\end{equation}
herangezogen. Daran lässt sich direkt der Term
$-\raisebox{0.1ex}{\scalebox{0.8}{\(\pfrac{\symup dE}{\symup dx}\)}} = v$
für den Energieverlust ablesen.

\begin{figure}[H]
	\includegraphics{build/plot_E_6.pdf}
	\caption{Modus $E$ der Energieverteilung zum effektiven Abstand $x$ bei $x_0 = \qty{6}{\centi\meter}$.}
	\label{fig:E_6}
\end{figure}

In Abbildung~\ref{fig:E_6} werden Messpunkte und lineare Ausgleichsrechnung angezeigt. Die Funktion \verb+numpy.polyfit+ produziert
\begin{align*}
	v = \input{build/v_6.tex} && w = \input{build/w_6.tex}
\end{align*}
als Koeffizienten der optimalen Näherung. Mit
$-\raisebox{0.1ex}{\scalebox{0.8}{\(\pfrac{\symup dE}{\symup dx}\)}} = \input{build/v_6.tex}$ folgt daraus die pro Streckeneinheit
abgegebene Energie. 

Analog zum bisherigen Vorgehen für $x = \qty{6}{\centi\meter}$ werden nun die Ergebnisse für einen weiteren Abstand
$x_0 = \qty{4}{\centi\meter}$ betrachtet. Die entsprechenden Messwerte dazu sind in Tabelle~\ref{tab:4} eingetragen.

\begin{table}[H]
	\centering
	\caption{Messdaten bei festem Abstand $x_0 = \qty{4}{\centi\meter}$ zwischen Probe und Detektor.}
	\input{build/table_4.tex}
	\label{tab:4}
\end{table}

Abbildung~\ref{fig:qty_4} stellt die Zählraten sowie den Graphen der optimierten Fit-Funktion dar.

\enlargethispage*{\baselineskip}

\begin{figure}[H]
	\includegraphics{build/plot_qty_4.pdf}
	\caption{Gesamtzählrate $N_\text{tot}$ der über einen Zeitraum von \qty{120}{\second} gemessenen Impulse in Abhängigkeit zur
			 effektiven Länge $x$ bei $x_0 = \qty{4}{\centi\meter}$.}
	\label{fig:qty_4}
\end{figure}

\newpage

Die der Sigmoidfunktion~\eqref{eqn:sig} zugehörigen Parameter lauten jetzt
\begin{align*}
	a &= \input{build/a_4.tex} & b &= \input{build/b_4.tex} \\
	c &= \input{build/c_4.tex} & d &= \input{build/d_4.tex}
\end{align*}
und führen über den Wendepunkt auf eine mittlere Reichweite $R = \input{build/c_4.tex}$ sowie auf die Energie
$E = \input{build/E_val_4.tex}$ als abgeleitete Größe. Mit
$\raisebox{0.2ex}{\scalebox{0.8}{\(\pfrac{1}{2}\)}} \hspace{0.1ex} \hat{N}_\text{tot} = \input{build/h_4.tex}$
ergibt sich nach Einsetzen in Zusammenhang \eqref{eqn:inv} der Wert $R = \input{build/R_4.tex}$ mit $E = \input{build/E_R_4.tex}$
für Reichweite und Energie. 

\begin{figure}[H]
	\includegraphics{build/plot_E_4.pdf}
	\caption{Modus $E$ der Energieverteilung zum effektiven Abstand $x$ bei $x_0 = \qty{4}{\centi\meter}$.}
	\label{fig:E_4}
\end{figure}

In Abbildung~\ref{fig:E_4} sind die Moden der Energie und eine dazu passende lineare Regression nach
Ansatz~\eqref{eqn:lin} mit optimalen Steuerwerten
\begin{align*}
	v = \input{build/v_4.tex} && w = \input{build/w_4.tex}
\end{align*}
wiedergegeben. Daraus folgt der Energieverlust
$-\raisebox{0.1ex}{\scalebox{0.8}{\(\pfrac{\symup dE}{\symup dx}\)}} = \input{build/v_4.tex}$.

\subsection{Zerfallsstatistik}

\begin{table}[H]
	\centering
	\caption{Totale Impulszählrate $N_\text{tot}$ über einen Zeitraum von \qty{10}{\second} bei Parametern
			 $x_0 = \qty{4}{\centi\meter}$ und $p = \qty{300}{\milli\bar}$. Das entspricht einem Abstand
			 $x = \qty{1.18}{\centi\meter}$ bei Normaldruck. Aufgeführt werden $n_\text{tot} = 100$ Messungen,
			 die zur besseren Nachvollziehbarkeit aufsteigend sortiert sind.}
	\input{build/table_st.tex}
	\label{tab:st}
\end{table}

\begin{figure}[H]
	\includegraphics{build/plot_st.pdf}
	\caption{Histogramm der gemessenen relativen Zählratenverteilung ($n_\text{tot} = 100$) mit überlagerter
			 Gauß- und Poisson-Verteilung (je $n_\text{tot} = 500$).}
	\label{fig:st}
\end{figure}

\begin{equation*}
	P_{\mu \, \sigma}(x) = \pfrac{1}{\sqrt{2\pi} \sigma} \exp \left( -\pfrac{(x - \mu)^2}{2\sigma^2} \right)
\end{equation*}

\begin{equation*}
	P_\lambda (k) = \pfrac{\lambda^k}{k!} \exp \left( -k \right)
\end{equation*}

\begin{equation*}
	\overline{x} = \pfrac{1}{N \,} \sum_{n=1}^N x_n
\end{equation*}

\begin{equation*}
	(\symup{\Delta}\overline{x})^2 = \pfrac{1}{N(N-1)} \sum_{n=1}^N (x_n \! - \overline{x})^2
\end{equation*}

\begin{equation*}
	(\symup{\Delta}f)^2 = \sum_{n=1}^N
	\left( \! \pfrac{\partial^{\!} f}{\partial x_{\raisebox{0.2ex}{$\scriptstyle{n}$}}} \!
	\right)^{\!\! 2} \!\! (\symup{\Delta}x_{\raisebox{0.2ex}{$\scriptstyle{n}$}})^2
\end{equation*}

$\,\,\overline{\!\! N}_\text{tot} = \input{build/mean.tex}$

$\left( \increment \,\,\overline{\!\! N}_\text{tot} \right)^{\! 2} = \input{build/var.tex}$

$\increment \,\,\overline{\!\! N}_\text{tot} = \input{build/std.tex}$
