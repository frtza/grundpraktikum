% Messwerte: Alle gemessenen Größen tabellarisch darstellen
% Auswertung: Berechnung geforderter Ergebnisse mit Schritten/Fehlerformeln/Erläuterung/Grafik (Programme)
\section{Auswertung}
\label{sec:auswertung}

\begin{table}[H]
	\centering
	\caption{}
	\input{build/table_6.tex}
	\label{tab:6}
\end{table}

\begin{figure}[H]
	\includegraphics{build/plot_qty_6.pdf}
	\caption{}
	\label{fig:qty_6}
\end{figure}

\begin{equation*}
	\symup{sig}(x) = \pfrac{a}{1 + \exp \left( b(x - c) \right)} + d
\end{equation*}

\begin{align*}
	a &= \input{build/a_6.tex} & b &= \input{build/b_6.tex} \\
	c &= \input{build/c_6.tex} & d &= \input{build/d_6.tex}
\end{align*}

$R = \input{build/c_6.tex}$

$E = \input{build/E_val_6.tex}$

\begin{equation*}
	\symup{sig}^{-1}(x) = \pfrac{\ln \left( \displaystyle{\pfrac{a}{x - d}} - 1 \right)}{b} + c
\end{equation*}

$\pfrac{1}{2} \hat{N}_\text{tot} = \input{build/h_6.tex}$

$R = \input{build/R_6.tex}$

$E = \input{build/E_R_6.tex}$

\begin{figure}[H]
	\includegraphics{build/plot_E_6.pdf}
	\caption{}
	\label{fig:E_6}
\end{figure}

\begin{equation*}
	E(x) = w - v x
\end{equation*}

\begin{align*}
	v = \input{build/v_6.tex} && w = \input{build/w_6.tex}
\end{align*}

$-\pfrac{\symup dE}{\symup dx} = \input{build/v_6.tex}$

\begin{table}[H]
	\centering
	\caption{}
	\input{build/table_4.tex}
	\label{tab:4}
\end{table}

\begin{figure}[H]
	\includegraphics{build/plot_qty_4.pdf}
	\caption{}
	\label{fig:qty_4}
\end{figure}

\begin{align*}
	a &= \input{build/a_4.tex} & b &= \input{build/b_4.tex} \\
	c &= \input{build/c_4.tex} & d &= \input{build/d_4.tex}
\end{align*}

$R = \input{build/c_4.tex}$

$E = \input{build/E_val_4.tex}$

$\pfrac{1}{2} \hat{N}_\text{tot} = \input{build/h_4.tex}$

$R = \input{build/R_4.tex}$

$E = \input{build/E_R_4.tex}$

\begin{figure}[H]
	\includegraphics{build/plot_E_4.pdf}
	\caption{}
	\label{fig:E_4}
\end{figure}

\begin{align*}
	v = \input{build/v_4.tex} && w = \input{build/w_4.tex}
\end{align*}

$-\pfrac{\symup dE}{\symup dx} = \input{build/v_4.tex}$

\begin{table}[H]
	\centering
	\caption{}
	\input{build/table_st.tex}
	\label{tab:st}
\end{table}

\begin{figure}[H]
	\includegraphics{build/plot_st.pdf}
	\caption{}
	\label{fig:st}
\end{figure}

\begin{equation*}
	P_{\mu \, \sigma}(x) = \pfrac{1}{\sqrt{2\pi} \sigma} \exp \left( -\pfrac{(x - \mu)^2}{2\sigma^2} \right)
\end{equation*}

\begin{equation*}
	P_\lambda (k) = \pfrac{\lambda^k}{k!} \exp \left( -k \right)
\end{equation*}

$\,\,\overline{\!\! N}_\text{tot} = \input{build/mean.tex}$

$\sigma^2 = \input{build/var.tex}$

$\sigma = \input{build/std.tex}$
