% Messwerte: Alle gemessenen Größen tabellarisch darstellen
% Auswertung: Berechnung geforderter Ergebnisse mit Schritten/Fehlerformeln/Erläuterung/Grafik (Programme)
\section{Auswertung}
\label{sec:auswertung}

\begin{table}
	\centering
	\caption{}
	\input{build/table_6.tex}
	\label{tab:6}
\end{table}

\begin{figure}[H]
	\includegraphics{build/plot_qty_6.pdf}
	\caption{}
	\label{fig:qty_6}
\end{figure}

\begin{figure}[H]
	\includegraphics{build/plot_E_6.pdf}
	\caption{}
	\label{fig:E_6}
\end{figure}

\begin{table}
	\centering
	\caption{}
	\input{build/table_4.tex}
	\label{tab:4}
\end{table}

\begin{figure}[H]
	\includegraphics{build/plot_qty_4.pdf}
	\caption{}
	\label{fig:qty_4}
\end{figure}

\begin{figure}[H]
	\includegraphics{build/plot_E_4.pdf}
	\caption{}
	\label{fig:E_4}
\end{figure}

\begin{table}
	\centering
	\caption{}
	\input{build/table_st.tex}
	\label{tab:st}
\end{table}

\begin{figure}[H]
	\includegraphics{build/plot_st.pdf}
	\caption{}
	\label{fig:st}
\end{figure}

