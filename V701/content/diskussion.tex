% Diskussion: Resultate mit Fehler/Genauigkeit zusammenstellen, Literaturwerte/Messmethoden/Ursachen vergleichen
% Literatur: Verwendete Literatur/Grafiken/Werte/Programme
% Anhang: Kopie der analog eingetragenen Messdaten
\newpage
\section{Diskussion}
\label{sec:diskussion}

Die mittlere Reichweite für Alphateilchen in Luft ergibt sich zu $\input{build/c_6.tex}$ und $\input{build/R_6.tex}$ für
$x_0 = \qty{6}{\centi\meter}$ sowie $\input{build/c_4.tex}$ und $\input{build/R_4.tex}$ für $x_0 = \qty{4}{\centi\meter}$, wobei
jeweils die Methoden über Wendepunkt und Messung verwendet werden. Diese Werte weisen eine gute Übereinstimmung auf und liegen
größtenteils innerhalb der gegenseitigen Fehlerintervalle. Grafisch lässt sich am Verlauf der Sigmoid-Kurven in den
Abbildungen~\ref{fig:qty_6}~und~\ref{fig:qty_4} erkennen, dass sich die Funktion im relevanten Bereich nahezu geradlinig verhält.
Entsprechend könnte durch Einschränkung der zur Regression herangezogenen Messpunkte auch ein linearer Ansatz gewählt werden,
welcher durch reduzierte Komplexität und erhöhte numerische Stabilität potentiell zuverlässigere Lösungen produziert. Aufgrund
der hohen Übereinstimmung der Ergebnisse können mögliche Abweichungen zum realen Wert am wahrscheinlichsten auf unbekannte
Fehler innerhalb der Messapparatur zurückgeführt werden.

Zur Bestimmung der korrespondierenden Energien gelten die gleichen Methoden und Abstände. Für $x_0 = \qty{6}{\centi\meter}$
ergeben sich $\input{build/E_val_6.tex}$ und $\input{build/E_R_6.tex}$, für $x_0 = \qty{4}{\centi\meter}$ sind es
$\input{build/E_val_4.tex}$ und $\input{build/E_R_4.tex}$. Da es sich hier um abgeleitete Größen handelt, besitzen sie ähnlich
große Übereinstimmungen und Toleranzbereiche wie zuvor. Allerdings ist die verwendete Näherung $E = (\num{0.32} R)^{2/3}$
auf Energien bis $\qty{2.5}{\mega\electronvolt}$ beschränkt. Die Ergebnisse befinden sich also außerhalb des Gültigkeitsbereichs
und sind daher von fragwürdiger Aussagekraft.

Aus einer linearen Näherung ergibt sich der Energieverlust zu $\input{build/v_6.tex}$ für $x_0 = \qty{6}{\centi\meter}$ und
$\input{build/v_4.tex}$ für $x_0 = \qty{4}{\centi\meter}$. Diese Werte passen gut zu den Messergebnissen und weisen ebenso eine
hohe gegenseitige Übereinstimmung auf. Die Beobachtung, dass die Verlustrate für einen Abstand von $\qty{6}{\centi\meter}$ höher
ausfällt als bei $\qty{4}{\centi\meter}$, ist damit zu erklären, dass der Zylinderdruck zu Beginn der Messung für
$\qty{6}{\centi\meter}$ nicht in ausreichender Genauigkeit auf $p = 0$ geregelt ist. Eigentlich müsste der erste Energiewert
also bereits unter $\qty{4}{\mega\electronvolt}$ liegen, sodass insgesamt ein flacherer Verlauf mit geringerem Energieverlust
auftreten würde. Eine weitere Fehlerquelle könnte in der Annahme eines perfekten linearen Zusammenhangs zwischen Kanal und Energie
liegen, obwohl die Anordnung der Messungen diese weitestgehend zu bestätigen scheint. Die mit wachsendem effektiven Abstand zunehmende
Streuung der Messdaten, welche in den Abbildungen~\ref{fig:E_6}~und~\ref{fig:E_4} erkennbar ist, lässt sich auf die stark abfallenden
Zählraten und die damit einhergehende schlechte statistische Signifikanz der Modalwerte zurückführen.

Ein Vergleich der statistischen Verteilungen in Abbildung~\ref{fig:st} zeigt, dass der Verlauf der Messwerte in seiner Form
eher der Form einer Gaußschen Distribution besitzt. Die Poissonverteilung verläuft dagegen deutlich steiler. An dieser Stelle
sei angemerkt, dass der Sperrschichtzähler eine hohe Zeitauflösung im Bereich einiger $\unit{\nano\second}$ besitzt. Dennoch werden
manche Impulse der eindringenden Alphastrahlung unweigerlich nicht registriert, weshalb die Zählrate auch nicht direkt der Anzahl der
Zerfälle pro Zeiteinheit entspricht. Die relative Anzahl $n_\text{rel}$ sollte dadurch aufgrund der zufallsverteilten Signale jedoch
keine Verfälschung erfahren.
