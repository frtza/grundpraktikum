% Theorie: Physikalische Grundlagen von Versuch/Messverfahren, Gleichungen ohne Herleitung knapp erklären
\section[Theorie]{Theorie \textnormal{\cite{alpha}}}
\label{sec:theorie}

Durch das Messen der Reichweite von $\alpha$-Teilchen kann die Energie dieser bestimmt werden.
Die $\alpha$-Teilchen geben durch elastische Stöße mit dem Material Energie ab, dies spielt bei dem Energieverlust schlussendlich
nur eine untergeordnete Rolle. Die Teilchen können durch Anregung oder Dissoziation von Molkülen Energie verlieren.
Der Energieverlust $\frac{d E_\alpha}{d x}$ hängt von der Energie der $\alpha$-Teilchen und der Dichte des zu durchlaufenden 
Materials ab. Dabei ist zu beachten, dass bei kleinerer Geschwindigkeit die Wahrscheinlichkeit der Wechselwirkungen zunimmt.
Für hinreichend große Energien lässt sich der Energieverlust der $\alpha$-Teilchen über die Bethe-Bloch-Gleichung beschreiben 
\begin{equation}
    -\frac{d E_\alpha}{d x}=\frac{z^2 e^4}{4 \pi \epsilon_0 m_e} \frac{n Z}{v^2} \ln \left(\frac{2 m_e v^2}{I}\right),
    \label{eqn:bethe}
\end{equation}
wobei $z$ die Ladung der $\alpha$-Teilchen ist und $v$ die Geschwindigkeit dieser. $Z$ ist die Ordnungszahl, $n$ die Teilchendichte 
und $I$ die Ionisierungsenergie des Targetgases. Die \autoref{eqn:bethe} verliert an Gültigkeit, wenn das $\alpha$-Teilchen sehr 
kleine Energien hat.
Die Reichweite der $\alpha$-Teilchen, also die Strecke bis zur vollkommenen Ausbremsung, lässt sich über den Zusammenhang
\begin{equation}
    R=\int_0^{E_\alpha} \frac{d E_\alpha}{-d E_\alpha / d x}
    \label{eqn:reichweite}
\end{equation}
bestimmen.
Da bei niedrig werdender Energie die \autoref{eqn:bethe} nicht mehr gilt, werden zur Bestimmung der mittleren Reichweite
empirisch gewonnene Kurven verwendet. Für die mittlere Reichweite von $\alpha$-Strahlung in Luft mit der Energie $E_{\alpha} \preceq \SI{2.5}{MeV}$
kann die Bezeichnung $R_m = \SI{3.1} {E_{\alpha}}^{3/2}$ verwendet werden.
Bei einer konstanten Temperatur und konstantem Volumen ist die Reichweite von $\alpha$-Teilchen in Gasen proportional zum Druck $\rho$.
Dementsprechend kann eine Absorptionsmessung, bei der der Druck variiert wird, durchgeführt werden. Für einen festen
Abstand $x_0$ zwischen Detektor und $\alpha$-Strahler gilt für die effektive Länge $x$ 
\begin{equation}
    x = x_0 \frac{\rho}{\rho_0},
	\label{eqn:effektiv}
\end{equation}
wobei der Normaldruck mit $\rho_0 = \SI{1013}{m\bar}$ eingesetzt werden muss.

\subsection{Vorbereitungsaufgabe}
\label{sec:Vorbereitungsaufgabe}

Das Funktionsprinzip eines Halbleiterzählers basiert auf den Eigenschaften eines Halbleiters.
Werden n-dotiertes und p-dotiertes Material in Kontakt miteinander gebracht, entsteht durch Diffusion eine Zone, in 
der Ladungsträger freibeweglich sind. Dises erhält sich bis das elektrische Feld die Diffuison verhindert.
Wenn der n-Bereich mit einer Anode und der p-Bereich mit einer Kathode verbunden wird, vergrößert sich der Bereich
mit den freien Ladungsträgern. Dieser Bereich wird Sperrzone genannt.
Wenn ein ionisierendes Teilchen diese Sperrzone durchquert, werden Löcher und Elektronen erzeugt. Dabei entsteht
ein Stromfluss, welcher gemessen werden kann.
